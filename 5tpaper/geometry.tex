% !TEX root = main.tex

\section{Geometry of the problem}
\label{geometry}

In this Section we will show a proof of Theorem \ref{thm:octagonalization}. The key technical ingredient is Lemma \ref{lem:morpion_graphs}.
Let $\mathcal{M} = \mathcal{M}(G)$ be the set of all possible moves in a graph $G$. 
A lattice graph with a vertex set $V$ is \emph{full} if its edge set is maximal, i.e.
\[ 
  E = \{ (u,v) \colon  u, v \in V, u \neq v,   |u_x - v_x| \leq 1, | u_y - v_y | \leq 1 \}.
\]


A graph $G = (V, E)$ is \emph{$1$-connected} if a full lattice graph with a vertex set $V$ is connected.
A \emph{boundary} of a lattice graph $G = (V, E)$ is a set 
\[
  \mathcal{B}(G) = \{ \langle u, v \rangle \in \mathbb{Z}^2 \times \mathbb{Z}^2 \colon u \in V, (u, v) \not\in E \}.
\]
Observe that elements of $\mathcal{B}(G)$ are directed edges with start points in $V$ such that the corresponding undirected edge is not in $E$. Let us notice, that 
\[
  \pot(G) = \# \mathcal{B}(G).
\]
where $\pot(G)$ is the number defined at the beginning of Section \ref{board_bound}. Here we will analyze the potential more closely and divide it into {\em external} and {\em internal} potentials. 

An edge $e = \langle u, v \rangle \in \mathcal{B}(G)$ is an \emph{external edge} if $u + k \cdot (v - u) \not\in V$ for every
$k \geq 1$. Let $\mathcal{B}^{\external}(G)$ denote the set of all external edges of $G$.

An edge $e \in \mathcal{B}(G)$ is an \emph{internal edge} if it is not an external edge. Let $\mathcal{B}^{\internal}(G)$ denote the set of all internal edges of $G$.

%\begin{definition}We define potential  of a graph $G$\footnote{The notion of potential introduced in \cite[Section 3.1]{demaine} is equivalent to the notion introduced here.} \todo{How does it relate to the definition in Section 3?} as the size of its boundary, i.e.
%\[
%  \pot(G) = \# \mathcal{B}(G).
%\]
The \emph{external potential} $\pot^{\external}(G)$ is the cardinality of the set $\mathcal{B}^{\external}(G)$. 
The \emph{internal potential} $\pot^{\internal}(G)$ is the cardinality of the set $\mathcal{B}^{\internal}(G)$.
%\end{definition}

\begin{lemma}
%  For $G = (V, E)$, $\pot(G) = 8 \cdot \# V - 2\cdot \# E$. In particular, 
 If $G$ is a Morpion position graph, then $\pot(G) = 288$.
\end{lemma}
\begin{proof}
  From the definition of potential at the beginning of Section \ref{board_bound} we have %$\mathcal{B}(G)$ tha
%  \[
%    \pot(G) = 
%     \# \{ \la u, v \ra \colon u \in V \} 
%     - \# \{ \la u, v \ra \colon u \in V, (u, v) \in E \} =
%  \]
  \[
     \pot(G) = 8 \# V - 2 \# E.
  \]
 The number $288$ for Morpion position graphs follows from property \ref{m2} in Lemma \ref{lem:char5pluplus}.
\end{proof}

%\noindent
%{\bf Remark.} At the beginning of Section \ref{board_bound} we defined $\hull(G)$. %Let us notice, that there are just $8$ potential\todo{Remark seems out of place.} assignments of constants $a,b$ in Definition \ref{def:half-plane}, in correspondence with with {\em directions}
%More specifically
%Let us notice, that we can identify $\hull(G)$ with a mapping \[\param_G:{\mathcal D}\to{\mathbb Z},\]
%where 
%\[ \mathcal{D} = \{ -1, 0, 1 \}^2 \setminus \{ (0,0) \}. \]
%and $\param_G(\la a,b\ra)$ is the {\em optimal} constant $c$ such that $G$ is contained in a half--plane $ax+by+c\geq 0$. If the original graph $G$ does not play a role in a given reasoning, then we write $\param$ instead of $\param_G$. 

\noindent
In the proof of Theorem \ref{thm:octagonalization} we need some additional definitions. 
\begin{definition}
Let $\mathcal{L}$  denote the 
\[
  \mathcal{L} = \{ l_{a,b,c} \colon (a,b) \in \mathcal{D},c \in \mathbb{Z} \}.
\]
where
\[
  l_{a,b,c} = \{ (x,y) \in \mathbb{Z}^2 \colon ax + by + c = 0 \}
\]
A line $l_{a,b,c}\in \mathcal{L}$ is called diagonal if $\la a,b\ra\in \{ -1,1\}^2$. 
A graph $G = (V,E)$ is \emph{degenerated} if there exists a line $l \in \mathcal{L}$ such that $V \subset l$. %A graph is \emph{non-degenerated} if it is not degenerated.
\end{definition}

\begin{definition}
  A line $l \in \mathcal{L}$ is a \emph{gap line} for graph $G$ if $l$ is diagonal, 
    does not contain any vertices of $G$,
    but there are vertices of $G$ on both sides of $l$.
    We let $\gap(G)$ be the number of gap lines of a graph $G$.
\end{definition}

\begin{lemma}
% geometry of octagons
\label{lem:geo_of_oct}
If a lattice graph $G=\la V,E\ra$ is an octagon, then for every $l \in \mathcal{L}$ the intersection $V \cap l$ is $1$--connected and  $\pot(G) = \pot^{\external}(G)$.
\end{lemma}
\begin{proof}
For every $l_{a,b,c}\in\mathcal{L}$ the intersection in ${\mathbb R}^2$ of $\{ (x,y)\in \mathbb{R}^2: ax + by + c = 0\}$ with the convex hull of $G$ in ${\mathbb R}^2$ is an interval in ${\mathbb R}^2$ and the lattice points of this interval are $1$--connected and coincide with $V\cap l$. %\todo{Better Lemma about $0$--potential halfplanes and their intersections. Finally we want to remove all Morpion-indepdent statements about potential into a separate section. }
%A formal proof of the equality follows from the decomposition into lines. 
\end{proof}


\begin{lemma}
If $G$ is not degenerated, then $\gap(\hull(G)) = 0$.
\label{lem:gap_hull_0}
\end{lemma}
\begin{proof}
Fix a diagonal line $l$. We will show that $l$ is not a gap line for $\hull(G)$.
Take vertices $u, v \in \hull(G)$ on both sides of $l$.
Since $G$ is non-degenerated, we may assume that $u$ and $v$ are not on a same line perpendicular to $l$ (first we select arbitrary two points on both sides of $l$ and if they are located on the same line then from degeneracy we can find another either on the side of $u$ or on the side of $v$).
%\todo{Include the second configuration as another figure next to the above figure.}

From the definition, $\hull(G)$ contains the intersection of all half-planes that contain $u$ and $v$, hence it contains a parallelogram with the following characteristics:
\begin{itemize}
\item opposite vertices of the parallelogram are $u$ and $v$,
\item a pair of edges of parallelogram is horizontal (or vertical)
\item another pair of edges is diagonal (see Figure~\ref{gaphull2}).
\end{itemize}
% Observe that every line parallel to $l$ that is between $u$ and $v$ intersects this parallelogram in at least one vertex. 

\noindent
   \begin{minipage}[l]{0.45\textwidth}
	     \begin{figure}[H]
%\begin{wrapfigure}[10]{L}{0.45\textwidth}
\vspace{-15pt}
%  \begin{center}
  \begin{tikzpicture}[scale=0.45]
    \draw[blue, dashed, fill=blue!10] (2,5) -- (6,5) -- (10,1) -- (6,1) -- cycle; 
    \draw[color=gray, style=dotted] (0,0) 
      grid[xstep=1cm, ystep=1cm] (11cm,6cm);
  	\node at (2,5) [draw,fill=green] {u};
  	\node at (10,1) [draw,fill=green] {v};
	\draw[red,thick] (2,0) -- (8,6); 
    \node at (3,5) [draw, fill=black] {};
    \node at (3,4) [draw, fill=black] {};
    \node at (4,4) [draw, fill=black] {};
    \node at (4,3) [draw, fill=black] {};
    \node at (5,3) [draw, fill=black] {};
    \node at (5,2) [draw, fill=black] {};
    \node at (6,2) [draw, fill=black] {};
    \node at (6,1) [draw, fill=black] {};
    \node at (7,1) [draw, fill=black] {};
    \node at (8,1) [draw, fill=black] {};
    \node at (9,1) [draw, fill=black] {};
  \end{tikzpicture}
  \caption{Every line parallel to $l$ with $u$ and $v$ on different sides must contain one of the vertices marked with black squares.}
  \label{gaphull2}
%  \end{center}
  \end{figure}
%\end{wrapfigure}

   \end{minipage}%
   \hfill
   \begin{minipage}[r]{0.5\textwidth} 
We start from the vertex $v$ and mark the black dots along the horizontal/vertical edge. Then along the diagonal edge we mark ``the staircase'' as in Figure~\ref{gaphull2}. Notice, that $l$ passes through one of the black dots of the staircase. Indeed, since $u$ and $v$ are not on a line perpendicular to $l$, if $l$ intersect the parallelogram between two points on the diagonal edge, then $l$ necessarily pass through a black dot between these two points. 
\end{minipage}

If $l$ intersects one of the horizontal/vertical edges of the parallelogram, then $l$ passes through one of the black dots located on the edge.  
Hence $l$ has a non--empty intersection with $\hull(G)$ and it follows that $l$ is not a gap line for $\hull(G)$.
\end{proof}
The following Lemma provides a sufficient condition  for a line in ${\mathcal L}$ to be a gap line. 
\begin{lemma}
  Let $G$ be a $1$-connected, bounded and non-degenerated lattice graph and $l \in \mathcal{L}$.
  If $l$ does not contain a vertex of $G$
     and contains a vertex of $\hull(G)$,
     then $l$ is a gap line for $G$.
     \label{lem:crit_gap_lines}
\end{lemma}
\begin{proof}
  Every half-plane graph that contains $l$ must contain at least one vertex of $G$, since $l$ has a vertex in $\hull(G)$ (otherwise the opposite half-plane with $l$ removed would contain whole $G$ and hence also $\hull(G)$, so $\hull(G)$ would be disjoint from $l$).
Therefore both half-planes that have $l$ as a boundary contain vertices of $G$.  
Since $G$ is disjoint with $l$, there are vertices of $G$ on both sides of~$l$.
In order to prove that $l$ is a gap line it is enough to verify that $l$ is diagonal.

Indeed, observe that horizontal and vertical lines disconnect the grid $\mathbb{Z}^2$ into two $1$-connected components. Since $G$ contains vertices on both sides of $l$ and is $1$-connected, the line $l$ must be diagonal, hence it is a gap line for $G$.
\end{proof}

The above Lemma \ref{lem:crit_gap_lines} along with Lemma \ref{lem:gap_hull_0} shows a characterization of gap lines among lines in ${\mathcal L}$. 

\begin{lemma}\label{def:diagonal-gaps}
If $G$ is a $1$-connected, non--degenerated lattice graph, then 
  \[ \pot(\hull(G)) = \pot^{\external}(G)+2\gap(G).\] 
\end{lemma}

\begin{proof}
In this proof it will be convenient to mark as  $\overline{e}$ the set consisting of two vertices at the ends of a given edge $e$. 
Observe that for any graph $\Gamma$ and any line $l \in \mathcal{L}$
\[
\tag{*}
  \# \{ \overline{e}  \colon e\in \mathcal{B}^{\external}(\Gamma),\ \overline{e} \subset l \}
\]
is $0$ iff $V_\Gamma \cap l = \emptyset$ and $2$ otherwise.
By Lemma~\ref{lem:geo_of_oct},
\[
  \pot(\hull(G)) = \pot^{\external}(\hull(G)).
\]

We have
\[
\pot^{\external} (\hull(G)) =  \sum_{l \in \mathcal{L}} \# (l \cap \mathcal{B}^{\external}(\hull(G)))
\]
and%\todo{Eliminate unbounded graphs, because they can have intersections of size $1$ with the external potential.}
\[
  \pot^{\external} (G) = \sum_{l \in \mathcal{L}} \# (l \cap \mathcal{B}^{\external}(G)).
\]

By (*) and by Lemma~\ref{lem:crit_gap_lines}, for a given $l\in{\mathcal L}$ either
  $\# (l \cap \mathcal{B}^{\external}(\hull(G))) =
  \# (l \cap \mathcal{B}^{\external}(G))
  $
or
  $l$ is a gap line for $G$ and $\# (l \cap \mathcal{B}^{\external}(\hull(G))) = 2$, $\# (l \cap \mathcal{B}^{\external}(G)) = 0$. Hence
    \[ \pot(\hull(G)) = \pot^{\external}(G)+2\gap(G).\] 
\end{proof}
 
\noindent

The main technical difficulty in the Section is the following geometric Lemma. This Lemma together with Lemma \ref{def:diagonal-gaps} finish the proof of Theorem \ref{thm:octagonalization}.
\begin{lemma}
\label{lem:morpion_graphs}
If $G$ is a position of the Morpion 5T then \[ 2\gap(G) \leq \pot^{\internal}(G)+\modifier(\hull(G)). \]
\end{lemma}
The notion of $\modifier$ was defined in Theorem \ref{thm:octagonalization}.


\begin{proof}
Let $G = (V, E)$. Let $\mathcal{L}(G)$ denote the set of all gap lines of $G$. Let $l \in \mathcal{L}(G)$. The two halfplanes bounded by $l$ decompose the set $V$ of vertices of $G$ into two disjoint subsets, one of which contains all dots of the initial cross. If the other set contains only a single vertex, then we say that $l$ is a singular gap line. Otherwise we say that $l$ is a non-singular gap line. If $l$ is a singular gap line, then we let $v_l$ denote the single vertex of $V$ separated from the initial cross by line $l$ and call $v_l$ the singular vertex of $l$.


\noindent
   \begin{minipage}[l]{0.45\textwidth}
	
%\tikzset{->,>=stealth'}
\begin{figure}[H]
\vspace{-25pt}
%\begin{wrapfigure}[10]{L}{0.45\textwidth}
%\begin{cutout}{2}{0pt}{\dimexpr0.4\linewidth\relax}{6}
%  \begin{center}
  \begin{tikzpicture}[scale=0.36]
    \draw[blue, dashed, fill=blue!10, opacity=0.3] (4,4) -- (4,14) -- (14,14) -- (14,4) -- cycle; 
    \draw[blue, dashed, fill=yellow!10, opacity=0.3] (0,0) -- (0,10) -- (10,10) -- (10,0) -- cycle; 
    \draw[color=gray, style=dotted] (0,0)  
      grid[xstep=2cm, ystep=2cm] (14cm,14cm);
  	% \node at (4,5) [draw,fill=green,inner sep=0pt,minimum size=1pt] {\scriptsize $u^1_l$};
  	\node(u1l) at (2,4) [draw,fill=green,inner sep=2pt] {$u^1_l$};
  	\node(v1l) at (4,2) [draw,fill=green,inner sep=2pt] {$v^1_l$};
	\draw[red,thick] (0,0) -- (14,14); 
    \node(d1l) at (4,4) [draw,fill=red,inner sep=2pt] {$d_l^1$};
    \node(d2l) at (10,10) [draw,fill=red,inner sep=2pt] {$d_l^2$};
    \node(v2l) at (10,12) [draw, fill=green, inner sep=2pt] {$v^2_l$};
    \node(u2l) at (12,10) [draw, fill=green, inner sep=2pt] {$u^2_l$};
    \node at (12,13) [draw=none] {$l$};
    \node at (3,2) [draw=none] {$e_l^1$};
    \node at (11,12) [draw=none] {$e_l^2$};
    \draw(u1l) to (v1l);
    \draw(v2l) to (u2l);
    \draw[color=black, style=dotted] (v1l) edge[line width=1.1pt,->] (d1l);    
    \draw[color=black, style=dotted] (u1l) edge[line width=1.1pt,->] (d1l);
    \draw[color=black, style=dotted] (v2l) edge[line width=1.1pt,->] (d2l);
    \draw[color=black, style=dotted] (u2l) edge[line width=1.1pt,->] (d2l);
  \end{tikzpicture}
  \caption{Green vertices belong to the graph, red vertices  do not belong to the graph. From the dotted arrows we will choose $2$ to compensate for the gap line $l$.}
  \label{key_lemma_pic}
%  \end{center}
%\end{cutout}
%\end{wrapfigure}
  \end{figure}

   \end{minipage}%
   \hfill
   \begin{minipage}[r]{0.5\textwidth} 
Let $m_1, m_2, \ldots, m_n$ be a sequence of the Morpion 5T moves that lead to a position $G$. Let $l$ be a singular gap line and let $m_k$ be a move that puts the singular vertex $v_l$ on board. Since there are no other vertices in the halfplane bounded by $l$ that contains $v_l$, no move in sequence $m_1, m_2, \ldots$ requires dot $v_l$ and the sequence $m_1, m_2, \ldots, m_{k-1}, m_{k+1}, \ldots, m_n$ is a valid move sequence. Likewise $m_1, m_2, \ldots, m_{k-1}, m_{k+1}, \ldots, m_n, m_k$ is valid.
Hence we may modify our sequence so that the moves that put singular dots on board are at the very end of the move sequence.
   \end{minipage}


Let $m_1, m_2, \ldots, m_k, m_{k+1}, \ldots, m_n$ be a sequence of Morpion 5T moves that lead to a position $G$ such that moves $m_1, m_2, \ldots, m_k$ put non-singular dots on board and moves $m_{k+1}, \ldots, m_n$ put singular dots on board.
Let $H = (V_H, E_H)$ be a Morpion 5T position obtained by a sequence $m_1, m_2, \ldots, m_k$. We will show that
\[
2\gap(H) \leq \pot^{\internal}(H).
\]
Observe that $H$ has no singular gap lines as removing a singular vertex cannot make a non-singular vertex singular.

Let $l \in \mathcal{L}(H)$. Since $l$ is non-singular and $H$ is obtained as a position in Morpion 5T game, there are two edges $e^1_l, e^2_l \in E_H$ that cross $l$. 
Consider labeling of dots as in Figure~\ref{key_lemma_pic}. Note that $d^1_l$ and $d^2_l$ are picked on $l$ between $e^1_l$ and $e^2_l$ (and they may be the same point when $e^1_l$ and $e^2_l$ are next to each other).


We will construct a map that assigns to each $e^i_l$ ($i = 1,2$, $l \in \mathcal{L}(H)$) one edge from the list
\[
  (u^i_l, d^i_l), (v^i_l, d^i_l) \tag{$**^i_l$}
\]
in such a way that the following conditions are satisfied.
\begin{enumerate}
\item The assigned edges realize the internal potential of $H$, i.e. they belong to $\mathcal{B}^{\internal}(H)$.
\item We do not assign the same edge twice.
\end{enumerate}

First we'll show that at least one edge from the edge list $(**^i_l)$ belongs to $\mathcal{B}^{\internal}(H)$.
Without a loss of generality we may assume that $i = 1$.
Consider two half-lines starting at $d^1_l$ in directions $(u^1_l, d^1_l)$ and $(v^1_l, d^1_l)$ (the dotted arrows in Figure~\ref{key_lemma_pic}). 
They disconnect the grid of lattice points into two $1$-connected components. 
Both components contain vertices of $H$ (e.g. $v^1_l$ and $v^{2}_l$ are in different components). 
Since $H$, as a Morpion 5T position, is $1$-connected, 
  there must be a vertex of $H$ on at least 
  one of those half-lines. 
Since $d^1_l$ does not belong to $V_H$ 
  (as $d^1_l \in l$ and $l$ is disjoint from $V_H$ 
  as a gap line), at least one of the 
  edges $(u^1_l, d^1_l), (v^1_l, d^1_l)$ belongs 
  to $\mathcal{B}^{\internal}(H)$. 

Second, we'll show how to pick edges from the edge list $(**^i_l)$ in such a way that the assignment is unique (one-to-one).
  \begin{figure}[H]
  \begin{center}
  \begin{tikzpicture}[scale=0.45]
    \draw[color=gray, style=dotted] (0,0)  
      grid[xstep=2cm, ystep=2cm] (8cm,8cm);
  	\node(u1l) at (2,4) [draw,fill=green,inner sep=2pt] {$u^1_l$};
  	\node(v1l) at (4,2) [draw,fill=green,inner sep=2pt] {$v^1_l$};
    \node at (8,4) [draw,fill=green] {};
	\draw[red,thick] (0,0) -- (8,8); 
	\draw[red,thick] (0,8) -- (8,0); 
    \node(d1l) at (4,4) [draw,fill=red,inner sep=2pt] {$d_l^1$};
    \node at (7.2,7.7) [draw=none] {$l$};
    \node at (7.5,0.2) [draw=none] {$m$};
    \node at (3,2.5) [draw=none] {$e_l^1$};
    \draw (0,6) -- (2,8);
    \node at (0.5,6) [draw=none] {$e_m^1$};
    \draw (6,0) -- (8,2); 
    \node at (7.5,2) [draw=none] {$e_m^2$};
    \draw(u1l) to (v1l);
    %\draw[color=black, style=dashed] (v1l) edge[line width=2.7pt,->] (d1l);    
    \draw[color=black] (u1l) edge[line width=2.7pt,->] (d1l);
    \draw (6,8) -- (8,6);
    \node at (6,7.3) [draw=none] {$e_l^2$};
    
    
  \begin{scope}[shift={(10,0)}]
    \draw[color=gray, style=dotted] (0,0)  
      grid[xstep=2cm, ystep=2cm] (8cm,8cm);
  	\node(u1l) at (2,4) [draw,fill=green,inner sep=2pt] {$u^1_l$};
  	\node(v1l) at (4,2) [draw,fill=green,inner sep=2pt] {$v^1_l$};  	
    \node(v1m) at (4,6) [draw,fill=green,inner sep=2pt] {$v^1_m$};
	\draw[red,thick] (0,0) -- (8,8); 
	\draw[red,thick] (0,8) -- (8,0); 
    \node(d1l) at (4,4) [draw,fill=red,inner sep=2pt] {$d_l^1$};
    \node at (7.2,7.7) [draw=none] {$l$};
    \node at (7.5,0.2) [draw=none] {$m$};
    \node at (3,2.5) [draw=none] {$e_l^1$};
    \draw (u1l) -- (v1m);
    \node at (2,5) [draw=none] {$e_m^1$};
    \draw (6,0) -- (8,2); 
    \node at (7.5,2) [draw=none] {$e_m^2$};
    \draw(u1l) to (v1l);
    \draw[color=black] (v1m) edge[line width=2.7pt,->] (d1l);    
    \draw[color=black] (v1l) edge[line width=2.7pt,->] (d1l);  
    \draw (6,8) -- (8,6);
    \node at (6,7.3) [draw=none] {$e_l^2$};
  \end{scope}
  
    \begin{scope}[shift={(0,-10)}]
    \draw[color=gray, style=dotted] (0,0)  
      grid[xstep=2cm, ystep=2cm] (8cm,8cm);
  	\node(u1l) at (2,4) [draw,fill=green,inner sep=2pt] {$u^1_l$};
  	\node(v1l) at (4,2) [draw,fill=green,inner sep=2pt] {$v^1_l$};  	
    \node(v1m) at (4,6) [draw,fill=green,inner sep=2pt] {$v^1_m$};    
    \node(u2m) at (6,4) [draw,fill=green,inner sep=2pt] {$u^2_m$};
	\draw[red,thick] (0,0) -- (8,8); 
	\draw[red,thick] (0,8) -- (8,0); 
    \node(d1l) at (4,4) [draw,fill=red,inner sep=2pt] {$d_l^1$};
    \node at (7.2,7.7) [draw=none] {$l$};
    \node at (7.5,0.2) [draw=none] {$m$};
    \node at (3,2.5) [draw=none] {$e_l^1$};
    \draw (u1l) -- (v1m);
    \node at (2,5) [draw=none] {$e_m^1$};
    \draw (v1l) -- (u2m); 
    \node at (6,3) [draw=none] {$e_m^2$};
    \draw(u1l) to (v1l);
    \draw[color=black] (u1l) edge[line width=2.7pt,->] (d1l);    
    \draw[color=black] (v1m) edge[line width=2.7pt,->] (d1l); 
    \draw[color=black] (u2m) edge[line width=2.7pt,->] (d1l);  
    \draw (6,8) -- (8,6);
    \node at (6,7.3) [draw=none] {$e_l^2$};
  \end{scope}
  
      \begin{scope}[shift={(10,-10)}]
    \draw[color=gray, style=dotted] (0,0)  
      grid[xstep=2cm, ystep=2cm] (8cm,8cm);
  	\node(u1l) at (2,4) [draw,fill=green,inner sep=2pt] {$u^1_l$};
  	\node(v1l) at (4,2) [draw,fill=green,inner sep=2pt] {$v^1_l$};  	
    \node(v1m) at (4,6) [draw,fill=green,inner sep=2pt] {$v^1_m$};    
    \node(u2m) at (6,4) [draw,fill=green,inner sep=2pt] {$u^2_m$};
	\draw[red,thick] (0,0) -- (8,8); 
	\draw[red,thick] (0,8) -- (8,0); 
    \node(d1l) at (4,4) [draw,fill=red,inner sep=2pt] {$d_l^1$};
    \node at (7.2,7.7) [draw=none] {$l$};
    \node at (7.5,0.2) [draw=none] {$m$};
    \node at (3,2.5) [draw=none] {$e_l^1$};
    \node at (5,6) [draw=none] {$e_l^2$};
    \draw (u1l) -- (v1m);
    \node at (2,5) [draw=none] {$e_m^1$};
    \draw (v1l) -- (u2m); 
    \node at (6,3) [draw=none] {$e_m^2$};
    \draw(u1l) to (v1l);
    \draw(v1m) to (u2m);
    \draw[color=black] (u1l) edge[line width=2.7pt,->] (d1l);    
    \draw[color=black] (v1m) edge[line width=2.7pt,->] (d1l); 
    \draw[color=black] (u2m) edge[line width=2.7pt,->] (d1l);  
    \draw[color=black] (v1l) edge[line width=2.7pt,->] (d1l);  
  \end{scope}
  
  
  \end{tikzpicture}
  \caption{Two gap lines $l$ and $m$ intersecting at a point $d^1_l$. Four figures relate to four cases in the proof. The top left is related to Case 1, the top right to Case 2, the bottom left to Case 3 and the bottom right to Case 4.}
  \label{key_lemma_two_gaps}
  \end{center}
  \end{figure}


\vspace{-20pt}
Consider edge $e^1_l$ that crosses a gap line $l$. There may be only one gap line $m$ such that the edge lists $(**^1_m)$ or $(**^2_m)$ overlap with the edge list $(**^i_l)$. We consider four cases about how edges $e^2_l$, $e^1_m$, $e^2_m$ are placed around $e^i_l$.

Case 1. If both $e^1_m$ and $e^2_m$ are vertex disjoint from $e^1_l$, then we assign to $e^1_l$ any edge from $(**^1_l)$ that belongs to $\mathcal{B}^{\internal}(H)$.

Case 2. Exactly one of $e^1_m$ and $e^2_m$ has a common vertex with $e^1_l$. Without a loss of generality we may assume that $e^1_m$ has a common vertex $u^1_l = u^1_m$ with $e^1_l$.
We must be careful to not assign edge $(u^1_l, d^1_l)$ to both edges $e^1_l$ and $e^1_m$.
We assign $(v^1_l, d^1_l)$ to $e^1_l$ and
  $(v^1_m, d^1_m)$ to $e^1_m$.

Case 3. Both $e^1_m$ and $e^2_m$ have a common vertex with $e^1_l$ but $e^2_l$ is vertex disjoint from $e^1_m$ and $e^2_m$. Assume that $v^1_m$ and $u^2_m$ are vertices of $e^1_m$ and $e^2_m$ that are disjoint from $e^i_l$. 
We assign $(v^1_m, d^1_m)$ to $e^1_m$, $(u^2_m, d^2_m)$ to $e^2_m$ and $(u^1_l, d^1_l)$ to $e^1_l$.

Case 4. Edges $e^1_l$, $e^2_l$, $e^1_m$, $e^2_m$ form a small "diamond" (they pairwise intersect) with $d^1_l = d^2_l = d^1_m = d^2_m$ inside. Assuming that the vertices are labeled in such a way that $u^j_k$ are disjoint, we assign $(u^j_k, d^j_k)$ to $e^j_k$.

\noindent
This concludes the argument that
$
2\gap(H) \leq \pot^{\internal}(H)$.


\noindent
\begin{figure}[H]
%\begin{minipage}[l]{0.45\textwidth}
%\begin{center}
\centering
\begin{tikzpicture}[scale=0.25]
\tikzstyle{every node}=[draw,%
			circle,%
			fill,%
			minimum size  = 0.5*\unitsize,%
			node distance = \unitsize] 
\begin{scope}[shift={(-18,-3)}]
 \morpion 
\end{scope}
\begin{scope}[shift={(-22,0)},scale=1.1]
    \draw[red,thick] (10,20) -- (20,10);           
    \node(v1) at (17,17) [draw,fill=green,inner sep=1pt] {$v_i$}; 
    \draw[color=black] (17,0) edge[line width=2.7pt,dotted] (v1);    
    \draw[color=black] (0,17) edge[line width=2.7pt,dotted] (v1); 
	\node(u1) at (17,8) [draw,fill=green,inner sep=1pt] {$u_i$};     
    \node(v) at (13,13) [draw,fill=green,inner sep=1pt] {$\ $}; 

    \draw[color=black] (u1) edge[line width=2.7pt,->] ($(u1)+(0,3.3)$);
    \draw[color=black] (v1) edge[line width=2.7pt,->] ($(v1)+(0,-3.3)$);
   \draw[color=black] (v) edge[line width=2.7pt] (v1); 
\end{scope}
%\end{tikzpicture}
%\end{minipage}%
%\hfill
%\begin{minipage{r}{0.45\textwidth}
%\begin{tikzpicture}[scale=0.3]
%\tikzstyle{every node}=[draw,%
%			circle,%
%			fill,%
%			minimum size  = 0.5*\unitsize,%
%			node distance = \unitsize] 
\begin{scope}[shift={(9,-3)}]
\morpion
\end{scope}
\begin{scope}[shift={(5,0)},scale=1.1]
\draw[red,thick] (10,20) -- (20,10);                        
%\begin{scope}[shift={(0,0)},scale=1]    
    \node(v1) at (17,17) [draw,fill=green,inner sep=1pt] {$v_i$}; 
    \draw[color=black] (17,0) edge[line width=2.7pt,dotted] (v1);    
    \draw[color=black] (0,17) edge[line width=2.7pt,dotted] (v1);   
    \node(v) at (13,13) [draw,fill=green,inner sep=1pt] {$\ $}; 
   \draw[color=black] (v) edge[line width=2.7pt] (v1);
\end{scope}
\end{tikzpicture}
%\end{minipage}
\caption{Two cases appearing in the analysis of the corners. The figure on the left relates to Case I and the figure on the right to Case II. }
\label{fig:corner}
%\end{center}
\end{figure}

We will now show that the singular moves $m_{k+1}, \ldots, m_n$ add at least $2 \cdot(n -k) - \modifier(\hull(G))$ to the internal potential of the position (i.e. $\pot(G) - \pot(H) \geq 2 \cdot(n -k) - \modifier(\hull(G))$.

First, observe that there are at most $4$ singular moves, that is $n - k \leq 4$. This is because there are two diagonal directions and two sides of a line where the initial cross may be.

Assume that move $m_i$ ($i > k$) places a singular vertex $v_i$. 
Let $l^1_i$, $l^2_i$ be half-lines starting from $v_i$ in the direction of the gap line created by move $m_i$ (the dashed lines in Figure~\ref{fig:corner}).
There are two possibilities.

Case I. At least one of the half-lines $l^1_i$, $l^2_i$ contain a vertex of the position graph. Let $u_i$ denote this vertex. If so, then placing of $v_i$ creates two new edges of internal potential (one starting in $v_i$ in the direction of $u_i$ and another one in $u_i$ in the direction of $v_i$). The new gap line is compensated.

Case II. Neither of the half-lines $l^1_i$ and $l^2_i$ contain a vertex of the position graph. Observe that this is possible only for at most two of the singular moves and each of those moves must create a corner in the hull of $G$. Moreover, if there are two such moves, the corners are opposite corners of $\hull(G)$.

\noindent
This concludes the proof of the Lemma.


\end{proof}