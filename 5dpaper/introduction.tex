\section{Introduction}
The Morpion Solitaire is a paper-and-pencil single-player game played on a square grid with 
  the initial configuration of 36 dots depicted in Figure~\ref{fig:initial}. 
In each move the player puts a dot on an unused grid position and draws a line that 
  consists of four consecutive segments passing through the dot. 
The line must be horizontal, vertical or diagonal. 
The goal is to find the longest possible sequence of moves.
There are two main variants of the game: 5T and 5D. 
They have different restrictions on how the moves may be placed.
In the 5T variant of the game, no segment may be drawn twice, i.e. the moves have to be segment-disjoint. 
In the 5D variant of the game, any two moves in the same direction have to be dot-disjoint.
The difference is demonstrated in Figure~\ref{fig:small}.
The 5D variant is more restrictive. In Morpion 5D there has to be a gap between moves
  placed on a same line.


  \begin{figure}
    \centering
      \includegraphics[width=0.49\textwidth]{figures/empty.pdf}
      \includegraphics[width=0.49\textwidth]{figures/empty.pdf}
      \caption{\label{fig:initial}
	The initial position of Morpion Solitaire is depicted on the right. On the left there is a position which is up to $4$--th move legal 
both in Morpion 5D and Morpion 5T variants, but the $5$--th move is legal only in Morpion 5T. 
      }
\end{figure}

The problem is notoriously hard for computers. 
For $34$ years the longest known sequence in the Morpion 5T game
  was one of 170 moves discovered by C.-H. Bruneau in $1976$. 
Despite considerable computational effort, until $2010$ the computer generated
  sequences were much shorter.
In $2010$, a Nested Rollout Policy Adapation algorithm (a Monte Carlo variant) 
  was developed by Christopher D. Rosin~\cite{rosin} (best paper award at IJCAI11).
Using NRPA, Rosin obtained the current world record of $178$ moves in Morpion 5T 
  and of $82$ moves in Morpion 5D.
The webpage~\cite{} maintained by Christian Boyer, contains an extensive and up-to-date information about records in all Morpion Solitaire variants.

As the rules of Morpion Solitaire do not limit the size of the grid on which the game is played, a priori
  it is not clear if the sequences have to be bounded.
A popular magazine \emph{Science \& Vie} published in $1970$'s different bounds submitted by its readers 
  for the maximal length of a sequence in Morpion 5T
  (the bounds ranged from $540$ to $20736$), but without detailed and/or valid proofs.
The first rigorously proved bound of $705$ in Morpion 5T was published by Demaine et al in $2006$~\cite{demaine}.
The best known bound of $485$ in Morpion 5T was proved in~\cite{}.
In this paper we prove an upper bound of $84$ on the length of Morpion 5D sequence, improving upon earlier bound of $121$ by A.~Kawamura~\cite{japonczycy}.

Proofs of upper bounds discussed above exploit geometric and combinatorial properties of graphs that are obtained as Morpion Solitaire positions. 
To set the mood we'll discuss the bound of $705$ moves for Morpion 5T as it was proved in~\cite{demaine}.
A position of a Morpion Solitaire playout (Figure~\ref{fig:small}) has a graph structure.
Its vertices are placed in the $\mathbb{Z}^2$ grid.
Let $n$ denote the number of vertices, 
  corresponding both to the dots placed by moves and to the dots from the starting cross.
The edges correspond to the segments placed on the grid by moves.
Every move adds a single vertex and four edges to the graph.
Therefore a Morpion position graph has the following two properties: 1) its edges have unit length in $\ell_\infty$ metric; 2) $4n - e = 4 \cdot 36 = 144$. \todo{Notation: $l_\infty$, $n$, $e$}
In~\cite{brass} P. Brass proved that if a planar graph $G$ has $n$ vertices, 
  then the maximum number of edges in $G$ that have unit length in $\ell_\infty$ metric is equal to
  $
    s(n) = \lfloor 4n - \sqrt{28n - 12} \rfloor.
  $
Hence
$
  4n - 144 = e \leq s(n) = \lfloor 4n - \sqrt{28n - 12} \rfloor.
$
The maximal $n$ that satisfies this inequality is $n = 741$. 
Considering the initial $36$ dots this gives an upper bound of $705$ on the number of moves 
  in Morpion Solitaire~\cite{demaine}.
  
Observe that the Morpion position graph has additional geometrical property.
The set of its edges may be covered by a set of segment-disjoint \emph{moves} consisting of four consecutive, parallel, distinct unit length segments. 
We call such graphs \emph{unmarked unordered Morpion graphs} (see Section~\ref{sec:linear} for formal definitions).
In~\cite{ijcai} we used linear programming to show a bound of $485$ on the number of vertices
  in such a graph, under additional constraints about the size of its bounding box 
   that follow from rules of Morpion 5T and from a variant of an isoperimetric inequality. \todo{The same stuff as one paragraph below}
  
In the present paper we shall use additional combinatorial property of Morpion graphs:
%  which states that 
one can find an assignment such that to each move is assigned one of its dots and the assignment is one-to-one\todo{Formal definition in Section 3}.
We call such graphs \emph{unordered Morpion graphs} (Figure~\ref{fig:85}). Every Morpion 5D gameplay generates such assignment but there are assignments 
which are not generated by any gameplay.\todo{Example in Figure 3}
 
We prove that the
  size of graphs with these combinatorial and geometrical properties does not exceed $85$.
We will also use additional constraint about a size of the bounding box of the graph that follows from
  calculations combined with the rules of the game.
We then use additional argument to show that the graphs of size $85$ do not correspond to Morpion positions.

In~\cite{ijcai} we used isoperimetric inequality combined with mixed integer programming to obtain an upper bound of $485$ for Morpion 5T. 
The method employed in~\cite{ijcai} does not give useful upper bound for the 5D variant.
In the present paper we use a new technique of gemmating to limit the size of a bounding box of a Morpion 5D graph combined with a new reduction of Morpion Solitaire to a mixed integer programming to obtain a bound of $84$ for Morpion 5D. \todo{List precisely results which are important for this paper}
