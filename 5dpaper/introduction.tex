\section{Introduction}
The Morpion Solitaire is a paper-and-pencil single-player game played on a square grid with 
  the initial configuration of 36 dots depicted in Figure~\ref{fig:initial}. 
In each move the player puts a dot on an unused grid position and draws a line that 
  consists of four consecutive segments passing through the dot. 
The line must be horizontal, vertical or diagonal. 
The goal is to find the longest possible sequence of moves.
There are two main variants of the game: 5T and 5D. 
They have different restrictions on how the moves may be placed.
In the 5T variant of the game, no segment may be drawn twice, i.e. the moves have to be segment-disjoint. 
In the 5D variant of the game, any two moves in the same direction have to be dot-disjoint.
The difference is demonstrated in Figure~\ref{fig:initial}.\todo{Complete the figure}
% The 5D variant is more restrictive. In Morpion 5D there has to be a gap between moves placed on a same line.

  \begin{figure}
    \centering
      \includegraphics[width=0.49\textwidth]{figures/empty.pdf}
      \includegraphics[width=0.49\textwidth]{figures/empty.pdf}
      \caption{\label{fig:initial}
	The initial position of Morpion Solitaire is depicted on the right. On the left there is a position which is up to $4$--th move legal 
both in Morpion 5D and Morpion 5T variants, but the $5$--th move is legal only in Morpion 5T. 
      }
\end{figure}

The problem is notoriously hard for computers. 
For $34$ years the longest known sequence in the Morpion 5T game
  was one of 170 moves discovered by Bruneau in $1976$. In Morpion 5D a record of 68 was found by Langerman in 1999, later improved by Cazenave \cite{tristan78} to 78, and by Hyyrö and Poranen \cite{finowie} to 79\footnote{An account of this development can be found at the webpage \cite{langerman}.}.
%Despite considerable computational effort, until $2010$ the computer generated
%  sequences were much shorter. 
In $2010$ Rosin \cite{rosin} obtained the current world records of $82$ moves in Morpion 5D and of $178$ moves in Morpion 5T 
  using a specialized Monte Carlo algorithm. This work  was
   recognized as a best paper of the IJCAI conference in 2011. The webpage~\cite{boyer} maintained by Boyer, contains an extensive and up-to-date information about records in all Morpion Solitaire variants.

\subsection{Morpion Solitaire and linear programming}
The rules of Morpion Solitaire do not limit the size of the grid on which the game is played, hence a priori
  it is not clear if the sequences have to be bounded.
%A popular magazine \emph{Science \& Vie} published in $1970$'s different bounds submitted by its readers for the maximal length of a sequence in Morpion 5T  (the bounds ranged from $540$ to $20736$), but without detailed and/or valid proofs.
%The first rigorously proved bound of $705$ in Morpion 5T was published in $2006$ \cite[Demaine et al.]{demaine}.
%The best known bound of $485$ in Morpion 5T was proved in~\cite{ijcai}.
%In this paper we prove an upper bound of $84$ on the length of Morpion 5D sequence, 
%improving upon a bound of $121$ found earlier by \cite[Kawamura et al.]{japonczycy}.
%\begin{comment}
%Proofs of upper bounds discussed above exploit geometric and combinatorial properties of graphs that are obtained as Morpion Solitaire positions. 
%To set the mood we'll discuss the bound of $705$ moves for Morpion 5T as it was proved in~\cite{demaine}.
%A position of a Morpion Solitaire gameplay (Figure~\ref{fig:small}) has a graph structure.
%Its vertices are placed in the $\mathbb{Z}^2$ grid.
%Let $n$ denote the number of vertices, 
%  corresponding both to the dots placed by moves and to the dots from the starting cross.
%The edges correspond to the segments placed on the grid by moves.
%Every move adds a single vertex and four edges to the graph.
%Therefore a Morpion position graph has the following two properties: 1) its edges have unit length in $\ell_\infty$ metric; 2) $4n - e = 4 \cdot 36 = 144$. \todo{I do not like the idea of putting here $n%%$, $e$, $l_\infty$ and basically I am afraid that someone less patient may stop reading here without getting to the main result - maybe we can make a picture instead of this?}
%In~\cite{brass} P. Brass proved that if a planar graph $G$ has $n$ vertices, 
%  then the maximum number of edges in $G$ that have unit length in $\ell_\infty$ metric is equal to
%  $
%    s(n) = \lfloor 4n - \sqrt{28n - 12} \rfloor.
%  $
%Hence
%$
%  4n - 144 = e \leq s(n) = \lfloor 4n - \sqrt{28n - 12} \rfloor.
%$
%The maximal $n$ that satisfies this inequality is $n = 741$. 
%Considering the initial $36$ dots this gives an upper bound of $705$ on the number of moves 
%  in Morpion Solitaire~\cite{demaine}.
%  
%Observe that the Morpion position graph has additional geometrical property.
%The set of its edges may be covered by a set of segment-disjoint \emph{moves} consisting of four consecutive, parallel, distinct unit length segments. 
%We call such graphs \emph{unmarked unordered Morpion graphs} (see Section~\ref{sec:linear} for formal definitions).
% In~\cite{ijcai} we used linear programming to show a bound of $485$ on the number of vertices in such a graph, under additional constraints about the size of its bounding box  that follow from rules of %%Morpion 5T and from a variant of an isoperimetric inequality. %\todo{The same stuff as one paragraph below}
%\end{comment}
 %\todo{Insert why the game is epxressible via a linear program}
However, as a single player game, Morpion Solitaire can be  encoded in a natural way as a linear optimization problem with the optimization target being the length of the sequence. On its own the encoding is not very helpful, because the problem is too large to be practically solvable. Nevertheless, this inspires a natural approach towards construction of upper bounds. Instead of solving the Morpion Solitaire, we solve
a more general game such that
\begin{itemize}
\item every gameplay of ordinary  Morpion Solitaire is a gameplay of the generalization,
\item the generalization is practically solvable and
\item an upper bound proved for this more general game is still interesting for the Morpion Solitaire. 
\end{itemize}
We do not expect that the more general game will be playable by humans. Here is a summary of results we managed to obtain using this approach:
\begin{itemize}
\item the %first rigorously proved 
bound of $705$ in Morpion 5T was proved in $2006$ in \cite{demaine} using a careful geometric analysis of moves in Morpion 5T and an argument based
on the potential method. The current best bound of $485$ in Morpion 5T was obtained in~\cite{ijcai} using an appropriate generalization of Morpion 5T. In particular in \cite{ijcai} it was shown that the bound of $586$ can be obtained via the relaxation of the linear program encoding the rules of  Morpion 5T;
\item a geometric analysis of Morpion 5D led in \cite{japonczycy} to an upper bound of $121$ which improved on an earlier bound of $144$ found in \cite{demaine} using the potential method; considering a generalization of Morpion 5D, in this paper we prove an upper bound of $84$ on the length of Morpion 5D sequences;
%, improving upon a bound of $121$ found earlier by  using a geometrical analysis of moves in Morpion 5D. 
% will significantly improve on the above geometric bounds \cite[Kawamura et al.]{japonczycy};  
%A certain generlization of Morpion Solitaire 5T we analyzed in \cite{ijcai}, however this method 
the method used in  \cite{ijcai} does not seem very helpful in Morpion 5D --- later in this paper we discuss differences between these two generalizations. 
\end{itemize}

%\begin{comment}  
%In the present paper we shall use additional combinatorial property of Morpion graphs:
%one can find an assignment such that to each move is assigned one of its dots and the assignment is one-to-one (see Section \ref{sec:linear} for a precise definition).
%We call such graphs \emph{unordered Morpion graphs}. % (Figure~\ref{fig:85}).
%Every Morpion 5D gameplay generates such assignment but there are assignments 
%which are not generated by any gameplay (see example of such situation in Figure\ref{fig:85})
%\end{comment} 

%\todo{The following paragraph should be written in a different way --- emphasize Theorem, maybe Lemmas, the table} 
\subsection{Organization of the paper}
In Section \ref{sec:geometry} we formulate the main results:
\begin{itemize}
\item in Theorem \ref{thm:boxes} we show that every Morpion 5D gameplay must be contained in a relatively small rectangular board around the initial cross in Figure \ref{fig:initial}; this already proves the bound of $85$ in Morpion 5D,
\item in Corollary \ref{cor:84} we improve this bound to 84 through an analysis of $3$ special cases.
\end{itemize}
The proof of these results is divided between next four sections. In Section \ref{geometry} we introduce basic notions and formulate the main theorem. In Section \ref{sec:linear} we define a useful generalization of Morpion 5D  and encode it as a linear program. In Section \ref{sec:gemmating} we generate a list of rectangular boards (``boxes'') such that any position of Morpion 5D is contained in one of the boxes --- this is achieved using an auxiliary linear program. In Section \ref{sec:upper} we solve 
the generalization of Morpion 5D on all boards found in Section \ref{sec:gemmating} and conclude the proof of Theorem \ref{thm:boxes} along with Corollary \ref{cor:84}. Moreover, we apply the same approach to symmetric variants of Morpion 5T and Morpion 5D and solve completely the second of these
two games.

%We prove that the
%  size of graphs with these combinatorial and geometrical properties does not exceed $85$.
%We will also use additional constraint about a size of the bounding box of the graph that follows from
%  calculations combined with the rules of the game.
%We then use additional argument to show that the graphs of size $85$ do not correspond to Morpion positions.

%In~\cite{ijcai} we used isoperimetric inequality combined with mixed integer programming to obtain an upper bound of $485$ for Morpion 5T. 
%The method employed in~\cite{ijcai} does not give useful upper bound for the 5D variant.
%In the present paper we use a new technique of gemmating to limit the size of a bounding box of a Morpion 5D graph combined with a new reduction of Morpion Solitaire to a mixed integer programming to obtain a bound of $84$ for Morpion 5D. \todo{List precisely results which are important for this paper}
