\section{Morpion Solitaire via Linear Programming}
\label{sec:linear}

%\begin{definition}
  Let $\Cross \subset \mathbb{Z}^2$ be a set of $36$ dots that form an initial cross of Morpion Solitaire.
%  We shall use the following notions.
\begin{definition}
The maximum metric in the square grid $\mathbb{Z}^2$ is defined as $\max (|s_1^1-s_2^1|,|s_1^2-s_2^2|)$
for $s_1,s_2\in \mathbb{Z}^2$. 
A \emph{unit segment} is a pair of points $\{s_1,s_2\}\in \mathbb{Z}^2$ in distance $1$ in the maximum metric. % $s_1\neq s_2$, $s_i = (s_i^1,s_i^2)$, $i=1,2$
 %such that $\max{|s_1^1-s_2^1|,|s_1^2-s_2^2|}=1$. That is, a unit segment is a pair of points in the
%in the square grid $\mathbb{Z}^2$ of length $1$ in the $\ell_\infty$ metric.
\end{definition}


\begin{definition} Let 
    \[
      \mathcal{M} = \{ \{ \{s_1, s_2\}, \{s_2, s_3\}, \{s_3, s_4\}, \{s_4, s_5\} \} \colon s_i \in \mathbb{Z}^2, s_{i+1} - s_i = s_i - s_{i-1} \}
    \]
    Elements of $\mathcal{M}$ are called \emph{moves}. Up to a sign, a given move has one of four directions $(1,0)$, $(0,1)$, $(1,1)$, $(1,-1)$ determined by
the difference $s_{2} - s_1 = s_3 - s_2 = s_4 - s_3 = s_5 - s_4$. 
We call moves \[ \{ \{s_1, s_2\}, \{s_2, s_3\}, \{s_3, s_4\}, \{s_4, s_5\} \}, \{ \{t_1, t_2\}, \{t_2, t_3\}, \{t_3, t_4\}, \{t_4, t_5\} \}\in \mathcal{M}\]
parallel if $s_1-s_2 = \pm (t_1-t_2)$.  
\end{definition}

Notice, that moves consists of four consecutive, distinct, parallel unit segments that intersect at endpoints.

\begin{definition} 
  Let $G = (V, E)$ be a graph. 
\begin{enumerate}
    \item We call $G$  a \emph{lattice graph} if $V \subset \mathbb{Z}^2$
      and each edge of $G$ is a unit segment.
    We let
    \[
      \mathcal{M}(G) = \{ m \in \mathcal{M} \colon m \subset E \}
    \]
    It is the set of all moves that are contained in the edges of $G$. We call $\mathcal{M}(G)$ the set of \emph{moves in $G$}.
    \item Let $d \colon \mathcal{M}(G) \to V$ be an arbitrary mapping. We call $d$ a marking if for every $m\in \mathcal{M}(G)$ holds
      \[
      	d(m) \in \{s_i, s_{i+1}\} \in m
      \]
    for some $i\in \{1,2,3,4\}$.
   %  A \emph{marked move} $m\in \mathcal{M}(G)$ is the pair $(m,d(m))$.%  with a selected vertex $d(m)$ that is one of the endpoints of its segments, i.e
    \item We say that a set $\mathcal{M}(G)$ of moves in $G$ with a marking $d \colon \mathcal{M}(G) \to V$ is \emph{5D-disjoint} if
      \begin{enumerate}
        \item $\bigcup \mathcal{M}(G) = E$,
        \item for every $m_1, m_2 \in \mathcal{M}(G)$ parallel, the moves $m_1$, $m_2$ are vertex disjoint.
        \item $V \setminus d(\mathcal{M}(G)) = \Cross$, that is the set of unmarked vertices $V \setminus d(\mathcal{M}(G))$ forms the initial cross of Morpion Solitaire.
      \end{enumerate}
  \end{enumerate}
\end{definition}

\begin{definition}
  A lattice graph $G$ is an \emph{unordered Morpion 5D graph} if there exists a 5D-disjoint marking of $\mathcal{M}(G)$. 
\end{definition}

(formulation of linear problem, with boundary conditions)

\begin{lemma}
  graph gives variable valuation
\end{lemma}

\begin{lemma}
  variable valuation gives graph
\end{lemma}

(additional variables and constraints that force move order)

(additional constraints that force center symmetry)
