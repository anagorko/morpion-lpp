\section{Morpion Solitaire via Linear Programming}
\label{sec:linear}

%\begin{definition}
  Let $\Cross \subset \mathbb{Z}^2$ be a set of $36$ dots that form an initial cross of Morpion Solitaire.
%  We shall use the following notions.
\begin{definition}
The maximum metric in the square grid $\mathbb{Z}^2$ is defined as $\max (|s_1^1-s_2^1|,|s_1^2-s_2^2|)$
for $s_1,s_2\in \mathbb{Z}^2$. 
A \emph{unit segment} is a pair of points $\{s_1,s_2\}\in \mathbb{Z}^2$ in distance $1$ in the maximum metric. % $s_1\neq s_2$, $s_i = (s_i^1,s_i^2)$, $i=1,2$
 %such that $\max{|s_1^1-s_2^1|,|s_1^2-s_2^2|}=1$. That is, a unit segment is a pair of points in the
%in the square grid $\mathbb{Z}^2$ of length $1$ in the $\ell_\infty$ metric.
\end{definition}


%\begin{definition} 
\subsection{Definition of moves}
Let 
    $
      \mathcal{M} = \{ \{ \{s_1, s_2\}, \{s_2, s_3\}, \{s_3, s_4\}, \{s_4, s_5\} \} \colon s_i \in \mathbb{Z}^2, s_{i+1} - s_i = s_i - s_{i-1} \}.
    $
    Elements of $\mathcal{M}$ are called \emph{moves}. 
Notice, that moves consists of four consecutive, distinct, parallel unit segments that intersect at endpoints. 

Up to a sign, a given move has one of four directions $(1,0)$, $(0,1)$, $(1,1)$, $(1,-1)$ determined by
the difference $s_{2} - s_1 = s_3 - s_2 = s_4 - s_3 = s_5 - s_4$. 
We call two moves $\{ \{s_1, s_2\}, \{s_2, s_3\}, \{s_3, s_4\}, \{s_4, s_5\} \}$ and $\{ \{t_1, t_2\}, \{t_2, t_3\}, \{t_3, t_4\}, \{t_4, t_5\} \}$ 
in $\mathcal{M}$ {\em parallel} if $s_1-s_2 = \pm (t_1-t_2)$.  
% \end{definition}

\subsection{Definition of Morpion 5D graphs}
In the following 
three definitions  $G = (V, E)$ is a fixed graph. 
\begin{definition} 
\begin{itemize}
\item We call $G$  a \emph{lattice graph} if $V \subset \mathbb{Z}^2$
      and each edge of $G$ is a unit segment.
    We let
    $
      \mathcal{M}(G) = \{ m \in \mathcal{M} \colon m \subset E \}
    $
    It is the set of all moves that are contained in the edges of $G$. We call $\mathcal{M}(G)$ the set of \emph{moves in $G$}.
\item Let $d \colon \mathcal{M}(G) \to V$ be an arbitrary mapping. We call $d$ a marking if for every $m\in \mathcal{M}(G)$ holds
      $
      	d(m) \in \{s_i, s_{i+1}\} \in m
      $
    for some $i\in \{1,2,3,4\}$.
\end{itemize}
\end{definition}

Below we define a concept of a Morpion 5D graph, which is a lattice graph with a special marking of vertices. Every Morpion 5D position is
a Morpion 5D graph, but as we discuss below, there are Morpion 5D graphs not related to any Morpion 5D position.

\begin{definition}
\label{def:disjoint}
   %  A \emph{marked move} $m\in \mathcal{M}(G)$ is the pair $(m,d(m))$.%  with a selected vertex $d(m)$ that is one of the endpoints of its segments, i.e
    We say that the set $\mathcal{M}(G)$ of moves in $G$ with a marking $d \colon \mathcal{M}(G) \to V$ is \emph{5D-disjoint} if
      \begin{enumerate}
        \item $\bigcup \mathcal{M}(G) = E$,
        \item for every $m_1, m_2 \in \mathcal{M}(G)$ parallel, the moves $m_1$, $m_2$ are vertex disjoint.
        \item $V \setminus d(\mathcal{M}(G)) = \Cross$, that is the set of unmarked vertices $V \setminus d(\mathcal{M}(G))$ forms the initial cross of Morpion Solitaire. \label{markings}
      \end{enumerate}
\end{definition}

\begin{definition}
  A lattice graph $G$ is an \emph{unordered Morpion 5D graph} if there exists a 5D-disjoint marking of $\mathcal{M}(G)$. 
\end{definition}

Figure~\ref{fig:small} (right) contains an example of an unordered Morpion 5D graph. \todo{We have to decide either for ``unordered Morpion 5D graph'' or ``Morpion 5D graph'' --- currently we have both} %(see Section~\ref{sec:linear} for definitions an  unordered Morpion 5D graph and unmarked unordered Morpion 5D graph).
 The size of such a graph  is defined as the number of vertices minus $36$, that is minus the number of vertices in the \Cross. %\todo{Define the size of such a graph - number of vertices - $36$}.
%We will call every such graph an \emph{unordered Morpion 5T graph}. See section~\ref{sec:linear} for a formal definition.

\begin{figure}[h]
    \includegraphics[width=0.531\textwidth]{figures/85.pdf}
    \includegraphics[width=0.4645\textwidth]{figures/94.pdf}
    \caption{\label{fig:85}
      An unordered Morpion 5D graph of size $85$ on the left side
        and unmarked unordered Morpion 5D graph of size $94$ on the right side. 
    }
\end{figure}

\begin{figure}
  \centering
  \includegraphics[width=0.687\textwidth]{figures/uncon.pdf}
  \caption{
    Unmarked unordered Morpion 5D graph may have arbitrarily large bounding box.
  }
  \label{fig:uncon}
\end{figure}

\begin{remark}
a) There are Morpion 5D graphs that do not correspond to Morpion 5D positions. %\todo{Unmarked defined?}
Figure~\ref{fig:85} shows such an example. The graph on the left is not a Morpion 5D position, 
because every Morpion 5D position admits the last move, characterized by the fact that the last vertex is of degree $1$ or $2$ and if it is
of degree $2$, then the neighbours must be located on a straight line. There is no such vertex in the graph on the left.
%\todo{Inline comments from Figure}
%\end{remark}
%
%\begin{remark}

\noindent
b) If one drops the condition \ref{markings} from Definition \ref{def:disjoint} then we would obtain a concept of a umarked unordered Morpion 5D graph.\todo{Is that correct?} A Morpion 5T equivalent
of this definition was used in~\cite{ijcai} to compute a bound for Morpion 5T.
% These are graphs that can be covered by four-segment lines that are segment-disjoint and such that $4\cdot \# V - \# E = 144$.
%We call such graphs \emph{unmarked unordered Morpion graphs} (see section~\ref{sec:linear} for a formal definition).
Considering this class for Morpion 5D  does not give useful bounds. There are two reasons why this method does not work:
\begin{itemize}
\item these graphs need not to be connected and therefore the bounding box of such a graph
  can be arbitrarily large, see Figure \ref{fig:uncon}, 
\item even if we insist on connectedness, there are examples of unmarked unordered Morpion 5D graphs of size exceeding sizes in 
table \ref{tbl:boundingboxes}, see Figure~\ref{fig:85}. In this specific example, the graph has $94$ vertices.
\end{itemize}
\end{remark}

(formulation of linear problem, with boundary conditions)

\begin{lemma}
  graph gives variable valuation
\end{lemma}

\begin{lemma}
  variable valuation gives graph
\end{lemma}

(additional variables and constraints that force move order)

(additional constraints that force center symmetry)
