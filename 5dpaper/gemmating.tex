% !TEX root = morpion5d.tex

\section{Resizing process}
\label{sec:gemmating}

In this section we describe an algorithm that uses mixed integer programming problems defined
  in Section~\ref{sec:linear} to % prove an upper bound for Morpion 5D.
%The algorithm computes a set of
find a family boxes that contain all Morpion 5D \emph{positions}. In particular, as announced in Theorem \ref{thm:boxes}.\ref{thm:boxes:list},
this algorithm shows that up to a symmetry every Morpion 5D {\em position} is contained in one of the following boxes:
$(4, 3, 3, 3)$, $(4, 4, 3, 2)$, $(5, 3, 2, 3)$, $(4, 3, 4, 2)$, $(5, 3, 3, 2)$, $(5, 4, 2, 1)$, 
$(6, 2, 2, 1)$, $(5, 4, 0, 2)$, $(5, 1, 4, 0)$. 
%Then, for each box $\mathcal{B}$ from this set it computes the maximal size of a Morpion 5D \emph{graph}
%   that has bounding box $\mathcal{B}$.
%The maximal computed size is $85$, which is an upper bound for Morpion 5D.
%In the next section we describe an additional argument that allows us to lower
%  this bound to $84$.

\subsection{Symmetries}
\label{subsec:symmetries}

In order to control the number of involved cases we consider bounding boxes up to symmetries.
There are $8$ isometries of $\mathbb{Z}^2$ that leave $\Cross$ in place: 
1) identity
2) horizontal, vertical and two diagonal reflections,
3) rotations by $90^\circ$, $180^\circ$ and $270^\circ$, centered at the center of the $\Cross$.
  
\begin{lemma}
Up to the $8$ isometries that leave $\Cross$ in place, every box is symmetric to a box $(a,b,c,d)$ such that
\begin{itemize}
\item $a$ is the maximum of $a,b,c,d$,
\item $d\leq b$,
\item if $a=b$, then $d\leq c$. 
\end{itemize}
\end{lemma}

\begin{proof}
Let us recall that $a,b,c,d$ are responsible for the top, left, bottom and right dimensions of the bounding box. 
We have to prove that every box $(a,b,c,d)$ through symmetries can be turned into one of the above boxes. 

First we make sure that $a$ is the maximum 
of $a,b,c,d$ via rotations. In order to satisfy the requirement $d\leq b$ we apply a symmetry with respect to the vertical axis. 
Assume now that $a=b$. In order to satisfy the requirements $d\leq c$ we apply (when needed) a diagonal reflection that interchanges $a$ with $b$ and $c$ with $d$.
\end{proof}

\subsection{Feasible and infeasible boxes}

Let $\mathcal{B}$ be a box.
By Lemma~\ref{lem:solutions}, solutions of mixed integer programming problem
  defined in Section~\ref{sec:linear} correspond to marked Morpion 5D graphs with bounding
  box $\mathcal{B}$.
Let
  \[
    \operatorname{Obj} = \sum_{m \in \mathcal{B}} \sum_{v \in m} \mv_{m, v}
  \]
be an objective function for this problem. The value of $\operatorname{Obj}$ is equal to the number
 of moves in the   Morpion 5D graph corresponding to the solution, i.e. it is equal to the size of this graph.
Hence the maximal value of $\operatorname{Obj}$ is equal to the maximal size of Morpion 5D graph with bounding box~$\mathcal{B}$.
If the model is infeasible, then there is no Morpion 5D graph that has bounding box $\mathcal{B}$ and we call box $\mathcal{B}$ \emph{infeasible}. Otherwise we say that $\mathcal{B}$ is \emph{feasible}.

\subsection{Resizing process}

\begin{definition} A box \emph{resized} from box $(a, b, c, d)$ is one of the following boxes
\[
  (a + 1, b, c, d), (a + 1, b +1 , c, d),  (a, b + 1, c, d), 
(a, b + 1, c + 1, d), 
\]
\[
(a, b, c + 1, d), (a, b, c + 1, d + 1),
(a, b, c, d + 1), (a + 1, b, c, d + 1).     
\]
\end{definition}

\begin{lemma}\label{lem:resizing}
If $\mathcal{B}$ is a bounding box of a Morpion 5D position, then there exists a sequence of
  feasible
  boxes $\mathcal{B}_0, \mathcal{B}_1, \ldots, \mathcal{B}_n$ such that
  $\mathcal{B}_0 = (0,0,0,0)$, $\mathcal{B}_n = \mathcal{B}$ and 
  $\mathcal{B}_{i+1}$ is resized from $\mathcal{B}_{i}$.
\end{lemma}

Even though the Lemma is obvious, its distinguishing property is that
  is uses the key property of Morpion 5D positions that there is an order in which moves are played.
  
\begin{proof}
Let $G$ be a Morpion 5D position and let $G_k$ be a Morpion 5D position
  obtained after first $k$ moves of the game.
Let $\mathcal{B}_k$ be a bounding box of $G_k$.
Observe that $G_{k+1}$ contains one more vertex that $G_{k}$ and this vertex is at the unit distance
  from $G_{k}$.
Therefore either $\mathcal{B}_{k+1}$ is equal to $\mathcal{B}_k$ or it is resized from $\mathcal{B}_k$.
Take a subsequence of $B_n$ with removed duplicates as sequence $\mathcal{B}_0, \mathcal{B}_1, \ldots, \mathcal{B}_n$.
\end{proof}

\subsection{Algorithm for resizing of boxes}
Starting from the smallest box containing the $\Cross$, we recursively generate a list of of all boxes 
  that can be resized from feasible boxes.
By Lemma~\ref{lem:resizing} this list contains all boxes that contain Morpion 5D positions.
The recursive step is summarized in the following code.
\begin{lstlisting}[language = Python,
  basicstyle=\ttfamily\scriptsize,keywordstyle=\color{red},backgroundcolor=\color{white}]
while unsolved:
    box = unsolved.pop()
    result = solve(box) # we use a linear solver to establish if a given box is feasible
    solved.append(box)

    if result.bound > bound:
        bound = result.bound
        
    if result.feasible:
        [ a, b, c, d ] = box

        # we add eight potential new boxes to the list of boxes which should be analyzed
        resized = [ [ a+1, b, c, d], [a, b+1, c, d], [a, b, c+1, d], [a, b, c, d+1],
                    [ a+1,b+1, c,d], [a, b+1,c+1,d], [a,b, c+1,d+1], [a+1,b,c, d+1] ]
        
        # ... and eliminate these cases which were 
        #          - already solved or 
        #          - are already present on the stack.
        # this process is done up to symmetry 
        for g in resized:
            if symmetryClass(g) not in solved + unsolved:
                unsolved.append(symmetryClass(g))
\end{lstlisting}

\begin{lemma}
Every Morpion 5D position fits into one of the feasible boxes found by the resizing algorithm.
The computed bound is an upper bound for Morpion 5D.
\end{lemma}

\begin{proof}
  The proof follows from Lemma~\ref{lem:resizing}.
\end{proof} 
%  infeasible models, 
%  algorithm pseudo-code (python ?),

