\section{Resizing process}
\label{sec:gemmating}

In order to control number of involved cases which must be solved by the linear solever
we:\begin{itemize}
\item consider bounding boxes up to symmetries and 
\item make sure that the number of considered bounding boxes is close to the minimum.
\end{itemize}

\subsection{Symmetries}
\label{subsec:symmetries}
The $8$-element group of isometries of the plane generated by 
\begin{itemize}
\item reflections with respect to the axes and 
\item rotations around the center of the $\Cross$ through the straight angle,
\end{itemize}
 %There is an $8$-element group of isometries of the $\mathbb{Z}^2$ grid that 
leave the $\Cross$ in place. 
\begin{lemma}
Up to these $8$ symmetries every Morpion 5D position is contained in a bounding box $(a,b,c,d)$ such that
\begin{itemize}
\item $a$ is the maximum of $a,b,c,d$,
\item $d\leq b$,
\item if the maximum appears twice in $a,b,c,d$ and $a=b$ then we require that $c\leq d$. 
\end{itemize}
\end{lemma}

\begin{proof}
Let us recall that $a,b,c,d$ are responsible for the top, left, bottom and right dimensions of the bounding box. 
We have to prove that every box $(a,b,c,d)$ through symmetries can be turned into one of the above boxes. 
In Appendix \ref{app:symmetry} we include a pseudocode finding the approriate box and here we explain how it works. 

First we make sure that $a$ is the maximum 
of $a,b,c,d$ via rotations. In order to satisfy the requirement $d\leq b$ we apply a symmetry with respect to the vertical axis. 
Assume now that $a=b$. In order to satisfy the requirements  $c\leq d$ we apply a  reflection with respect to the line $y=x$.\todo{Andrzej, please have a look at this "proof"}
%  $a \leq b,c,d$ and $b \leq d$. 
\end{proof}


\subsection{Resizing of boxes}
Starting from the smallest box containing the $\Cross$, we generate a list of of all boxes relevant for 
Morpion 5D gameplays. That is, iteratively we check using a linear solver whether the box can be extended. 
The iterative step is summarized in the following code.
\begin{lstlisting}[language = Python,
  basicstyle=\ttfamily\scriptsize,keywordstyle=\color{red},backgroundcolor=\color{white}]
while unsolved:
    box = unsolved.pop()
    # we use a linear solver to establish if a given box can extended
    result = solve(box)         
    solved.append(box)

    if result.bound > bound:
        bound = result.bound
        
    if result.type == "FEASIBLE":
        [ a, b, c, d ] = box

        # we add four potential new boxes to the list of boxes which should be analyzed
        gemmate = [ [ a+1, b, c, d], [a, b+1, c, d], [a, b, c+1, d], [a, b, c, d+1],
                    [ a+1,b+1, c,d], [a, b+1,c+1,d], [a,b, c+1,d+1], [a+1,b,c, d+1] ]
        
        # ... and eliminate these casses which were 
        #          - already solved or 
        #          - are already present on the stack.
        # this process is done up to symmetry 
        # the symmetry code is the Appendix             
        for g in gemmate:
            if symmetryClass(g) not in solved + unsolved:
                unsolved.append(symmetryClass(g))
\end{lstlisting}

\begin{lemma}
Every graph of a Morpion 5D position fits into one of the bounding boxes found by the resizing procedure.   
\end{lemma}

\begin{proof}
Let us remark that this is a special property of Morpion 5D positions. In general, we do not expect Morpion 5D graphs to fit into one of the boxes, however the argument presented 
in \cite{demaine} using the potential method applies Morpion 5D graphs (even to relaxations in the sense of Lemma \todo{add Lemma's number}), hence we know a priort that Morpion 5D graphs
does not exceed 144 vertices. 

However, in this Lemma we specifically use information about the structure of a move in Morpion 5D. Namely, in the linear program mentioned in the procedure we specifically require that there 
exists a legal move of Morpion 5D crossing the boundary of the box. If it is the case, then we add $8$ new bounding boxes to the stack of unprocessed boxes. Since every
Morpion 5D position results from extending a given position by a legal move, this guarantees that we can fit such position in one of the generated boxes. 
\todo{Andrzej, please have a look at this "proof"}
\end{proof} 
%  infeasible models, 
%  algorithm pseudo-code (python ?),

