\section{Upper bounds}
\label{sec:upper}

%  discussion of results
  
\subsection{An upper bound of \therecord for Morpion 5D}
\label{subsec:84}

% We summarize arguments which amounts to the proof of main Theorem \ref{thm:boxes}
% \begin{proof}
% We use Lemma about resizing and then Lemma about coding as linear programs. 
% \end{proof}
In order to prove Theorem \ref{thm:boxes} we apply the process described in Section \ref{sec:gemmating} to produce all bounding boxes relavent 
to Morpion 5D game. We solve the linear problem \L{1}-L{5} specified in Definition \ref{def:mip} in all these boxes. Thanks to 
Lemmas \ref{lem:solutions} and \ref{lem:graphs}, the solutions of linear programs are in one-to-one corespondences with existence of
Moprion 5D graphs. This concludes the proof of Theorem \ref{thm:boxes}.


Theorem \ref{thm:boxes} implies the upper bound of $85$ which follows from solving the optimization
problem \L{1}-L{5} on all boxes generated in Section \ref{sec:gemmating}. 
In order to improve the result to $84$ we have to consider three bounding boxes which admit marked Morpion 5D graphs of size $85$. These
are specifically boxes $(4,3,1,1)$, $(4,3,1,2)$, $(4,3,1,3)$.
On these three boards %we solve a more difficult problem with the cutoff condition set to $84.9$ --- 
not only we require that a given solution is a 
Morpion 5D graph (conditions \L{1}-L{5}) of Definition \ref{def:mip}), but additionally we also require that the graph conforms to condition \L{7} of Definition \ref{def:mip}. 
By Lemmas \ref{lem:solutions} and \ref{lem:graphs} this means that we try to solve the full Morpion 5D game on the bounding boxes  $(4,3,1,1)$, $(4,3,1,2)$, $(4,3,1,3)$.
In general it does not seem to be a feasible task, but we succeeded to complete the computations with the cutoff condition of $84.9$.
%\todo{Explain what is acyclity - covered in a Lemma}

\subsection{A solution of Morpion 5D with center symmetry}
We consider Morpion 5D positions which are invariant with respect to $8$ symmetries 
specified in Subsection \ref{subsec:symmetries}. We say that a Morpion 5D position or a graph is {\em symmetric} is invariant with respect to these symmetries.  
In analogy to Theorem \ref{thm:boxes} we can formulate 
\begin{theorem}
\begin{enumerate}
\item Up to symmetry every symmetric Morpion 5D graph that corresponds to a symmetric Morpion 5D position
    is contained in one of the bounding boxes listed in table~\ref{tbl:sym_boundingboxes}.
\item For every row in the table $(a,b,c,d)$ and for every symmetric Morpion 5D graph contained in the bounding box  $(a,b,c,d)$,
its size does not exceed the maximal size indicated in the table.
\end{enumerate} 
\label{thm:sym_boxes}
\end{theorem}

The proof is analogous to the proof of Theorem \ref{thm:boxes}. In the proof we additionally use the condition \L{6} of Definition \ref{def:mip}.

\begin{table}[ht]
\centering
%
    \begin{tabular}{|l|l|l|l|}
    \hline
    No &  Bounding box &  Max size  \\
    \hline%
    
1&(3, 4, 1, 1)& 85.0\\
2&(3, 4, 2, 1)& 85.0\\
3&(4, 3, 1, 2)& 85.0\\
4&(4, 3, 1, 3)& 85.0\\
5&(2, 4, 2, 1)& 84.0\\
6&(2, 4, 2, 2)& 84.0\\
7&(2, 5, 1, 2)& 84.0\\
8&(2, 5, 2, 1)& 84.0\\
9&(2, 5, 2, 2)& 84.0\\
10&(3, 3, 1, 2)& 84.0\\
11&(3, 3, 2, 2)& 84.0\\
12&(3, 4, 1, 2)& 84.0\\
13&(3, 4, 1, 3)& 84.0\\
14&(3, 4, 2, 2)& 84.0\\
15&(3, 4, 3, 2)& 84.0\\
16&(4, 3, 0, 2)& 84.0\\
%
    \hline
    \end{tabular}%
\hspace*{5mm}
%
    \begin{tabular}{|l|l|l|l|}
    \hline
    No &  Bounding box &  Max size  \\
    \hline%
    
17&(4, 3, 2, 3)& 84.0\\
18&(2, 3, 2, 1)& 83.0\\
19&(2, 3, 2, 2)& 83.0\\
20&(2, 5, 1, 1)& 83.0\\
21&(3, 2, 1, 2)& 83.0\\
22&(3, 3, 1, 3)& 83.0\\
23&(3, 3, 3, 3)& 83.0\\
24&(3, 4, 2, 3)& 83.0\\
25&(3, 5, 1, 1)& 83.0\\
26&(3, 5, 2, 1)& 83.0\\
27&(4, 3, 0, 3)& 83.0\\
28&(4, 4, 0, 1)& 83.0\\
29&(4, 4, 0, 2)& 83.0\\
30&(4, 4, 1, 1)& 83.0\\
31&(4, 4, 1, 2)& 83.0\\
32&(4, 4, 1, 3)& 83.0\\
33&(4, 5, 1, 2)& 83.0\\
%
    \hline
    \end{tabular}%
 

\caption{Bounding boxes mentioned in Theorem \ref{thm:sym_boxes} for sizes $68$, $67$ and $65$. All bounding boxes are listed in the Appendix. }
\label{tbl:sym_boundingboxes}
\end{table}

\begin{corollary}
\label{cor:68}
The longest sequence of moves leading to a symmetric Morpion 5D position is equal to $68$.
\end{corollary}
\begin{proof} 
The upper bound follows from Theorem \ref{thm:sym_boxes}.
A record consisting of $68$ moves was found by M.~Quist and is presented in \cite{boyer}. \todo{It may be worthwhile to reproduce it and comment can it be found by a computer}
\end{proof}

We are not aware of any previous upper bounds for symmetric Morpion 5D or Morpion 5T.

\ignore{
\subsection{An upper bound of $222$ for Morpion 5T with center symmetry}

In order to improve the result from $85$ to $8$ we have to consider three cases of bounding boxes which allow marked Morpion 5D graphs of size $85$. These
are specifically boxes $(4,3,1,1)$, $(4,3,1,2)$, $(4,3,1,3)$  
On these three boards we solve a more difficult problem with the cutoff condition set to $84.9$ --- not only we require that a given solution is a 
Morpion 5D graph, but additionally we also require that the graph is {\em acyclic}. \todo{Explain what is acyclity}

\subsection{A solution of Morpion 5D with center symmetry}
As in the previous subsection we consider Morpion 5T positions which are invariant with respect to $8$ symmetries 
specified in Subsection \ref{subsec:symmetries}. %We say that a Morpion 5D position or a graph is {\em symmetric} is invariant with respect to these symmetries.  
In analogy to Theorems \ref{thm:boxes} and \ref{thm:sym_boxes} we can formulate 
\begin{theorem}
\begin{enumerate}
\item Up to symmetry every symmetric Morpion 5T graph that corresponds to a symmetric Morpion 5T position
    is contained in one of the bounding boxes listed in table~\ref{tbl:sym_5t_boundingboxes}.
\item For every row in the table $(a,b,c,d)$ and for every symmetric Morpion 5T graph contained in the bounding box  $(a,b,c,d)$,
its size does not exceed the maximal size indicated in the table.
\end{enumerate} 
\label{thm:sym_5t_boxes}
\end{theorem}

The proof is analogous to the proof of Theorem \ref{thm:boxes}. % and \ref{thm:sym_5t_boxes}. 

\begin{table}[ht]
\centering
%
    \begin{tabular}{|l|l|l|l|}
    \hline
    No &  Bounding box &  Max size  \\
    \hline%
    
1&(3, 4, 1, 1)& 85.0\\
2&(3, 4, 2, 1)& 85.0\\
3&(4, 3, 1, 2)& 85.0\\
4&(4, 3, 1, 3)& 85.0\\
5&(2, 4, 2, 1)& 84.0\\
6&(2, 4, 2, 2)& 84.0\\
7&(2, 5, 1, 2)& 84.0\\
8&(2, 5, 2, 1)& 84.0\\
9&(2, 5, 2, 2)& 84.0\\
10&(3, 3, 1, 2)& 84.0\\
11&(3, 3, 2, 2)& 84.0\\
12&(3, 4, 1, 2)& 84.0\\
13&(3, 4, 1, 3)& 84.0\\
14&(3, 4, 2, 2)& 84.0\\
15&(3, 4, 3, 2)& 84.0\\
16&(4, 3, 0, 2)& 84.0\\
%
    \hline
    \end{tabular}%
\hspace*{5mm}
%
    \begin{tabular}{|l|l|l|l|}
    \hline
    No &  Bounding box &  Max size  \\
    \hline%
    
17&(4, 3, 2, 3)& 84.0\\
18&(2, 3, 2, 1)& 83.0\\
19&(2, 3, 2, 2)& 83.0\\
20&(2, 5, 1, 1)& 83.0\\
21&(3, 2, 1, 2)& 83.0\\
22&(3, 3, 1, 3)& 83.0\\
23&(3, 3, 3, 3)& 83.0\\
24&(3, 4, 2, 3)& 83.0\\
25&(3, 5, 1, 1)& 83.0\\
26&(3, 5, 2, 1)& 83.0\\
27&(4, 3, 0, 3)& 83.0\\
28&(4, 4, 0, 1)& 83.0\\
29&(4, 4, 0, 2)& 83.0\\
30&(4, 4, 1, 1)& 83.0\\
31&(4, 4, 1, 2)& 83.0\\
32&(4, 4, 1, 3)& 83.0\\
33&(4, 5, 1, 2)& 83.0\\
%
    \hline
    \end{tabular}%
 

\caption{Bounding boxes mentioned in Theorem \ref{thm:sym_5t_boxes} for sizes $222$, $221$ and $220$. All bounding boxes are listed in the Appendix. }
\label{tbl:sym_5t_boundingboxes}
\end{table}

\begin{corollary}
\label{cor:222}
The longest sequence of moves leading to a symmetric Morpion 5D position does not exceed $222$.
\end{corollary}
\begin{proof} 

\end{proof}

\begin{remark}
A record of 136 moves in symmetric Morpion 5T was found by M.~Quist and is presented in \cite{boyer}. We are not aware of any previous upper bounds for symmetric Morpion 5D or Morpion 5T.
\end{remark}
}
