% !TEX root = morpion5d.tex

\section{Upper bounds}
\label{sec:upper}

%  discussion of results
  
\subsection{An upper bound of \therecord for Morpion 5D}
\label{subsec:84}

% We summarize arguments which amounts to the proof of main Theorem \ref{thm:boxes}
% \begin{proof}
% We use Lemma about resizing and then Lemma about coding as linear programs. 
% \end{proof}
In order to conclude the proof of Theorem \ref{thm:boxes} we apply the process described in Section \ref{sec:gemmating} to produce all bounding boxes relevant 
to Morpion 5D game. We solve the linear problem \L{1}-\L{5} specified in Definition \ref{def:mip} in all these boxes. Thanks to 
Lemmas \ref{lem:solutions} and \ref{lem:graphs}, the solutions of linear programs are in one-to-one correspondence with existence of
Morpion 5D graphs. This concludes the proof of Theorem \ref{thm:boxes}.


Theorem \ref{thm:boxes} implies the upper bound of $85$ which follows from solving the optimization
problem \L{1}-\L{5} on all boxes generated in Section \ref{sec:gemmating}. 
In order to improve the result to $84$ we have to consider three bounding boxes which admit marked Morpion 5D graphs of size $85$. These
are specifically boxes $(4,3,1,1)$, $(4,3,1,2)$, $(4,3,1,3)$.
On these three boards %we solve a more difficult problem with the cutoff condition set to $84.9$ --- 
not only we require that a given solution is a 
Morpion 5D graph (conditions \L{1}-\L{5} of Definition \ref{def:mip}), but additionally we also require that the graph conforms to condition \L{7} of Definition \ref{def:mip}. 
By Lemmas \ref{lem:solutions} and \ref{lem:graphs} this means that we try to solve the full Morpion 5D game on the bounding boxes  $(4,3,1,1)$, $(4,3,1,2)$, $(4,3,1,3)$.
In general it does not seem to be a feasible task, but we succeeded to complete the computations with the cutoff condition of $84.9$.
%\todo{Explain what is acyclity - covered in a Lemma}

\subsection{A solution of Morpion 5D with center symmetry}
We consider Morpion 5D positions which are invariant with respect to $8$ symmetries 
specified in Subsection \ref{subsec:symmetries}. We say that a Morpion 5D position or a graph is {\em symmetric} is invariant with respect to these symmetries.  
In analogy to Theorem \ref{thm:boxes} we can formulate 
\begin{theorem}
\begin{enumerate}
\item Up to symmetry every symmetric Morpion 5D {\bf position} is contained in box
$(4, 2, 4, 2)$ or box $(3, 3, 3, 3)$.
\item Every box contained in one of the boxes listed in 1 is a bounding box of a symmetric Morpion 5D {\bf graph}.
\item For each box described in $2$, the size of a maximal Morpion 5D graph is $76$ for boxes $(2, 1, 2, 1), (2, 2, 2, 2)$; $74$ for boxes $(4,2,4,2)$, $(4,1,4,1)$, $(3,2,3,2)$, $(3,1,3,1)$; $72$ for boxes $(1,1,1,1)$, $(1,0,1,0)$.
\item For each box described in $3$, the size of a maximal Morpion 5D position is $68$.
\end{enumerate} 
\label{thm:sym_boxes}
\end{theorem}

The proof is analogous to the proof of Theorem \ref{thm:boxes}. In the proof we additionally use the condition \L{6} of Definition \ref{def:mip}. To prove 4, we use the condition \L{7}.

\begin{corollary}
\label{cor:68}
The longest sequence of moves leading to a symmetric Morpion 5D position is equal to $68$.
\end{corollary}
\begin{proof} 
The upper bound follows from Theorem \ref{thm:sym_boxes}.
A record consisting of $68$ moves was found by M.~Quist and is presented in \cite{boyer}. % \todo{It may be worthwhile to reproduce it and comment can it be found by a computer}
\end{proof}

We are not aware of any previous upper bounds for symmetric Morpion 5D or Morpion 5T.

