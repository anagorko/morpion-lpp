\documentclass[a4paper,UKenglish]{lipics}
%This is a template for producing LIPIcs articles. 
%See lipics-manual.pdf for further information.
%for A4 paper format use option "a4paper", for US-letter use option "letterpaper"
%for british hyphenation rules use option "UKenglish", for american hyphenation rules use option "USenglish"
% for section-numbered lemmas etc., use "numberwithinsect"
 
\usepackage{microtype}%if unwanted, comment out or use option "draft"
\usepackage{graphicx}

\usepackage{array}
\usepackage{epsfig}
\usepackage[all]{xy}
\usepackage{enumerate}
\usepackage{graphicx}
\usepackage{tikz}
\usepackage{xspace}
\usepackage{float}
\usepackage{comment}

%%%% Packages needed by Matteo
\usepackage[all]{xy}
\usepackage{stmaryrd}
\usepackage{amssymb}
\usepackage{amsmath}
\usepackage{verbatim}


%%%% End of Packages needed by Matteo

\newcommand{\fun}[3]{\ensuremath{#1\colon #2 \to #3}}
\newcommand{\parfun}[3]{\ensuremath{#1\colon #2 \rightharpoonup #3}}

%%%%%%%%%%%%%%%%%%%%%%%%%%%%%%%%
%
%
% The record
%
%
%%%%%%%%%%%%%%%%%%%%%%%%%%%%%%%%

\newcommand{\therecord}{{$84$}\ }
\newcommand{\theoctagons}{{$126912$}\ }
\newcommand{\theinstances}{{$42889$}\ }

%%%%%%%%%%%%%%%%%%%%%%%%%%%%%%%%
%
%
% The biblatex requirements 
%
%
%%%%%%%%%%%%%%%%%%%%%%%%%%%%%%%%

\usepackage[backend=bibtex]{biblatex}
\bibliography{games.bib}

%%%%%%%%%%%%%%%%%%%%%%%%%%%%%%%
%
%
% Stuff needed for pictures
%
%
%%%%%%%%%%%%%%%%%%%%%%%%%%%%%%%%

%\usepackage[usenames,dvipsnames]{xcolor}
\usepackage{tikz}
\usepackage{xifthen}
\usepackage{intcalc}
\usetikzlibrary{calc}
\usetikzlibrary{arrows}
\usepackage{pgfplots}
\usepackage{wrapfig}
\usepackage[font={scriptsize,it}]{caption}
\usepackage{cutwin}
%\usepackage{picins}


%%%%%%%%%%%%%%%%%%%%%%%%%%%%%%%%
%
%
% Added for the sake of internal 
% references
%
%
%%%%%%%%%%%%%%%%%%%%%%%%%%%%%%%%

\usepackage{enumitem, hyperref}
\makeatletter
\def\namedlabel#1#2{\begingroup
    #2%
    \def\@currentlabel{#2}%
    \phantomsection\label{#1}\endgroup
}

%%%%%%%%%%%%%%%%%%%%%%%%%%%%%%%%
%
%
% Added to have a sane enumeration 
% of conditions in Lemma 4 (o1)-(o8)
%
%
%%%%%%%%%%%%%%%%%%%%%%%%%%%%%%%%

\usepackage{mathtools} 

%%%%%%%%%%%%%%%%%%%%%%%%%%%%%%%%
%
%
% Commands needed by Andrzej
%
%
%%%%%%%%%%%%%%%%%%%%%%%%%%%%%%%%


%\usepackage[english]{babel}
\usepackage[utf8]{inputenc}
\usepackage{graphicx}
%\usepackage{amsthm}
\usepackage{amsmath}
\usepackage[colorinlistoftodos]{todonotes}


\numberwithin{equation}{section}

%\newtheorem{theorem}{Theorem}[section]
%\newtheorem{lemma}[theorem]{Lemma}
%\newtheorem{corollary}[theorem]{Corollary}
%\newtheorem{proposition}[theorem]{Proposition}
%\newtheorem{conjecture}[theorem]{Conjecture}
%\newtheorem*{theorem*}{Theorem}

%\theoremstyle{definition}
%\newtheorem*{definition}{Definition}
%\newtheorem{example}[theorem]{Example}
\newtheorem{xca}[theorem]{Exercise}

%\theoremstyle{remark}
%\newtheorem{remark}[theorem]{Remark}
%\newtheorem{question}[theorem]{Question}
%\newtheorem*{remark*}{Remark}
\newtheorem*{convention*}{Convention}
\newtheorem*{notation*}{Notation}

\DeclareMathOperator{\hull}{hull}
\DeclareMathOperator{\external}{ex}
\DeclareMathOperator{\modifier}{modifier}
\DeclareMathOperator{\internal}{int}
\DeclareMathOperator{\gap}{gap}
\DeclareMathOperator{\param}{param}
\DeclareMathOperator{\bd}{\partial}
\DeclareMathOperator{\pot}{potential}
\DeclareMathOperator{\dt}{dot}
\DeclareMathOperator{\move}{mv}
\DeclareMathOperator{\obj}{obj}
\DeclareMathOperator{\oct}{octagon}


\renewcommand{\comment}[1]{}


\newcommand {\nat}{\mathbb{N}}
\newcommand {\scc}{\mathsf{scc}}
\newcommand {\ord}{\mathsf{ORD}}
\newcommand {\un}{\mathsf{un}}
\newcommand {\head}{\mathsf{head}}
\newcommand {\tail}{\mathsf{tail}}
\newcommand {\ad}{\mathsf{ad}}
\newcommand {\sub}{\mathsf{sub}}
\newcommand {\dom}{\mathsf{dom}}
\newcommand {\rank}{\mathsf{rank}\xspace}
\newcommand {\red}{\mathsf{red}}
\newcommand {\LGA}{\mathsf{LGA}}
\newcommand {\WAA}{\mathsf{WAA}}
\newcommand {\SAA}{\mathsf{SAA}}
\newcommand {\mybox}{\textrm{{\scriptsize $\,\Box\,$}}}
\newcommand {\R}{${\mathcal R}$}
\newcommand {\RR}{{\mathcal R}}
\newcommand {\C}{${\mathcal C}$}
\newcommand {\BC}{\mathsf{BC}}


\newcommand{\dotcup}{\ensuremath{\mathaccent\cdot\cup}}


% Letters and numbers

\newcommand{\mathromnum}[1]{\ensuremath{\mathrm{#1}}\xspace}

\newcommand{\rI}{\mathromnum{I}}
\newcommand{\rII}{\mathromnum{II}}
\newcommand{\rIII}{\mathromnum{III}}
\newcommand{\rIV}{\mathromnum{IV}}
\newcommand{\rV}{\mathromnum{V}}

\newcommand{\mathcalsym}[1]{\ensuremath{\mathcal{#1}}\xspace}

\newcommand{\Cross}{\texttt{Cross}}
\newcommand{\Aa}{\mathcalsym{A}}
\newcommand{\Bb}{\mathcalsym{B}}
\newcommand{\Cc}{\mathcalsym{C}}
\newcommand{\Dd}{\mathcalsym{D}}
\newcommand{\Ee}{\mathcalsym{E}}
\newcommand{\Ff}{\mathcalsym{F}}
\newcommand{\Gg}{\mathcalsym{G}}
\newcommand{\Hh}{\mathcalsym{H}}
\newcommand{\Ii}{\mathcalsym{I}}
\newcommand{\Jj}{\mathcalsym{J}}
\newcommand{\Kk}{\mathcalsym{K}}
\newcommand{\Ll}{\mathcalsym{L}}
\newcommand{\Mm}{\mathcalsym{M}}
\newcommand{\Nn}{\mathcalsym{N}}
\newcommand{\Oo}{\mathcalsym{O}}
\newcommand{\Pl}{\mathcalsym{P}}
\newcommand{\Ql}{\mathcalsym{Q}}
\newcommand{\Rr}{\mathcalsym{R}}
\newcommand{\Ss}{\mathcalsym{S}}
\newcommand{\Tt}{\mathcalsym{T}}
\newcommand{\Uu}{\mathcalsym{U}}
\newcommand{\Vv}{\mathcalsym{V}}
\newcommand{\Ww}{\mathcalsym{W}}
\newcommand{\Xx}{\mathcalsym{X}}
\newcommand{\Yy}{\mathcalsym{Y}}
\newcommand{\Zz}{\mathcalsym{Z}}
\newcommand{\W}{\mathrm{W}}
\newcommand{\coR}{\mbox{co-}\Rr}

%-------------------------------

\newcommand{\trees}{\mathrm{Tr}\xspace}
\newcommand{\runs}{\mathrm{Runs}\xspace}
\newcommand{\otrees}{\mathrm{Tr}(\omega)\xspace}
\newcommand{\WF}{\mathrm{WF}\xspace}


\newcommand{\boldclass}[3]{\ensuremath{\mathbf{#1}^{#2}_{#3}}}
\newcommand{\lightclass}[3]{\ensuremath{{#1}^{#2}_{#3}}}

\newcommand{\borel}{\ensuremath{\mathcal B}\xspace}

\newcommand{\bsigma}[1]{\boldclass{\Sigma}{0}{#1}}
\newcommand{\bpi}[1]{\boldclass{\Pi}{0}{#1}}
\newcommand{\bdelta}[1]{\boldclass{\Delta}{0}{#1}}

\newcommand{\asigma}[1]{\boldclass{\Sigma}{1}{#1}}
\newcommand{\api}[1]{\boldclass{\Pi}{1}{#1}}
\newcommand{\adelta}[1]{\boldclass{\Delta}{1}{#1}}

\newcommand{\esigma}[2]{\lightclass{\Sigma}{#1}{#2}}
\newcommand{\epi}[2]{\lightclass{\Pi}{1}{#1}}
\newcommand{\edelta}[2]{\lightclass{\Delta}{#1}{#2}}


\newcommand{\bc}[1]{\mathcal{BC}({#1})}

\newcommand{\textscsym}[1]{{\sc #1}}

\newcommand{\eqdef}{\stackrel{\mathrm{def}}=}

\newcommand{\N}{\nat}

%\newcommand{\dL}{\mathrm{L}}
%\newcommand{\dR}{\mathrm{R}}

\newcommand{\rmin}{i}
\newcommand{\rmax}{k}

\newcommand{\G}{\mathcal{G}}

\newcommand{\Plab}{{\mathrm{label}}}
\newcommand{\guarantee}{{\mathrm{promise}}}

\newcommand{\Pgar}{X}

\newcommand{\eve}{\ensuremath{\exists}\xspace}
\newcommand{\adam}{\ensuremath{\forall}\xspace}

\newcommand{\ZFC}{{\sc ZFC}}
\newcommand{\LTL}{{\sc LTL}}
\newcommand{\CTL}{{\sc CTL}}
\newcommand{\PCTL}{{\sc PCTL}}
\newcommand{\CTLstar}{{\sc LTL$^\ast$}}
\newcommand{\ZFCMA}{{\sc ZFC+MA$_{\aleph_{1}}$}}
\newcommand{\MA}{{\sc MA$_{\aleph_{1}}$}}
\newcommand{\ZFCVL}{{\sc ZFC+V=L}}
\newcommand{\VL}{{\sc V=L}}
\newcommand{\AD}{{\sc AD}}
\newcommand{\CH}{{\sc CH}}

\newcommand{\powerset}{{\mathcal P}}
\newcommand{\la}{{\langle}}
\newcommand{\ra}{{\rangle}}

\newcommand{\hargame}{{H\Gg}}

\newcommand{\restr}{\!\upharpoonright}

\newcommand{\ignore}[1]{}
\newcommand{\Souslin}{\mathcal A}


\newenvironment{proofof}[1]
	{\vspace{1ex}\noindent{\emph{Proof of #1}}\hspace{0.5em}}
    {\hfill\qed\vspace{1ex}}
    
%%%%%%%%%%%%%%%%%%%%%%%%%%%%%%%%%%%%
%
%
% acknowledgments on the front page
%
%
%%%%%%%%%%%%%%%%%%%%%%%%%%%%%%%%%%%%    

\newcommand*\samethanks[1][\value{footnote}]{\footnotemark[#1]}


\lstdefinestyle{myCustomPythonStyle}{
  language=Python,
  numbers=left,
  stepnumber=1,
  numbersep=10pt,
  tabsize=4,
  showspaces=false,
  showstringspaces=false,
  basicstyle=\ttfamily\small
}




% Author macros::begin %%%%%%%%%%%%%%%%%%%%%%%%%%%%%%%%%%%%%%%%%%%%%%%%
\title{{An upper bound of 84 for Morpion Solitaire 5D}}%\footnote{This work was partially supported by someone.}}
\titlerunning{A Morpion 5D bound} %optional, in case that the title is too long; the running title should fit into the top page column

\author[1]{Henryk Michalewski}
\author[1]{Andrzej Nagórko}
\author[1]{Jakub Pawlewicz}
\affil[1]{Department of Mathematics, Informatics and Mechanics\\ University of Warsaw\\ \{H.Michalewski,A.Nagorko,J.Pawlewicz\}@mimuw.edu.pl} %\\ \texttt{open@dummyuni.org}}
%\affil[2]{Department of Informatics, Dummy College\\
%  Address, Country\\
%  \texttt{access@dummycollege.org}}
\authorrunning{H. Michalewski, A. Nagórko and J. Pawlewicz} %mandatory. First: Use abbreviated first/middle names. Second (only in severe cases): Use first author plus 'et. al.'

\Copyright{Henryk Michalewski, Andrzej Nagórko and Jakub Pawlewicz}%mandatory, please use full first names. LIPIcs license is "CC-BY";  http://creativecommons.org/licenses/by/3.0/

\subjclass{Dummy classification -- please refer to \url{http://www.acm.org/about/class/ccs98-html}}% mandatory: Please choose ACM 1998 classifications from http://www.acm.org/about/class/ccs98-html . E.g., cite as "F.1.1 Models of Computation". 
\keywords{Morpion, linear optmization, relaxation}% mandatory: Please provide 1-5 keywords
% Author macros::end %%%%%%%%%%%%%%%%%%%%%%%%%%%%%%%%%%%%%%%%%%%%%%%%%

%Editor-only macros:: begin (do not touch as author)%%%%%%%%%%%%%%%%%%%%%%%%%%%%%%%%%%
\serieslogo{}%please provide filename (without suffix)
\volumeinfo%(easychair interface)
  {Billy Editor and Bill Editors}% editors
  {2}% number of editors: 1, 2, ....
  {Conference title on which this volume is based on}% event
  {1}% volume
  {1}% issue
  {1}% starting page number
\EventShortName{}
\DOI{10.4230/LIPIcs.xxx.yyy.p}% to be completed by the volume editor
% Editor-only macros::end %%%%%%%%%%%%%%%%%%%%%%%%%%%%%%%%%%%%%%%%%%%%%%%

\begin{document}

\maketitle

\begin{abstract} 
The Morpion Solitaire is a paper-and-pencil single-player game played on a square grid with initial position consisting of $36$ dots.
In each move the player puts a dot on an unused grid position and draws a line that 
  consists of four consecutive segments passing through the dot.
 The goal is to find the longest possible sequence of moves.
There are two main variants of the game: 5T and 5D. 
They have different restrictions on how the moves may intersect.

Providing lower and upper bounds in Morpion Solitaire is a significant computational and mathematical challenge
  that led to important algorithmical discoveries.
 Best lower bounds for Morpion Solitaire were obtained using a variant of the Monte Carlo method: 
 $178$ moves for the 5T variant, $82$ moves for the 5D variant.
 Both bounds were found in $2011$ by Rosin. 
 
 In the Morpion 5T variant an upper bound of $705$ was proved in $2006$ by Demaine at al. using a method of potential. 
A refinement of this approach was used in $2015$  by Michalewski et al. to show a bound of $485$ - this method combined an isoperimetric inequality with linear programming. 

In the Morpion 5D variant 
Kawamura et al. proved an upper bound of $121$ moves using a mathematical argument. 
We improve this bound to \therecord. 
This is done in two steps.
1) We state, using linear constraints, a geometric property that defines a class of graphs that includes all Morpion  positions - finding practically solvable linear constraints is a main conceptual advance of the present paper.
2) We solve the mixed integer problems and obtain 
an upper bound of \therecord for Morpion 5D. The same method a) fully solves Morpion 5D with center symmetry (the optimal sequence is $68$ moves long) and b) gives an upper bound of $222$ for Morpion 5T with center symmetry. 
\end{abstract}

\section{Introduction}
The Morpion Solitaire is a paper-and-pencil single-player game played on a square grid with 
  the initial configuration of 36 dots depicted in Figure~\ref{fig:initial}. 
In each move the player puts a dot on an unused grid position and draws a line that 
  consists of four consecutive segments passing through the dot. 
The line must be horizontal, vertical or diagonal. 
The goal is to find the longest possible sequence of moves.
There are two main variants of the game: 5T and 5D. 
They have different restrictions on how the moves may be placed.
In the 5T variant of the game, no segment may be drawn twice, i.e. the moves have to be segment-disjoint. 
In the 5D variant of the game, any two moves in the same direction have to be dot-disjoint.
The difference is demonstrated in Figure~\ref{fig:small}.
The 5D variant is more restrictive. In Morpion 5D there has to be a gap between moves
  placed on a same line.


  \begin{figure}
    \centering
      \includegraphics[width=0.49\textwidth]{figures/empty.pdf}
      \includegraphics[width=0.49\textwidth]{figures/empty.pdf}
      \caption{\label{fig:initial}
	test
      }
\end{figure}

The problem is notoriously hard for computers. 
For $34$ years the longest known sequence in the Morpion 5T game
  was one of 170 moves discovered by C.-H. Bruneau in $1976$. 
Despite considerable computational effort, until $2010$ the computer generated
  sequences were much shorter.
In $2010$, a Nested Rollout Policy Adapation algorithm (a Monte Carlo variant) 
  was developed by Christopher D. Rosin~\cite{rosin} (best paper award at IJCAI11).
Using NRPA, Rosin obtained the current world record of $178$ moves in Morpion 5T 
  and of $82$ moves in Morpion 5D.
The webpage~\cite{} maintained by Christian Boyer, contains an extensive and up-to-date information about records in all Morpion Solitaire variants.

As the rules of Morpion Solitaire do not limit the size of the grid on which the game is played, a priori
  it is not clear if the sequences have to be bounded.
A popular magazine \emph{Science \& Vie} published in $1970$'s different bounds submitted by its readers 
  for the maximal length of a sequence in Morpion 5T
  (the bounds ranged $540$ to $20736$), but without detailed and/or valid proofs.
The first rigorously proved bound of $705$ in Morpion 5T was published by Demaine et al in $2006$~\cite{}.
The best known bound of $485$ in Morpion 5T was proved in~\cite{}.
In this paper we prove an upper bound of $84$ on the length of Morpion 5D sequence, improving upon earlier bound of $121$ by A.~Kawamura~\cite{}.

Proofs of upper bounds discussed above exploit geometric and combinatorial properties of graphs that are obtained as Morpion Solitaire positions. 
To set the mood we'll discuss the bound of $705$ moves for Morpion 5T as it was proved in~\cite{demaine}.
A position of a Morpion Solitaire playout (Figure~\ref{fig:small}) has a graph structure.
Its vertices are placed in the $\mathbb{Z}^2$ grid.
Let $n$ denote the number of vertices, 
  corresponding both to the dots placed by moves and to the dots from the starting cross.
The edges correspond to the segments placed on the grid by moves.
Every move adds a single vertex and four edges to the graph.
Therefore a Morpion position graph has the following two properties: 1) its edges have unit length in $\ell_\infty$ metric; 2) $4n - e = 4 \cdot 36 = 144$.
In~\cite{brass} P. Brass proved that if a planar graph $G$ has $n$ vertices, 
  then the maximum number of edges in $G$ that have unit length in $\ell_\infty$ metric is equal to
  $
    s(n) = \lfloor 4n - \sqrt{28n - 12} \rfloor.
  $
Hence
$
  4n - 144 = e \leq s(n) = \lfloor 4n - \sqrt{28n - 12} \rfloor.
$
The maximal $n$ that satisfies this inequality is $n = 741$. 
Considering the initial $36$ dots this gives an upper bound of $705$ on the number of moves 
  in Morpion Solitaire~\cite{demaine}.
  
Observe that the Morpion position graph has additional geometrical property.
The set of its edges may be covered by a set of segment-disjoint \emph{moves} consisting of four consecutive, parallel, distinct unit length segments. 
We call such graphs \emph{unmarked unordered Morpion graphs} (see Section~\ref{sec:linear} for formal definitions).
In~\cite{} we used linear programming to show a bound of $485$ on the number of vertices
  in such a graph, under additional constraints about the size of its bounding box 
   that follow from rules of Morpion 5T and from a variant of an isoperimetric inequality.
  
In the present paper we shall use additional combinatorial property of Morpion graphs,
  which states that to each move we may assign one of its dots and this assignment may be one-to-one.
We call such graphs \emph{unordered Morpion graphs} (Figure~\ref{fig:85}).
We prove that the
  size of graphs with these combinatorial and geometrical properties does not exceed $85$.
We will also use additional constraint about a size of the bounding box of the graph that follows from
  calculations combined with the rules of the game.
We then use additional argument to show that the graphs of size $85$ do not correspond to Morpion positions.

In~\cite{} we used isoperimetric inequality combined with mixed integer programming to obtain an upper bound of $485$ for Morpion 5T. 
The method employed in~\cite{} does not give useful upper bound for the 5D variant.
In the present paper we use a new technique of gemmating to limit the size of a bounding box of a Morpion 5D graph combined with a new reduction of Morpion Solitaire to a mixed integer programming to obtain a bound of $84$ for Morpion 5D. 
  
\section{Geometry of Morpion Solitaire positions}


Consider a Morpion 5D position shown on the left in Figure~\ref{fig:small}. 
The diagram shows a position of a Morpion 5D playout.
We observe that such a diagram allows us to decode the move sequence,
  even though in the example given there are three diffrent lines passing through dot with number $1$
  and it might be not clear which line corresponds to the first move.
However, the last move (in the given ordering) is unique. 
The rest of the moves can be decoded recursively, descending from the last moves in the sequence.
Observe that some of the moves in the sequence may be interchangeable.
However, the set of the played moves is unique with respect to the positions of the moves.

\begin{figure}
    \includegraphics[width=0.49\textwidth]{figures/small1.pdf}
    \includegraphics[width=0.49\textwidth]{figures/small2.pdf}
    \caption{\label{fig:small}
      A Morpion 5D position (left) and a corresponding unordered Morpion 5D graph (right)
    }
\end{figure}

A bounding box of a Morpion 5D position is the smallest rectangular part of the grid that contains it 
  (with edges parallel to the coordinate axes).
Every Morpion 5D graph contains the starting cross and the smallest bounding box has dimensions $9 \times 9$.
The graph depicted in Figure~\ref{fig:85} has a bounding box with dimensions $14 \times 13$.
However, the dimensions of the bounding box do not give information about the position of the starting cross inside, which is important.
We will employ the following convention.
We will describe a bounding box by distances of its edges from the edges of the cross.
For graph depicted in Figure~\ref{fig:85} the distance from top edge of the cross to the top side of the bounding box is equal to $3$. For right side it is $4$, bottom $1$ and left $1$. 
We write that graph from Figure~\ref{fig:85} has a bounding box $(3,4,1,1)$.
  
There is an $8$-element group of isometries of the $\mathbb{Z}^2$ grid that leave the starting cross in place.
Up to these symmetries every Morpion 5D position is contained in a bounding box $(a,b,c,d)$ such that
  $a \leq b,c,d$ and $b \leq d$.

In the sequel we show that every Morpion 5D graph that corresponds to a Morpion 5D position
    must be contained (up to symmetry) in one of the bounding boxes listed in table~\ref{tbl:boundingboxes}.

\begin{table}[ht]
\centering
%
    \begin{tabular}{|l|l|l|l|}
    \hline
    No &  Bounding box &  Max size  \\
    \hline%
    
1&(3, 4, 1, 1)& 85.0\\
2&(3, 4, 2, 1)& 85.0\\
3&(4, 3, 1, 2)& 85.0\\
4&(4, 3, 1, 3)& 85.0\\
5&(2, 4, 2, 1)& 84.0\\
6&(2, 4, 2, 2)& 84.0\\
7&(2, 5, 1, 2)& 84.0\\
8&(2, 5, 2, 1)& 84.0\\
9&(2, 5, 2, 2)& 84.0\\
10&(3, 3, 1, 2)& 84.0\\
11&(3, 3, 2, 2)& 84.0\\
12&(3, 4, 1, 2)& 84.0\\
13&(3, 4, 1, 3)& 84.0\\
14&(3, 4, 2, 2)& 84.0\\
15&(3, 4, 3, 2)& 84.0\\
16&(4, 3, 0, 2)& 84.0\\
%
    \hline
    \end{tabular}%
\hspace*{5mm}
%
    \begin{tabular}{|l|l|l|l|}
    \hline
    No &  Bounding box &  Max size  \\
    \hline%
    
17&(4, 3, 2, 3)& 84.0\\
18&(2, 3, 2, 1)& 83.0\\
19&(2, 3, 2, 2)& 83.0\\
20&(2, 5, 1, 1)& 83.0\\
21&(3, 2, 1, 2)& 83.0\\
22&(3, 3, 1, 3)& 83.0\\
23&(3, 3, 3, 3)& 83.0\\
24&(3, 4, 2, 3)& 83.0\\
25&(3, 5, 1, 1)& 83.0\\
26&(3, 5, 2, 1)& 83.0\\
27&(4, 3, 0, 3)& 83.0\\
28&(4, 4, 0, 1)& 83.0\\
29&(4, 4, 0, 2)& 83.0\\
30&(4, 4, 1, 1)& 83.0\\
31&(4, 4, 1, 2)& 83.0\\
32&(4, 4, 1, 3)& 83.0\\
33&(4, 5, 1, 2)& 83.0\\
%
    \hline
    \end{tabular}%
 

\caption{}
\label{tbl:boundingboxes}
\end{table}

We will consider a class of graphs that are obtained from Morpion 5T positions by forgetting about 
  the order in which the moves were played.
(...) See Figure~\ref{fig:small} (right) to see a diagram of an unordered 5D graph. (... also define the size of such a graph ... )
We will call every such graph an \emph{unordered Morpion 5T graph}. See section~\ref{sec:linear} for a formal definition.

\begin{figure}[h]
    \includegraphics[width=0.49\textwidth]{figures/85.pdf}
    \includegraphics[width=0.49\textwidth]{figures/94.pdf}
    \caption{\label{fig:85}
      An unordered Morpion 5D graph of size $85$ on the left side
        and unmarked unordered Morpion 5D graph of size $94$ on the right side
    }
\end{figure}

There are Morpion 5D graphs that do not correspond to Morpion 5D positions.
Figure~\ref{fig:85} shows such an example.
We will prove using linear programming that that the maximum size of  an unordered Morpion 5D graph with 
  a bounding box equal to one of the bounding boxes listed in Table~\ref{tbl:boundingboxes}
  is $85$.
There are four bounding boxes with maximal graphs of size $85$. 
For these we formulate mixed integer problems with additional constraints that force the graph to
  be a Morpion 5D position.
These problems are much harder to solve, but with correct choice of solver optimization parameters we are able to show
  that there are no solutions of these problems of size $85$.
This gives us an upper bound of $84$.

In~\cite{} we computed a bound for a different class of graphs that contains all Morpion 5T positions.
These are graphs that can be covered by four-segment lines that are segment-disjoint and such that $4\cdot \# V - \# E = 144$.
We call such graphs \emph{unmarked unordered Morpion graphs} (see section~\ref{sec:linear} for a formal definition).
Considering this class for Morpion 5D  does not give useful bounds. 
Figure~\ref{fig:} shows an example of an unmarked unordered Morpion 5D graph of size $94$.
Moreover, these graphs need not to be connected and therefore the bounding box of such a graph
  can be arbitrarily large.
  
\section{Morpion Solitaire via Linear Programming}
\label{sec:linear}

\begin{definition}
  Let $G = (V, E)$ be a graph. 
  Let $C \subset \mathbb{Z}^2$ be a set of $36$ dots that form an initial cross of Morpion Solitaire.
  We shall use the following notions.
  \begin{enumerate}
    \item A \emph{unit segment} is a segment with endpoints in the square grid $\mathbb{Z}^2$ and length 
    	equal to $1$ in the $\ell_\infty$ metric.
    \item A graph $G$ is a \emph{lattice graph} if $V \subset \mathbb{Z}^2$
      and each edge of $G$ is a unit segment.
    \item Let 
    \[
      \mathcal{M} = \{ \{ (s_1, s_2), (s_2, s_3), (s_3, s_4), (s_4, s_5) \} \colon s_i \in \mathbb{Z}^2, s_{i+1} - s_i = s_i - s_{i-1} \}
    \]
    Elements of $\mathcal{M}$ are called \emph{moves}. 
    Every move consists of four consecutive, distinct, parallel unit segments that intersect at endpoints.
    We let
    \[
      \mathcal{M}(G) = \{ m \in \mathcal{M} \colon m \subset E \}
    \]
    It is the set of all moves that cover edges of $G$. We call $\mathcal{M}(G)$ the set of \emph{moves in $G$}.
    \item A \emph{marked move} is a move $m$ with a selected vertex $d(m)$ that is one of the endpoints
      of its segments, i.e
      \[
      	d(m) \in (s_i, s_{i+1}) \in m.
      \]
    \item We say that a set $\mathcal{M}(G)$ of moves in $G$ with a marking $d \colon \mathcal{M}(G) \to V$ is \emph{5D-disjoint} if
      \begin{enumerate}
        \item If $m_1, m_2 \in \mathcal{M}(G)$ are parallel, then $m_1$, $m_2$ are vertex disjoint.
        \item The set of unmarked vertices $V \setminus d(\mathcal{M}(G))$ forms the initial cross $C$ of Morpion Solitaire.
      \end{enumerate}
  \end{enumerate}
\end{definition}

\begin{definition}
  A lattice graph $G$ is an \emph{unordered Morpion 5D graph} if the set $\mathcal{M}(G)$ of moves of $G$ allows a marking such that it is 5D-disjoint.
\end{definition}

(formulation of linear problem, with boundary conditions)

\begin{theorem}
  graph gives variable valuation
\end{theorem}

\begin{theorem}
  variable valuation gives graph
\end{theorem}

(additional variables and constraints that force move order)

(additional constraints that force center symmetry)

\section{Gemmating process}

  infeasible models, 
  algorithm pseudo-code (python ?),
  discussion of results

\section{Upper bounds}

  discussion of results
  
\subsection{An upper bound of \therecord for Morpion 5D}

  4 boards with result of 85
  solutions of acyclic problem with cutoff=84.9

\subsection{A solution of Morpion 5D with center symmetry}

  grid with upper bounds (todo: picture)
  
\subsection{An upper bound of $222$ for Morpion 5T with center symmetry}

  grid with upper bounds (todo: picture)

\section{Final remarks}
  nrpa on small boards?
  list of boards with upper bound 84 (partially solved?)
  remark about octagonal boards
  how it is different from our 485 result
  other Morpion variants (no starting cross etc)
    (with acyclic solver)

\printbibliography 
%\bibliography{games.bib}
    
\end{document}

