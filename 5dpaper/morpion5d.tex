\documentclass[a4paper,UKenglish]{lipics}
%This is a template for producing LIPIcs articles. 
%See lipics-manual.pdf for further information.
%for A4 paper format use option "a4paper", for US-letter use option "letterpaper"
%for british hyphenation rules use option "UKenglish", for american hyphenation rules use option "USenglish"
% for section-numbered lemmas etc., use "numberwithinsect"
 
\usepackage{microtype}%if unwanted, comment out or use option "draft"
\usepackage{graphicx}
\usepackage{enumerate}
\usepackage{fixltx2e}

\usepackage{array}
\usepackage{epsfig}
\usepackage[all]{xy}
\usepackage{enumerate}
\usepackage{graphicx}
\usepackage{tikz}
\usepackage{xspace}
\usepackage{float}

%%%% Packages needed by Matteo
\usepackage[all]{xy}
\usepackage{stmaryrd}
\usepackage{amssymb}
\usepackage{amsmath}
\usepackage{verbatim}


%%%% End of Packages needed by Matteo

\newcommand{\fun}[3]{\ensuremath{#1\colon #2 \to #3}}
\newcommand{\parfun}[3]{\ensuremath{#1\colon #2 \rightharpoonup #3}}

%%%%%%%%%%%%%%%%%%%%%%%%%%%%%%%%
%
%
% The record
%
%
%%%%%%%%%%%%%%%%%%%%%%%%%%%%%%%%

\newcommand{\therecord}{{$84$}\ }
\newcommand{\theoctagons}{{$126912$}\ }
\newcommand{\theinstances}{{$42889$}\ }

%%%%%%%%%%%%%%%%%%%%%%%%%%%%%%%%
%
%
% The biblatex requirements 
%
%
%%%%%%%%%%%%%%%%%%%%%%%%%%%%%%%%

\usepackage[backend=bibtex]{biblatex}
\bibliography{games.bib}

%%%%%%%%%%%%%%%%%%%%%%%%%%%%%%%
%
%
% Stuff needed for pictures
%
%
%%%%%%%%%%%%%%%%%%%%%%%%%%%%%%%%

%\usepackage[usenames,dvipsnames]{xcolor}
\usepackage{tikz}
\usepackage{xifthen}
\usepackage{intcalc}
\usetikzlibrary{calc}
\usetikzlibrary{arrows}
\usepackage{pgfplots}
\usepackage{wrapfig}
\usepackage[font={scriptsize,it}]{caption}
\usepackage{cutwin}
%\usepackage{picins}


%%%%%%%%%%%%%%%%%%%%%%%%%%%%%%%%
%
%
% Added for the sake of internal 
% references
%
%
%%%%%%%%%%%%%%%%%%%%%%%%%%%%%%%%

\usepackage{enumitem, hyperref}
\makeatletter
\def\namedlabel#1#2{\begingroup
    #2%
    \def\@currentlabel{#2}%
    \phantomsection\label{#1}\endgroup
}

%%%%%%%%%%%%%%%%%%%%%%%%%%%%%%%%
%
%
% Added to have a sane enumeration 
% of conditions in Lemma 4 (o1)-(o8)
%
%
%%%%%%%%%%%%%%%%%%%%%%%%%%%%%%%%

\usepackage{mathtools} 

%%%%%%%%%%%%%%%%%%%%%%%%%%%%%%%%
%
%
% Commands needed by Andrzej
%
%
%%%%%%%%%%%%%%%%%%%%%%%%%%%%%%%%


%\usepackage[english]{babel}
\usepackage[utf8]{inputenc}
\usepackage{graphicx}
%\usepackage{amsthm}
\usepackage{amsmath}
\usepackage[colorinlistoftodos]{todonotes}


\numberwithin{equation}{section}

%\newtheorem{theorem}{Theorem}[section]
%\newtheorem{lemma}[theorem]{Lemma}
%\newtheorem{corollary}[theorem]{Corollary}
%\newtheorem{proposition}[theorem]{Proposition}
%\newtheorem{conjecture}[theorem]{Conjecture}
%\newtheorem*{theorem*}{Theorem}

%\theoremstyle{definition}
%\newtheorem*{definition}{Definition}
%\newtheorem{example}[theorem]{Example}
\newtheorem{xca}[theorem]{Exercise}

%\theoremstyle{remark}
%\newtheorem{remark}[theorem]{Remark}
%\newtheorem{question}[theorem]{Question}
%\newtheorem*{remark*}{Remark}
\newtheorem*{convention*}{Convention}
\newtheorem*{notation*}{Notation}

\DeclareMathOperator{\hull}{hull}
\DeclareMathOperator{\external}{ex}
\DeclareMathOperator{\modifier}{modifier}
\DeclareMathOperator{\internal}{int}
\DeclareMathOperator{\gap}{gap}
\DeclareMathOperator{\param}{param}
\DeclareMathOperator{\bd}{\partial}
\DeclareMathOperator{\pot}{potential}
\DeclareMathOperator{\dt}{dot}
\DeclareMathOperator{\move}{mv}
\DeclareMathOperator{\obj}{obj}
\DeclareMathOperator{\oct}{octagon}


\renewcommand{\comment}[1]{}


\newcommand {\nat}{\mathbb{N}}
\newcommand {\scc}{\mathsf{scc}}
\newcommand {\ord}{\mathsf{ORD}}
\newcommand {\un}{\mathsf{un}}
\newcommand {\head}{\mathsf{head}}
\newcommand {\tail}{\mathsf{tail}}
\newcommand {\ad}{\mathsf{ad}}
\newcommand {\sub}{\mathsf{sub}}
\newcommand {\dom}{\mathsf{dom}}
\newcommand {\rank}{\mathsf{rank}\xspace}
\newcommand {\red}{\mathsf{red}}
\newcommand {\LGA}{\mathsf{LGA}}
\newcommand {\WAA}{\mathsf{WAA}}
\newcommand {\SAA}{\mathsf{SAA}}
\newcommand {\mybox}{\textrm{{\scriptsize $\,\Box\,$}}}
\newcommand {\R}{${\mathcal R}$}
\newcommand {\RR}{{\mathcal R}}
\newcommand {\C}{${\mathcal C}$}
\newcommand {\BC}{\mathsf{BC}}


\newcommand{\dotcup}{\ensuremath{\mathaccent\cdot\cup}}


% Letters and numbers

\newcommand{\mathromnum}[1]{\ensuremath{\mathrm{#1}}\xspace}

\newcommand{\rI}{\mathromnum{I}}
\newcommand{\rII}{\mathromnum{II}}
\newcommand{\rIII}{\mathromnum{III}}
\newcommand{\rIV}{\mathromnum{IV}}
\newcommand{\rV}{\mathromnum{V}}

\newcommand{\mathcalsym}[1]{\ensuremath{\mathcal{#1}}\xspace}

\newcommand{\Cross}{\texttt{Cross}}
\newcommand{\Aa}{\mathcalsym{A}}
\newcommand{\Bb}{\mathcalsym{B}}
\newcommand{\Cc}{\mathcalsym{C}}
\newcommand{\Dd}{\mathcalsym{D}}
\newcommand{\Ee}{\mathcalsym{E}}
\newcommand{\Ff}{\mathcalsym{F}}
\newcommand{\Gg}{\mathcalsym{G}}
\newcommand{\Hh}{\mathcalsym{H}}
\newcommand{\Ii}{\mathcalsym{I}}
\newcommand{\Jj}{\mathcalsym{J}}
\newcommand{\Kk}{\mathcalsym{K}}
\newcommand{\Ll}{\mathcalsym{L}}
\newcommand{\Mm}{\mathcalsym{M}}
\newcommand{\Nn}{\mathcalsym{N}}
\newcommand{\Oo}{\mathcalsym{O}}
\newcommand{\Pl}{\mathcalsym{P}}
\newcommand{\Ql}{\mathcalsym{Q}}
\newcommand{\Rr}{\mathcalsym{R}}
\newcommand{\Ss}{\mathcalsym{S}}
\newcommand{\Tt}{\mathcalsym{T}}
\newcommand{\Uu}{\mathcalsym{U}}
\newcommand{\Vv}{\mathcalsym{V}}
\newcommand{\Ww}{\mathcalsym{W}}
\newcommand{\Xx}{\mathcalsym{X}}
\newcommand{\Yy}{\mathcalsym{Y}}
\newcommand{\Zz}{\mathcalsym{Z}}
\newcommand{\W}{\mathrm{W}}
\newcommand{\coR}{\mbox{co-}\Rr}

%-------------------------------

\newcommand{\trees}{\mathrm{Tr}\xspace}
\newcommand{\runs}{\mathrm{Runs}\xspace}
\newcommand{\otrees}{\mathrm{Tr}(\omega)\xspace}
\newcommand{\WF}{\mathrm{WF}\xspace}


\newcommand{\boldclass}[3]{\ensuremath{\mathbf{#1}^{#2}_{#3}}}
\newcommand{\lightclass}[3]{\ensuremath{{#1}^{#2}_{#3}}}

\newcommand{\borel}{\ensuremath{\mathcal B}\xspace}

\newcommand{\bsigma}[1]{\boldclass{\Sigma}{0}{#1}}
\newcommand{\bpi}[1]{\boldclass{\Pi}{0}{#1}}
\newcommand{\bdelta}[1]{\boldclass{\Delta}{0}{#1}}

\newcommand{\asigma}[1]{\boldclass{\Sigma}{1}{#1}}
\newcommand{\api}[1]{\boldclass{\Pi}{1}{#1}}
\newcommand{\adelta}[1]{\boldclass{\Delta}{1}{#1}}

\newcommand{\esigma}[2]{\lightclass{\Sigma}{#1}{#2}}
\newcommand{\epi}[2]{\lightclass{\Pi}{1}{#1}}
\newcommand{\edelta}[2]{\lightclass{\Delta}{#1}{#2}}


\newcommand{\bc}[1]{\mathcal{BC}({#1})}

\newcommand{\textscsym}[1]{{\sc #1}}

\newcommand{\eqdef}{\stackrel{\mathrm{def}}=}

\newcommand{\N}{\nat}

%\newcommand{\dL}{\mathrm{L}}
%\newcommand{\dR}{\mathrm{R}}

\newcommand{\rmin}{i}
\newcommand{\rmax}{k}

\newcommand{\G}{\mathcal{G}}

\newcommand{\Plab}{{\mathrm{label}}}
\newcommand{\guarantee}{{\mathrm{promise}}}

\newcommand{\Pgar}{X}

\newcommand{\eve}{\ensuremath{\exists}\xspace}
\newcommand{\adam}{\ensuremath{\forall}\xspace}

\newcommand{\ZFC}{{\sc ZFC}}
\newcommand{\LTL}{{\sc LTL}}
\newcommand{\CTL}{{\sc CTL}}
\newcommand{\PCTL}{{\sc PCTL}}
\newcommand{\CTLstar}{{\sc LTL$^\ast$}}
\newcommand{\ZFCMA}{{\sc ZFC+MA$_{\aleph_{1}}$}}
\newcommand{\MA}{{\sc MA$_{\aleph_{1}}$}}
\newcommand{\ZFCVL}{{\sc ZFC+V=L}}
\newcommand{\VL}{{\sc V=L}}
\newcommand{\AD}{{\sc AD}}
\newcommand{\CH}{{\sc CH}}

\newcommand{\powerset}{{\mathcal P}}
\newcommand{\la}{{\langle}}
\newcommand{\ra}{{\rangle}}

\newcommand{\hargame}{{H\Gg}}

\newcommand{\restr}{\!\upharpoonright}

\newcommand{\ignore}[1]{}
\newcommand{\Souslin}{\mathcal A}


\newenvironment{proofof}[1]
	{\vspace{1ex}\noindent{\emph{Proof of #1}}\hspace{0.5em}}
    {\hfill\qed\vspace{1ex}}
    
%%%%%%%%%%%%%%%%%%%%%%%%%%%%%%%%%%%%
%
%
% acknowledgments on the front page
%
%
%%%%%%%%%%%%%%%%%%%%%%%%%%%%%%%%%%%%    

\newcommand*\samethanks[1][\value{footnote}]{\footnotemark[#1]}




% Author macros::begin %%%%%%%%%%%%%%%%%%%%%%%%%%%%%%%%%%%%%%%%%%%%%%%%
\title{{An upper bound of 84 for Morpion Solitaire 5D}}%\footnote{This work was partially supported by someone.}}
\titlerunning{A Morpion 5D bound} %optional, in case that the title is too long; the running title should fit into the top page column

\author[1]{Henryk Michalewski}
\author[1]{Andrzej Nagórko}
\author[1]{Jakub Pawlewicz}
\affil[1]{Department of Mathematics, Informatics and Mechanics\\ University of Warsaw\\ \{H.Michalewski,A.Nagorko,J.Pawlewicz\}@mimuw.edu.pl} %\\ \texttt{open@dummyuni.org}}
\authorrunning{H. Michalewski, A. Nagórko and J. Pawlewicz} %mandatory. First: Use abbreviated first/middle names. Second (only in severe cases): Use first author plus 'et. al.'

\Copyright{Henryk Michalewski, Andrzej Nagórko and Jakub Pawlewicz}%mandatory, please use full first names. LIPIcs license is "CC-BY";  http://creativecommons.org/licenses/by/3.0/

\subjclass{Dummy classification -- please refer to \url{http://www.acm.org/about/class/ccs98-html}}% mandatory: Please choose ACM 1998 classifications from http://www.acm.org/about/class/ccs98-html . E.g., cite as "F.1.1 Models of Computation". 
\keywords{Morpion, linear optmization, relaxation}% mandatory: Please provide 1-5 keywords
% Author macros::end %%%%%%%%%%%%%%%%%%%%%%%%%%%%%%%%%%%%%%%%%%%%%%%%%

%Editor-only macros:: begin (do not touch as author)%%%%%%%%%%%%%%%%%%%%%%%%%%%%%%%%%%
\serieslogo{}%please provide filename (without suffix)
\volumeinfo%(easychair interface)
  {Billy Editor and Bill Editors}% editors
  {2}% number of editors: 1, 2, ....
  {Conference title on which this volume is based on}% event
  {1}% volume
  {1}% issue
  {1}% starting page number
\EventShortName{}
\DOI{10.4230/LIPIcs.xxx.yyy.p}% to be completed by the volume editor
% Editor-only macros::end %%%%%%%%%%%%%%%%%%%%%%%%%%%%%%%%%%%%%%%%%%%%%%%

\begin{document}

\maketitle

\begin{abstract} 
The Morpion Solitaire is a paper-and-pencil single-player game played on a square grid with initial position consisting of $36$ dots.
In each move the player puts a dot on an unused grid position and draws a line that 
  consists of four consecutive segments passing through the dot.
 The goal is to find the longest possible sequence of moves.
There are two main variants of the game: 5T and 5D. 
They have different restrictions on how the moves may intersect.

Providing lower and upper bounds in Morpion Solitaire is a significant computational and mathematical challenge
  that led to a discovery of an important new algorithm.
 Best lower bounds for Morpion Solitaire were found in $2011$ by Rosin \cite{rosin} using a variant of the Monte Carlo method: 
 $178$ moves for the 5T variant, $82$ moves for the 5D variant.
 
% In the Morpion 5T variant an upper bound of $705$ was proved in $2006$ by Demaine at al. \cite{demaine}  using a method of potential. 
% A refinement of this approach was used in $2015$  in \cite{ijcai} to show a bound of $485$ - this method combined an isoperimetric inequality with linear programming. 

In the Morpion 5D variant 
Kawamura et al. proved in \cite{japonczycy} an upper bound of $121$ moves using a geometrical argument. 
We improve this bound to \therecord. 
This is done in two steps.
1) We state, using linear constraints, a geometric property that defines a class of graphs that includes all Morpion  positions - finding practically solvable linear constraints is a main conceptual advance of the present paper.
2) We solve the mixed integer problems and obtain 
an upper bound of \therecord for Morpion 5D. %The same method a) fully solves Morpion 5D with center symmetry (the optimal sequence is $68$ moves long) and b) gives an upper bound of $222$ for Morpion 5T with center symmetry. 
\end{abstract}

\section{Introduction}
The Morpion Solitaire is a paper-and-pencil single-player game played on a square grid with 
  the initial configuration of 36 dots depicted in Figure~\ref{fig:initial}. 
In each move the player puts a dot on an unused grid position and draws a line that 
  consists of four consecutive segments passing through the dot. 
The line must be horizontal, vertical or diagonal. 
The goal is to find the longest possible sequence of moves.
There are two main variants of the game: 5T and 5D. 
They have different restrictions on how the moves may be placed.
In the 5T variant of the game, no segment may be drawn twice, i.e. the moves have to be segment-disjoint. 
In the 5D variant of the game, any two moves in the same direction have to be dot-disjoint.
The difference is demonstrated in Figure~\ref{fig:small}.
The 5D variant is more restrictive. In Morpion 5D there has to be a gap between moves
  placed on a same line.


  \begin{figure}
    \centering
      \includegraphics[width=0.49\textwidth]{figures/empty.pdf}
      \includegraphics[width=0.49\textwidth]{figures/empty.pdf}
      \caption{\label{fig:initial}
	The initial position of Morpion Solitaire is depicted on the right. On the left there is a position which is up to $4$--th move legal 
both in Morpion 5D and Morpion 5T variants, but the $5$--th move is legal only in Morpion 5T. 
      }
\end{figure}

The problem is notoriously hard for computers. 
For $34$ years the longest known sequence in the Morpion 5T game
  was one of 170 moves discovered by C.-H. Bruneau in $1976$. 
Despite considerable computational effort, until $2010$ the computer generated
  sequences were much shorter.
In $2010$, a Nested Rollout Policy Adapation algorithm (a Monte Carlo variant) 
  was developed by Christopher D. Rosin~\cite{rosin} recognized as a best paper of the IJCAI conference in 2011.
Using NRPA, Rosin obtained the current world record of $178$ moves in Morpion 5T 
  and of $82$ moves in Morpion 5D.
The webpage~\cite{boyer} maintained by Christian Boyer, contains an extensive and up-to-date information about records in all Morpion Solitaire variants.

As the rules of Morpion Solitaire do not limit the size of the grid on which the game is played, a priori
  it is not clear if the sequences have to be bounded.
A popular magazine \emph{Science \& Vie} published in $1970$'s different bounds submitted by its readers 
  for the maximal length of a sequence in Morpion 5T
  (the bounds ranged from $540$ to $20736$), but without detailed and/or valid proofs.
The first rigorously proved bound of $705$ in Morpion 5T was published in $2006$ \cite[Demaine et al.]{demaine}.
The best known bound of $485$ in Morpion 5T was proved in~\cite{ijcai}.
In this paper we prove an upper bound of $84$ on the length of Morpion 5D sequence, 
improving upon a bound of $121$ found earlier by \cite[Kawamura et al.]{japonczycy}.




Proofs of upper bounds discussed above exploit geometric and combinatorial properties of graphs that are obtained as Morpion Solitaire positions. 
To set the mood we'll discuss the bound of $705$ moves for Morpion 5T as it was proved in~\cite{demaine}.
A position of a Morpion Solitaire gameplay (Figure~\ref{fig:small}) has a graph structure.
Its vertices are placed in the $\mathbb{Z}^2$ grid.
Let $n$ denote the number of vertices, 
  corresponding both to the dots placed by moves and to the dots from the starting cross.
The edges correspond to the segments placed on the grid by moves.
Every move adds a single vertex and four edges to the graph.
Therefore a Morpion position graph has the following two properties: 1) its edges have unit length in $\ell_\infty$ metric; 2) $4n - e = 4 \cdot 36 = 144$. \todo{I do not like the idea of putting here $n$, $e$, $l_\infty$ and basically I am afraid that someone less patient may stop reading here without getting to the main result - maybe we can make a picture instead of this?}
In~\cite{brass} P. Brass proved that if a planar graph $G$ has $n$ vertices, 
  then the maximum number of edges in $G$ that have unit length in $\ell_\infty$ metric is equal to
  $
    s(n) = \lfloor 4n - \sqrt{28n - 12} \rfloor.
  $
Hence
$
  4n - 144 = e \leq s(n) = \lfloor 4n - \sqrt{28n - 12} \rfloor.
$
The maximal $n$ that satisfies this inequality is $n = 741$. 
Considering the initial $36$ dots this gives an upper bound of $705$ on the number of moves 
  in Morpion Solitaire~\cite{demaine}.
  
Observe that the Morpion position graph has additional geometrical property.
The set of its edges may be covered by a set of segment-disjoint \emph{moves} consisting of four consecutive, parallel, distinct unit length segments. 
We call such graphs \emph{unmarked unordered Morpion graphs} (see Section~\ref{sec:linear} for formal definitions).
% In~\cite{ijcai} we used linear programming to show a bound of $485$ on the number of vertices in such a graph, under additional constraints about the size of its bounding box  that follow from rules of Morpion 5T and from a variant of an isoperimetric inequality. %\todo{The same stuff as one paragraph below}
  
 %\todo{Insert why the game is epxressible via a linear program}
As a singer player game Morpion Solitaire can be in a natural way encoded as a linear optimization problem with the optimization target being the length of the sequence. This information is not very helpful, because the problem is too large to be practically solvable. However, this inspires a natural approach towards construction of upper bounds. We consider a generalization of  Morpion Solitaire such that every gameplay of ordinary  Morpion Solitaire is a gameplay of the generalization. Moreover, we expect from such a generalization to be practically solvable and so that solutions of such generalizations provide interesting upper bounds with respect to ordinary Morpion Solitaire. We do not expect from generalizations to be playable by humans. With help of linear optimizations we will significantly improve on the above geometric bounds. 
A certain generlization of Morpion Solitaire 5T we analyzed in \cite{ijcai}, however this method does not seem very helpful in Morpion 5D --- later in this paper we discuss how these generalizations differ one with another. 
  
In the present paper we shall use additional combinatorial property of Morpion graphs:
%  which states that 
one can find an assignment such that to each move is assigned one of its dots and the assignment is one-to-one (see Section \ref{sec:linear} for a precise definition).
We call such graphs \emph{unordered Morpion graphs}. % (Figure~\ref{fig:85}).
Every Morpion 5D gameplay generates such assignment but there are assignments 
which are not generated by any gameplay (see example of such situation in Figure\ref{fig:85})

\todo{The following paragraph should be written in a different way --- emphasize Theorem, maybe Lemmas, the table} 
We prove that the
  size of graphs with these combinatorial and geometrical properties does not exceed $85$.
We will also use additional constraint about a size of the bounding box of the graph that follows from
  calculations combined with the rules of the game.
We then use additional argument to show that the graphs of size $85$ do not correspond to Morpion positions.

%In~\cite{ijcai} we used isoperimetric inequality combined with mixed integer programming to obtain an upper bound of $485$ for Morpion 5T. 
%The method employed in~\cite{ijcai} does not give useful upper bound for the 5D variant.
%In the present paper we use a new technique of gemmating to limit the size of a bounding box of a Morpion 5D graph combined with a new reduction of Morpion Solitaire to a mixed integer programming to obtain a bound of $84$ for Morpion 5D. \todo{List precisely results which are important for this paper}

  
\section{Geometry of Morpion Solitaire positions}

%\todo{An introductory sentence}
In this section we will collect a number of observations regarding geometric properties of Morpion Solitaire positions. Let us start 
from a consideration how to identify from the shape of the graph of Morpion Solitaire all moves in a given gameplay starting from the last one. 

\begin{example}
Consider a Morpion 5D position shown on the left in Figure~\ref{fig:small}. 
The diagram shows a position of a Morpion 5D gameplay.
We observe that %such a diagram allows us to decode the move sequence 
  even though 
  \begin{itemize}
    \item   in the example in Figure~\ref{fig:small} there are three diffrent lines passing through dot with number $1$ and it might be not clear which line corresponds to the first move\todo{Correct the figure - it is about number 5 not number 1}, however
    \item   the last move is unique; in Figure~\ref{fig:small} this is move $19$ and we can identify the whole move thanks to the fact  that dot $19$ was not used in previous moves; the rest of the moves can be decoded recursively, descending from the last move in the sequence. \footnote{This simple observation was generalized to a non-trivial algorithm which not recovers the sequence of moves starting from the last one, but also recovers a correct numbering of moves, in particular identifies the last move \cite{demaine}.} %\todo{Maybe a footnote about Demain's generalization}
  \end{itemize}
\end{example}

%Observe that some of the moves in the sequence may be interchangeable.
%However, the set of the played moves is unique with respect to the positions of the moves.

\begin{figure}
    \includegraphics[width=0.49\textwidth]{figures/small1.pdf}
    \includegraphics[width=0.49\textwidth]{figures/small2.pdf}
    \caption{\label{fig:small}
      A Morpion 5D position (left) and a corresponding unordered Morpion 5D graph (right)
    }
\end{figure}

\begin{definition}
A box is a rectangular subset of ${\mathbb Z}^2$ bounded by $4$ edges parallel to the coordinate axes. A bounding box of a Morpion 5D position is the smallest box
containing this position. %rectangular part of the grid that contains this position
%edges parallel to the coordinate axes). 
\todo{Add a precise figure with a bounding box and dimensions, e.g. Figure~\ref{fig:small}}
\end{definition}

Every Morpion 5D graph contains the starting cross, hence the smallest bounding box has dimensions $9 \times 9$.
The graph depicted in Figure~\ref{fig:85} has a bounding box with dimensions $14 \times 13$.
However, the dimensions of the bounding box do not give information about the position of the starting cross inside. %, which is important.
%We will employ the following convention.

\begin{notation*}
The distance of a bounding box to the initial cross can be described by $4$ numbers: the distances of the sides of the box to the edges of the cross. 
% is distances of its edges from the edges of the cross.
For graph depicted in Figure~\ref{fig:85} the distance from top edge of the cross to the top side of the bounding box is equal to $3$. For the right side it is $4$, for the bottom side it is $1$ and for the left side it is also $1$. 
The bounding box of the position in Figure~\ref{fig:85} is described as $(3,4,1,1)$.
\end{notation*}
  
There is an $8$-element group of isometries of the $\mathbb{Z}^2$ grid that leave the starting cross in place.
Up to these symmetries every Morpion 5D position is contained in a bounding box $(a,b,c,d)$ such that
  $a \leq b,c,d$ and $b \leq d$. Below we formulate the key thoerem in this paper: %  \todo{Theorem, proved in Section 4, why is it important}
\begin{theorem}
\begin{enumerate}
\item Up to symmetry every Morpion 5D graph that corresponds to a Morpion 5D position
    is contained in one of the bounding boxes listed in table~\ref{tbl:boundingboxes}\todo{The actual table is slightly longer, there are some errors 83 -> 82}.
\item For every row in the table $(a,b,c,d)$ and for every Morpion 5D graph contained in the bounding box  $(a,b,c,d)$,
its size does not exceed the maximal size indicated in the table.
\end{enumerate} 
\label{thm:boxes}
\end{theorem}
We will present a proof of this theorem in Section \ref{sec:gemmating} along with a precise notion of a Morpion 5D graph. An intuitive 
idea is that Morpion 5D graphs are obtained from Morpion 5D positions by forgetting about the order in which the moves were played.

\begin{table}[ht]
\centering
%
    \begin{tabular}{|l|l|l|l|}
    \hline
    No & BBox & Size  \\
    \hline%
    
1&(4, 3, 1, 1)& 85.0\\
2&(4, 3, 1, 2)& 85.0\\
3&(4, 3, 1, 3)& 85.0\\
4&(4, 2, 1, 2)& 84.0\\
5&(4, 2, 2, 2)& 84.0\\
6&(5, 2, 2, 1)& 84.0\\
7&(5, 2, 1, 2)& 84.0\\
8&(5, 2, 2, 2)& 84.0\\
9&(3, 3, 2, 1)& 84.0\\
10&(3, 3, 2, 2)& 84.0\\
%
    \hline
    \end{tabular}%
\hspace*{5mm}
%
    \begin{tabular}{|l|l|l|l|}
    \hline
    No & BBox & Size  \\
    \hline%
    
11&(4, 3, 2, 1)& 84.0\\
12&(4, 3, 3, 1)& 84.0\\
13&(4, 3, 2, 2)& 84.0\\
14&(4, 3, 2, 3)& 84.0\\
15&(4, 3, 0, 2)& 84.0\\
16&(3, 2, 1, 2)& 83.0\\
17&(3, 2, 2, 2)& 83.0\\
18&(5, 2, 1, 1)& 83.0\\
19&(3, 3, 3, 1)& 83.0\\
20&(3, 3, 3, 3)& 83.0\\
%
    \hline
    \end{tabular}%
\hspace*{5mm}
%
    \begin{tabular}{|l|l|l|l|}
    \hline
    No & BBox & Size  \\
    \hline%
    
21&(4, 3, 3, 2)& 83.0\\
22&(5, 3, 1, 1)& 83.0\\
23&(5, 3, 1, 2)& 83.0\\
24&(4, 3, 0, 3)& 83.0\\
25&(4, 4, 1, 0)& 83.0\\
26&(4, 4, 2, 0)& 83.0\\
27&(4, 4, 1, 1)& 83.0\\
28&(4, 4, 2, 1)& 83.0\\
29&(4, 4, 3, 1)& 83.0\\
30&(5, 4, 2, 1)& 83.0\\
%
    \hline
    \end{tabular}%
 

\caption{Bounding boxes mentioned in Theorem \ref{thm:boxes} for sizes $85$, $84$ and $83$. All bounding boxes are listed in the Appendix. }
\label{tbl:boundingboxes}
\end{table}

%We will consider a class of graphs that are obtained from Morpion 5D positions by forgetting about 
%  the order in which the moves were played. %\todo{Definition of an unordered Morpion 5D graph}


%We will prove using linear programming that that the maximum size of  an unordered Morpion 5D graph with a bounding box equal to one of the bounding boxes listed in Table~\ref{tbl:boundingboxes}
% is $85$. \todo{This looks like a comment which should immediatelly after formulation of the theorem: it says, that we are not dealing
% with all morpion 5d graphs, but only with graphs generated by Morpion 5D positions - it is enough for the corollary that 85 is an upper bound -
% later there will be one more corollary improving on this bound to 84}

\begin{corollary}
\label{cor:84}
The longest sequence of moves in Morpion 5D does not exceed $84$.
\end{corollary}
\begin{proof} 
In table \ref{tbl:boundingboxes} there are four bounding boxes with maximal graphs of size $85$. 
For these we formulate mixed integer problems with additional constraints that force the graph to
  be a Morpion 5D position.
These problems are much harder to solve, but with correct choice of solver optimization parameters we are able to show
  that there are no solutions of these problems of size $85$.
\end{proof}
% This gives us an upper bound of $84$. \todo{formulate as a corollary}




\section{Morpion Solitaire via Linear Programming}
\label{sec:linear}

%\begin{definition}
  Let $\Cross \subset \mathbb{Z}^2$ be a set of $36$ dots that form an initial cross of Morpion Solitaire.
%  We shall use the following notions.
\begin{definition}
The maximum metric in the square grid $\mathbb{Z}^2$ is defined as $\max (|s_1^1-s_2^1|,|s_1^2-s_2^2|)$
for $s_1,s_2\in \mathbb{Z}^2$. 
A \emph{unit segment} is a pair of points $\{s_1,s_2\}\in \mathbb{Z}^2$ in distance $1$ in the maximum metric. % $s_1\neq s_2$, $s_i = (s_i^1,s_i^2)$, $i=1,2$
 %such that $\max{|s_1^1-s_2^1|,|s_1^2-s_2^2|}=1$. That is, a unit segment is a pair of points in the
%in the square grid $\mathbb{Z}^2$ of length $1$ in the $\ell_\infty$ metric.
\end{definition}


%\begin{definition} 
\subsection{Definition of moves}
Let 
    $
      \mathcal{M} = \{ \{ \{s_1, s_2\}, \{s_2, s_3\}, \{s_3, s_4\}, \{s_4, s_5\} \} \colon s_i \in \mathbb{Z}^2, s_{i+1} - s_i = s_i - s_{i-1} \}.
    $
    Elements of $\mathcal{M}$ are called \emph{moves}. 
Notice, that moves consists of four consecutive, distinct, parallel unit segments that intersect at endpoints. 

Up to a sign, a given move has one of four directions $(1,0)$, $(0,1)$, $(1,1)$, $(1,-1)$ determined by
the difference $s_{2} - s_1 = s_3 - s_2 = s_4 - s_3 = s_5 - s_4$. 
We call two moves $\{ \{s_1, s_2\}, \{s_2, s_3\}, \{s_3, s_4\}, \{s_4, s_5\} \}$ and $\{ \{t_1, t_2\}, \{t_2, t_3\}, \{t_3, t_4\}, \{t_4, t_5\} \}$ 
in $\mathcal{M}$ {\em parallel} if $s_1-s_2 = \pm (t_1-t_2)$.  
% \end{definition}

\subsection{Definition of Morpion 5D graphs}
In the following 
three definitions  $G = (V, E)$ is a fixed graph. 
\begin{definition} 
\begin{itemize}
\item We call $G$  a \emph{lattice graph} if $V \subset \mathbb{Z}^2$
      and each edge of $G$ is a unit segment.
    We let
    $
      \mathcal{M}(G) = \{ m \in \mathcal{M} \colon m \subset E \}
    $
    It is the set of all moves that are contained in the edges of $G$. We call $\mathcal{M}(G)$ the set of \emph{moves in $G$}.
\item Let $d \colon \mathcal{M}(G) \to V$ be an arbitrary mapping. We call $d$ a marking if for every $m\in \mathcal{M}(G)$ holds
      $
      	d(m) \in \{s_i, s_{i+1}\} \in m
      $
    for some $i\in \{1,2,3,4\}$.
\end{itemize}
\end{definition}

Below we define a concept of a Morpion 5D graph, which is a lattice graph with a special marking of vertices. Every Morpion 5D position is
a Morpion 5D graph, but as we discuss below, there are Morpion 5D graphs not related to any Morpion 5D position.

\begin{definition}
\label{def:disjoint}
   %  A \emph{marked move} $m\in \mathcal{M}(G)$ is the pair $(m,d(m))$.%  with a selected vertex $d(m)$ that is one of the endpoints of its segments, i.e
    We say that the set $\mathcal{M}(G)$ of moves in $G$ with a marking $d \colon \mathcal{M}(G) \to V$ is \emph{5D-disjoint} if
      \begin{enumerate}
        \item $\bigcup \mathcal{M}(G) = E$,
        \item for every $m_1, m_2 \in \mathcal{M}(G)$ parallel, the moves $m_1$, $m_2$ are vertex disjoint.
        \item $V \setminus d(\mathcal{M}(G)) = \Cross$, that is the set of unmarked vertices $V \setminus d(\mathcal{M}(G))$ forms the initial cross of Morpion Solitaire. \label{markings}
      \end{enumerate}
\end{definition}

\begin{definition}
  A lattice graph $G$ is an \emph{unordered Morpion 5D graph} if there exists a 5D-disjoint marking of $\mathcal{M}(G)$. 
\end{definition}

Figure~\ref{fig:small} (right) contains an example of an unordered Morpion 5D graph. \todo{We have to decide either for ``unordered Morpion 5D graph'' or ``Morpion 5D graph'' --- currently we have both} %(see Section~\ref{sec:linear} for definitions an  unordered Morpion 5D graph and unmarked unordered Morpion 5D graph).
 The size of such a graph  is defined as the number of vertices minus $36$, that is minus the number of vertices in the \Cross. %\todo{Define the size of such a graph - number of vertices - $36$}.
%We will call every such graph an \emph{unordered Morpion 5T graph}. See section~\ref{sec:linear} for a formal definition.

\begin{figure}[h]
    \includegraphics[width=0.531\textwidth]{figures/85.pdf}
    \includegraphics[width=0.4645\textwidth]{figures/94.pdf}
    \caption{\label{fig:85}
      An unordered Morpion 5D graph of size $85$ on the left side
        and unmarked unordered Morpion 5D graph of size $94$ on the right side. 
    }
\end{figure}

\begin{figure}
  \centering
  \includegraphics[width=0.687\textwidth]{figures/uncon.pdf}
  \caption{
    Unmarked unordered Morpion 5D graph may have arbitrarily large bounding box.
  }
  \label{fig:uncon}
\end{figure}

\begin{remark}
a) There are Morpion 5D graphs that do not correspond to Morpion 5D positions. %\todo{Unmarked defined?}
Figure~\ref{fig:85} shows such an example. The graph on the left is not a Morpion 5D position, 
because every Morpion 5D position admits the last move, characterized by the fact that the last vertex is of degree $1$ or $2$ and if it is
of degree $2$, then the neighbours must be located on a straight line. There is no such vertex in the graph on the left.
%\todo{Inline comments from Figure}
%\end{remark}
%
%\begin{remark}

\noindent
b) If one drops the condition \ref{markings} from Definition \ref{def:disjoint} then we would obtain a concept of a umarked unordered Morpion 5D graph.\todo{Is that correct?} A Morpion 5T equivalent
of this definition was used in~\cite{ijcai} to compute a bound for Morpion 5T.
% These are graphs that can be covered by four-segment lines that are segment-disjoint and such that $4\cdot \# V - \# E = 144$.
%We call such graphs \emph{unmarked unordered Morpion graphs} (see section~\ref{sec:linear} for a formal definition).
Considering this class for Morpion 5D  does not give useful bounds. There are two reasons why this method does not work:
\begin{itemize}
\item these graphs need not to be connected and therefore the bounding box of such a graph
  can be arbitrarily large, see Figure \ref{fig:uncon}, 
\item even if we insist on connectedness, there are examples of unmarked unordered Morpion 5D graphs of size exceeding sizes in 
table \ref{tbl:boundingboxes}, see Figure~\ref{fig:85}. In this specific example, the graph has $94$ vertices.
\end{itemize}
\end{remark}

(formulation of linear problem, with boundary conditions)

\begin{lemma}
  graph gives variable valuation
\end{lemma}

\begin{lemma}
  variable valuation gives graph
\end{lemma}

(additional variables and constraints that force move order)

(additional constraints that force center symmetry)

  
\section{Gemmating process}

  infeasible models, 
  algorithm pseudo-code (python ?),
  discussion of results



\section{Upper bounds}
\label{sec:upper}

%  discussion of results
  
\subsection{An upper bound of \therecord for Morpion 5D}
\label{subsec:84}

% We summarize arguments which amounts to the proof of main Theorem \ref{thm:boxes}
% \begin{proof}
% We use Lemma about resizing and then Lemma about coding as linear programs. 
% \end{proof}
In order to prove Theorem \ref{thm:boxes} we apply the process described in Section \ref{sec:gemmating} to produce all bounding boxes relavent 
to Morpion 5D game. We solve the linear problem \L{1}-L{5} specified in Definition \ref{def:mip} in all these boxes. Thanks to 
Lemmas \ref{lem:solutions} and \ref{lem:graphs}, the solutions of linear programs are in one-to-one corespondences with existence of
Moprion 5D graphs. This concludes the proof of Theorem \ref{thm:boxes}.


Theorem \ref{thm:boxes} implies the upper bound of $85$ which follows from solving the optimization
problem \L{1}-L{5} on all boxes generated in Section \ref{sec:gemmating}. 
In order to improve the result to $84$ we have to consider three bounding boxes which admit marked Morpion 5D graphs of size $85$. These
are specifically boxes $(4,3,1,1)$, $(4,3,1,2)$, $(4,3,1,3)$.
On these three boards %we solve a more difficult problem with the cutoff condition set to $84.9$ --- 
not only we require that a given solution is a 
Morpion 5D graph (conditions \L{1}-L{5}) of Definition \ref{def:mip}), but additionally we also require that the graph conforms to condition \L{7} of Definition \ref{def:mip}. 
By Lemmas \ref{lem:solutions} and \ref{lem:graphs} this means that we try to solve the full Morpion 5D game on the bounding boxes  $(4,3,1,1)$, $(4,3,1,2)$, $(4,3,1,3)$.
In general it does not seem to be a feasible task, but we succeeded to complete the computations with the cutoff condition of $84.9$.
%\todo{Explain what is acyclity - covered in a Lemma}

\subsection{A solution of Morpion 5D with center symmetry}
We consider Morpion 5D positions which are invariant with respect to $8$ symmetries 
specified in Subsection \ref{subsec:symmetries}. We say that a Morpion 5D position or a graph is {\em symmetric} is invariant with respect to these symmetries.  
In analogy to Theorem \ref{thm:boxes} we can formulate 
\begin{theorem}
\begin{enumerate}
\item Up to symmetry every symmetric Morpion 5D graph that corresponds to a symmetric Morpion 5D position
    is contained in one of the bounding boxes listed in table~\ref{tbl:sym_boundingboxes}.
\item For every row in the table $(a,b,c,d)$ and for every symmetric Morpion 5D graph contained in the bounding box  $(a,b,c,d)$,
its size does not exceed the maximal size indicated in the table.
\end{enumerate} 
\label{thm:sym_boxes}
\end{theorem}

The proof is analogous to the proof of Theorem \ref{thm:boxes}. In the proof we additionally use the condition \L{6} of Definition \ref{def:mip}.

\begin{table}[ht]
\centering
%
    \begin{tabular}{|l|l|l|l|}
    \hline
    No & BBox & Size  \\
    \hline%
    
1&(4, 3, 1, 1)& 85.0\\
2&(4, 3, 1, 2)& 85.0\\
3&(4, 3, 1, 3)& 85.0\\
4&(4, 2, 1, 2)& 84.0\\
5&(4, 2, 2, 2)& 84.0\\
6&(5, 2, 2, 1)& 84.0\\
7&(5, 2, 1, 2)& 84.0\\
8&(5, 2, 2, 2)& 84.0\\
9&(3, 3, 2, 1)& 84.0\\
10&(3, 3, 2, 2)& 84.0\\
%
    \hline
    \end{tabular}%
\hspace*{5mm}
%
    \begin{tabular}{|l|l|l|l|}
    \hline
    No & BBox & Size  \\
    \hline%
    
11&(4, 3, 2, 1)& 84.0\\
12&(4, 3, 3, 1)& 84.0\\
13&(4, 3, 2, 2)& 84.0\\
14&(4, 3, 2, 3)& 84.0\\
15&(4, 3, 0, 2)& 84.0\\
16&(3, 2, 1, 2)& 83.0\\
17&(3, 2, 2, 2)& 83.0\\
18&(5, 2, 1, 1)& 83.0\\
19&(3, 3, 3, 1)& 83.0\\
20&(3, 3, 3, 3)& 83.0\\
%
    \hline
    \end{tabular}%
\hspace*{5mm}
%
    \begin{tabular}{|l|l|l|l|}
    \hline
    No & BBox & Size  \\
    \hline%
    
21&(4, 3, 3, 2)& 83.0\\
22&(5, 3, 1, 1)& 83.0\\
23&(5, 3, 1, 2)& 83.0\\
24&(4, 3, 0, 3)& 83.0\\
25&(4, 4, 1, 0)& 83.0\\
26&(4, 4, 2, 0)& 83.0\\
27&(4, 4, 1, 1)& 83.0\\
28&(4, 4, 2, 1)& 83.0\\
29&(4, 4, 3, 1)& 83.0\\
30&(5, 4, 2, 1)& 83.0\\
%
    \hline
    \end{tabular}%
 

\caption{Bounding boxes mentioned in Theorem \ref{thm:sym_boxes} for sizes $68$, $67$ and $65$. All bounding boxes are listed in the Appendix. }
\label{tbl:sym_boundingboxes}
\end{table}

\begin{corollary}
\label{cor:68}
The longest sequence of moves leading to a symmetric Morpion 5D position is equal to $68$.
\end{corollary}
\begin{proof} 
The upper bound follows from Theorem \ref{thm:sym_boxes}.
A record consisting of $68$ moves was found by M.~Quist and is presented in \cite{boyer}. \todo{It may be worthwhile to reproduce it and comment can it be found by a computer}
\end{proof}

We are not aware of any previous upper bounds for symmetric Morpion 5D or Morpion 5T.

\ignore{
\subsection{An upper bound of $222$ for Morpion 5T with center symmetry}

In order to improve the result from $85$ to $8$ we have to consider three cases of bounding boxes which allow marked Morpion 5D graphs of size $85$. These
are specifically boxes $(4,3,1,1)$, $(4,3,1,2)$, $(4,3,1,3)$  
On these three boards we solve a more difficult problem with the cutoff condition set to $84.9$ --- not only we require that a given solution is a 
Morpion 5D graph, but additionally we also require that the graph is {\em acyclic}. \todo{Explain what is acyclity}

\subsection{A solution of Morpion 5D with center symmetry}
As in the previous subsection we consider Morpion 5T positions which are invariant with respect to $8$ symmetries 
specified in Subsection \ref{subsec:symmetries}. %We say that a Morpion 5D position or a graph is {\em symmetric} is invariant with respect to these symmetries.  
In analogy to Theorems \ref{thm:boxes} and \ref{thm:sym_boxes} we can formulate 
\begin{theorem}
\begin{enumerate}
\item Up to symmetry every symmetric Morpion 5T graph that corresponds to a symmetric Morpion 5T position
    is contained in one of the bounding boxes listed in table~\ref{tbl:sym_5t_boundingboxes}.
\item For every row in the table $(a,b,c,d)$ and for every symmetric Morpion 5T graph contained in the bounding box  $(a,b,c,d)$,
its size does not exceed the maximal size indicated in the table.
\end{enumerate} 
\label{thm:sym_5t_boxes}
\end{theorem}

The proof is analogous to the proof of Theorem \ref{thm:boxes}. % and \ref{thm:sym_5t_boxes}. 

\begin{table}[ht]
\centering
%
    \begin{tabular}{|l|l|l|l|}
    \hline
    No & BBox & Size  \\
    \hline%
    
1&(4, 3, 1, 1)& 85.0\\
2&(4, 3, 1, 2)& 85.0\\
3&(4, 3, 1, 3)& 85.0\\
4&(4, 2, 1, 2)& 84.0\\
5&(4, 2, 2, 2)& 84.0\\
6&(5, 2, 2, 1)& 84.0\\
7&(5, 2, 1, 2)& 84.0\\
8&(5, 2, 2, 2)& 84.0\\
9&(3, 3, 2, 1)& 84.0\\
10&(3, 3, 2, 2)& 84.0\\
%
    \hline
    \end{tabular}%
\hspace*{5mm}
%
    \begin{tabular}{|l|l|l|l|}
    \hline
    No & BBox & Size  \\
    \hline%
    
11&(4, 3, 2, 1)& 84.0\\
12&(4, 3, 3, 1)& 84.0\\
13&(4, 3, 2, 2)& 84.0\\
14&(4, 3, 2, 3)& 84.0\\
15&(4, 3, 0, 2)& 84.0\\
16&(3, 2, 1, 2)& 83.0\\
17&(3, 2, 2, 2)& 83.0\\
18&(5, 2, 1, 1)& 83.0\\
19&(3, 3, 3, 1)& 83.0\\
20&(3, 3, 3, 3)& 83.0\\
%
    \hline
    \end{tabular}%
\hspace*{5mm}
%
    \begin{tabular}{|l|l|l|l|}
    \hline
    No & BBox & Size  \\
    \hline%
    
21&(4, 3, 3, 2)& 83.0\\
22&(5, 3, 1, 1)& 83.0\\
23&(5, 3, 1, 2)& 83.0\\
24&(4, 3, 0, 3)& 83.0\\
25&(4, 4, 1, 0)& 83.0\\
26&(4, 4, 2, 0)& 83.0\\
27&(4, 4, 1, 1)& 83.0\\
28&(4, 4, 2, 1)& 83.0\\
29&(4, 4, 3, 1)& 83.0\\
30&(5, 4, 2, 1)& 83.0\\
%
    \hline
    \end{tabular}%
 

\caption{Bounding boxes mentioned in Theorem \ref{thm:sym_5t_boxes} for sizes $222$, $221$ and $220$. All bounding boxes are listed in the Appendix. }
\label{tbl:sym_5t_boundingboxes}
\end{table}

\begin{corollary}
\label{cor:222}
The longest sequence of moves leading to a symmetric Morpion 5D position does not exceed $222$.
\end{corollary}
\begin{proof} 

\end{proof}

\begin{remark}
A record of 136 moves in symmetric Morpion 5T was found by M.~Quist and is presented in \cite{boyer}. We are not aware of any previous upper bounds for symmetric Morpion 5D or Morpion 5T.
\end{remark}
}




\section{Final remarks}
  nrpa on small boards?
  list of boards with upper bound 84 (partially solved?)
  remark about octagonal boards
  how it is different from our 485 result
  other Morpion variants (no starting cross etc)
    (with acyclic solver)


\printbibliography 

\newpage
\appendix

\section{Symmetry}

\todo{As it is written now it does not make any good - we cannot expect that anyone will try to guess what this code is doing}
\begin{lstlisting}[language = Python,
  basicstyle=\ttfamily\scriptsize,keywordstyle=\color{red},backgroundcolor=\color{white}]
def symmetryClass(b):
    if b[0] == b[1] and b[1] == b[2] and b[2] == b[3]:
        return b
    
    while b[0] < b[1] or b[0] < b[2] or b[0] < b[3]:
        b = b[1:] + b[:1]
    
    if (b[1] < b[3]):
        b[1], b[3] = b[3], b[1]

    if (b[1] < b[0] and b[2] < b[0] and b[3] < b[0]):
        return b
        
    if b[0] == b[3] or b[0] == b[1]:
        while not b[1] == b[0]:
            b = b[1:] + b[:1]
        if (b[2] < b[3]):
            b[2], b[3] = b[3], b[2]
    return b
\end{lstlisting}

    
\end{document}

