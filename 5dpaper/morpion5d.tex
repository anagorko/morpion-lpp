\documentclass[a4paper,UKenglish]{lipics}
%This is a template for producing LIPIcs articles. 
%See lipics-manual.pdf for further information.
%for A4 paper format use option "a4paper", for US-letter use option "letterpaper"
%for british hyphenation rules use option "UKenglish", for american hyphenation rules use option "USenglish"
% for section-numbered lemmas etc., use "numberwithinsect"
 
\usepackage{microtype}%if unwanted, comment out or use option "draft"
\usepackage{graphicx}
\usepackage{enumerate}
\usepackage{fixltx2e}

\usepackage{array}
\usepackage{epsfig}
\usepackage[all]{xy}
\usepackage{enumerate}
\usepackage{graphicx}
\usepackage{tikz}
\usepackage{xspace}
\usepackage{float}
\usepackage{comment}

%%%% Packages needed by Matteo
\usepackage[all]{xy}
\usepackage{stmaryrd}
\usepackage{amssymb}
\usepackage{amsmath}
\usepackage{verbatim}


%%%% End of Packages needed by Matteo

\newcommand{\fun}[3]{\ensuremath{#1\colon #2 \to #3}}
\newcommand{\parfun}[3]{\ensuremath{#1\colon #2 \rightharpoonup #3}}

%%%%%%%%%%%%%%%%%%%%%%%%%%%%%%%%
%
%
% The record
%
%
%%%%%%%%%%%%%%%%%%%%%%%%%%%%%%%%

\newcommand{\therecord}{{$84$}\ }
\newcommand{\theoctagons}{{$126912$}\ }
\newcommand{\theinstances}{{$42889$}\ }

%%%%%%%%%%%%%%%%%%%%%%%%%%%%%%%%
%
%
% The biblatex requirements 
%
%
%%%%%%%%%%%%%%%%%%%%%%%%%%%%%%%%

\usepackage[backend=bibtex]{biblatex}
\bibliography{games.bib}

%%%%%%%%%%%%%%%%%%%%%%%%%%%%%%%
%
%
% Stuff needed for pictures
%
%
%%%%%%%%%%%%%%%%%%%%%%%%%%%%%%%%

%\usepackage[usenames,dvipsnames]{xcolor}
\usepackage{tikz}
\usepackage{xifthen}
\usepackage{intcalc}
\usetikzlibrary{calc}
\usetikzlibrary{arrows}
\usepackage{pgfplots}
\usepackage{wrapfig}
\usepackage[font={scriptsize,it}]{caption}
\usepackage{cutwin}
%\usepackage{picins}


%%%%%%%%%%%%%%%%%%%%%%%%%%%%%%%%
%
%
% Added for the sake of internal 
% references
%
%
%%%%%%%%%%%%%%%%%%%%%%%%%%%%%%%%

\usepackage{enumitem, hyperref}
\makeatletter
\def\namedlabel#1#2{\begingroup
    #2%
    \def\@currentlabel{#2}%
    \phantomsection\label{#1}\endgroup
}

%%%%%%%%%%%%%%%%%%%%%%%%%%%%%%%%
%
%
% Added to have a sane enumeration 
% of conditions in Lemma 4 (o1)-(o8)
%
%
%%%%%%%%%%%%%%%%%%%%%%%%%%%%%%%%

\usepackage{mathtools} 

%%%%%%%%%%%%%%%%%%%%%%%%%%%%%%%%
%
%
% Commands needed by Andrzej
%
%
%%%%%%%%%%%%%%%%%%%%%%%%%%%%%%%%


%\usepackage[english]{babel}
\usepackage[utf8]{inputenc}
\usepackage{graphicx}
%\usepackage{amsthm}
\usepackage{amsmath}
\usepackage[colorinlistoftodos]{todonotes}


\numberwithin{equation}{section}

%\newtheorem{theorem}{Theorem}[section]
%\newtheorem{lemma}[theorem]{Lemma}
%\newtheorem{corollary}[theorem]{Corollary}
%\newtheorem{proposition}[theorem]{Proposition}
%\newtheorem{conjecture}[theorem]{Conjecture}
%\newtheorem*{theorem*}{Theorem}

%\theoremstyle{definition}
%\newtheorem*{definition}{Definition}
%\newtheorem{example}[theorem]{Example}
\newtheorem{xca}[theorem]{Exercise}

%\theoremstyle{remark}
%\newtheorem{remark}[theorem]{Remark}
%\newtheorem{question}[theorem]{Question}
%\newtheorem*{remark*}{Remark}
\newtheorem*{convention*}{Convention}
\newtheorem*{notation*}{Notation}

\DeclareMathOperator{\hull}{hull}
\DeclareMathOperator{\external}{ex}
\DeclareMathOperator{\modifier}{modifier}
\DeclareMathOperator{\internal}{int}
\DeclareMathOperator{\gap}{gap}
\DeclareMathOperator{\param}{param}
\DeclareMathOperator{\bd}{\partial}
\DeclareMathOperator{\pot}{potential}
\DeclareMathOperator{\dt}{dot}
\DeclareMathOperator{\move}{mv}
\DeclareMathOperator{\obj}{obj}
\DeclareMathOperator{\oct}{octagon}


\renewcommand{\comment}[1]{}


\newcommand {\nat}{\mathbb{N}}
\newcommand {\scc}{\mathsf{scc}}
\newcommand {\ord}{\mathsf{ORD}}
\newcommand {\un}{\mathsf{un}}
\newcommand {\head}{\mathsf{head}}
\newcommand {\tail}{\mathsf{tail}}
\newcommand {\ad}{\mathsf{ad}}
\newcommand {\sub}{\mathsf{sub}}
\newcommand {\dom}{\mathsf{dom}}
\newcommand {\rank}{\mathsf{rank}\xspace}
\newcommand {\red}{\mathsf{red}}
\newcommand {\LGA}{\mathsf{LGA}}
\newcommand {\WAA}{\mathsf{WAA}}
\newcommand {\SAA}{\mathsf{SAA}}
\newcommand {\mybox}{\textrm{{\scriptsize $\,\Box\,$}}}
\newcommand {\R}{${\mathcal R}$}
\newcommand {\RR}{{\mathcal R}}
\newcommand {\C}{${\mathcal C}$}
\newcommand {\BC}{\mathsf{BC}}


\newcommand{\dotcup}{\ensuremath{\mathaccent\cdot\cup}}


% Letters and numbers

\newcommand{\mathromnum}[1]{\ensuremath{\mathrm{#1}}\xspace}

\newcommand{\rI}{\mathromnum{I}}
\newcommand{\rII}{\mathromnum{II}}
\newcommand{\rIII}{\mathromnum{III}}
\newcommand{\rIV}{\mathromnum{IV}}
\newcommand{\rV}{\mathromnum{V}}

\newcommand{\mathcalsym}[1]{\ensuremath{\mathcal{#1}}\xspace}

\newcommand{\Cross}{\texttt{Cross}}
\newcommand{\Aa}{\mathcalsym{A}}
\newcommand{\Bb}{\mathcalsym{B}}
\newcommand{\Cc}{\mathcalsym{C}}
\newcommand{\Dd}{\mathcalsym{D}}
\newcommand{\Ee}{\mathcalsym{E}}
\newcommand{\Ff}{\mathcalsym{F}}
\newcommand{\Gg}{\mathcalsym{G}}
\newcommand{\Hh}{\mathcalsym{H}}
\newcommand{\Ii}{\mathcalsym{I}}
\newcommand{\Jj}{\mathcalsym{J}}
\newcommand{\Kk}{\mathcalsym{K}}
\newcommand{\Ll}{\mathcalsym{L}}
\newcommand{\Mm}{\mathcalsym{M}}
\newcommand{\Nn}{\mathcalsym{N}}
\newcommand{\Oo}{\mathcalsym{O}}
\newcommand{\Pl}{\mathcalsym{P}}
\newcommand{\Ql}{\mathcalsym{Q}}
\newcommand{\Rr}{\mathcalsym{R}}
\newcommand{\Ss}{\mathcalsym{S}}
\newcommand{\Tt}{\mathcalsym{T}}
\newcommand{\Uu}{\mathcalsym{U}}
\newcommand{\Vv}{\mathcalsym{V}}
\newcommand{\Ww}{\mathcalsym{W}}
\newcommand{\Xx}{\mathcalsym{X}}
\newcommand{\Yy}{\mathcalsym{Y}}
\newcommand{\Zz}{\mathcalsym{Z}}
\newcommand{\W}{\mathrm{W}}
\newcommand{\coR}{\mbox{co-}\Rr}

%-------------------------------

\newcommand{\trees}{\mathrm{Tr}\xspace}
\newcommand{\runs}{\mathrm{Runs}\xspace}
\newcommand{\otrees}{\mathrm{Tr}(\omega)\xspace}
\newcommand{\WF}{\mathrm{WF}\xspace}


\newcommand{\boldclass}[3]{\ensuremath{\mathbf{#1}^{#2}_{#3}}}
\newcommand{\lightclass}[3]{\ensuremath{{#1}^{#2}_{#3}}}

\newcommand{\borel}{\ensuremath{\mathcal B}\xspace}

\newcommand{\bsigma}[1]{\boldclass{\Sigma}{0}{#1}}
\newcommand{\bpi}[1]{\boldclass{\Pi}{0}{#1}}
\newcommand{\bdelta}[1]{\boldclass{\Delta}{0}{#1}}

\newcommand{\asigma}[1]{\boldclass{\Sigma}{1}{#1}}
\newcommand{\api}[1]{\boldclass{\Pi}{1}{#1}}
\newcommand{\adelta}[1]{\boldclass{\Delta}{1}{#1}}

\newcommand{\esigma}[2]{\lightclass{\Sigma}{#1}{#2}}
\newcommand{\epi}[2]{\lightclass{\Pi}{1}{#1}}
\newcommand{\edelta}[2]{\lightclass{\Delta}{#1}{#2}}


\newcommand{\bc}[1]{\mathcal{BC}({#1})}

\newcommand{\textscsym}[1]{{\sc #1}}

\newcommand{\eqdef}{\stackrel{\mathrm{def}}=}

\newcommand{\N}{\nat}

%\newcommand{\dL}{\mathrm{L}}
%\newcommand{\dR}{\mathrm{R}}

\newcommand{\rmin}{i}
\newcommand{\rmax}{k}

\newcommand{\G}{\mathcal{G}}

\newcommand{\Plab}{{\mathrm{label}}}
\newcommand{\guarantee}{{\mathrm{promise}}}

\newcommand{\Pgar}{X}

\newcommand{\eve}{\ensuremath{\exists}\xspace}
\newcommand{\adam}{\ensuremath{\forall}\xspace}

\newcommand{\ZFC}{{\sc ZFC}}
\newcommand{\LTL}{{\sc LTL}}
\newcommand{\CTL}{{\sc CTL}}
\newcommand{\PCTL}{{\sc PCTL}}
\newcommand{\CTLstar}{{\sc LTL$^\ast$}}
\newcommand{\ZFCMA}{{\sc ZFC+MA$_{\aleph_{1}}$}}
\newcommand{\MA}{{\sc MA$_{\aleph_{1}}$}}
\newcommand{\ZFCVL}{{\sc ZFC+V=L}}
\newcommand{\VL}{{\sc V=L}}
\newcommand{\AD}{{\sc AD}}
\newcommand{\CH}{{\sc CH}}

\newcommand{\powerset}{{\mathcal P}}
\newcommand{\la}{{\langle}}
\newcommand{\ra}{{\rangle}}

\newcommand{\hargame}{{H\Gg}}

\newcommand{\restr}{\!\upharpoonright}

\newcommand{\ignore}[1]{}
\newcommand{\Souslin}{\mathcal A}


\newenvironment{proofof}[1]
	{\vspace{1ex}\noindent{\emph{Proof of #1}}\hspace{0.5em}}
    {\hfill\qed\vspace{1ex}}
    
%%%%%%%%%%%%%%%%%%%%%%%%%%%%%%%%%%%%
%
%
% acknowledgments on the front page
%
%
%%%%%%%%%%%%%%%%%%%%%%%%%%%%%%%%%%%%    

\newcommand*\samethanks[1][\value{footnote}]{\footnotemark[#1]}


\lstdefinestyle{myCustomPythonStyle}{
  language=Python,
  numbers=left,
  stepnumber=1,
  numbersep=10pt,
  tabsize=4,
  showspaces=false,
  showstringspaces=false,
  basicstyle=\ttfamily\small
}




% Author macros::begin %%%%%%%%%%%%%%%%%%%%%%%%%%%%%%%%%%%%%%%%%%%%%%%%
\title{{An upper bound of 84 for Morpion Solitaire 5D}}%\footnote{This work was partially supported by someone.}}
\titlerunning{A Morpion 5D bound} %optional, in case that the title is too long; the running title should fit into the top page column

\author[1]{Henryk Michalewski}
\author[1]{Andrzej Nagórko}
\author[1]{Jakub Pawlewicz}
\affil[1]{Department of Mathematics, Informatics and Mechanics\\ University of Warsaw\\ \{H.Michalewski,A.Nagorko,J.Pawlewicz\}@mimuw.edu.pl} %\\ \texttt{open@dummyuni.org}}
\authorrunning{H. Michalewski, A. Nagórko and J. Pawlewicz} %mandatory. First: Use abbreviated first/middle names. Second (only in severe cases): Use first author plus 'et. al.'

\Copyright{Henryk Michalewski, Andrzej Nagórko and Jakub Pawlewicz}%mandatory, please use full first names. LIPIcs license is "CC-BY";  http://creativecommons.org/licenses/by/3.0/

\subjclass{Dummy classification -- please refer to \url{http://www.acm.org/about/class/ccs98-html}}% mandatory: Please choose ACM 1998 classifications from http://www.acm.org/about/class/ccs98-html . E.g., cite as "F.1.1 Models of Computation". 
\keywords{Morpion, linear optmization, relaxation}% mandatory: Please provide 1-5 keywords
% Author macros::end %%%%%%%%%%%%%%%%%%%%%%%%%%%%%%%%%%%%%%%%%%%%%%%%%

%Editor-only macros:: begin (do not touch as author)%%%%%%%%%%%%%%%%%%%%%%%%%%%%%%%%%%
\serieslogo{}%please provide filename (without suffix)
\volumeinfo%(easychair interface)
  {Billy Editor and Bill Editors}% editors
  {2}% number of editors: 1, 2, ....
  {Conference title on which this volume is based on}% event
  {1}% volume
  {1}% issue
  {1}% starting page number
\EventShortName{}
\DOI{10.4230/LIPIcs.xxx.yyy.p}% to be completed by the volume editor
% Editor-only macros::end %%%%%%%%%%%%%%%%%%%%%%%%%%%%%%%%%%%%%%%%%%%%%%%

\begin{document}

\maketitle

\begin{abstract} 
The Morpion Solitaire is a paper-and-pencil single-player game played on a square grid with initial position consisting of $36$ dots.
In each move the player puts a dot on an unused grid position and draws a line that 
  consists of four consecutive segments passing through the dot.
 The goal is to find the longest possible sequence of moves.
There are two main variants of the game: 5T and 5D. 
They have different restrictions on how the moves may intersect.

Providing lower and upper bounds in Morpion Solitaire is a significant computational and mathematical challenge
  that led to a discovery of an important new algorithm.
 Best lower bounds for Morpion Solitaire were found in $2011$ by Rosin \cite{rosin} using a variant of the Monte Carlo method: 
 $178$ moves for the 5T variant, $82$ moves for the 5D variant.
 
% In the Morpion 5T variant an upper bound of $705$ was proved in $2006$ by Demaine at al. \cite{demaine}  using a method of potential. 
% A refinement of this approach was used in $2015$  in \cite{ijcai} to show a bound of $485$ - this method combined an isoperimetric inequality with linear programming. 

In the Morpion 5D variant 
Kawamura et al. proved in \cite{japonczycy} an upper bound of $121$ moves using a geometrical argument. 
We improve this bound to \therecord. 
This is done in two steps.
1) We state, using linear constraints, a geometric property that defines a class of graphs that includes all Morpion  positions - finding practically solvable linear constraints is a main conceptual advance of the present paper.
2) We solve the mixed integer problems and obtain 
an upper bound of \therecord for Morpion 5D. The same method a) fully solves Morpion 5D with center symmetry (the optimal sequence is $68$ moves long) and b) gives an upper bound of $222$ for Morpion 5T with center symmetry. 
\end{abstract}

\section{Introduction}
The Morpion Solitaire is a paper-and-pencil single-player game played on a square grid with 
  the initial configuration of 36 dots depicted in Figure~\ref{fig:initial}. 
In each move the player puts a dot on an unused grid position and draws a line that 
  consists of four consecutive segments passing through the dot. 
The line must be horizontal, vertical or diagonal. 
The goal is to find the longest possible sequence of moves.
There are two main variants of the game: 5T and 5D. 
They have different restrictions on how the moves may be placed.
In the 5T variant of the game, no segment may be drawn twice, i.e. the moves have to be segment-disjoint. 
In the 5D variant of the game, any two moves in the same direction have to be dot-disjoint.
The difference is demonstrated in Figure~\ref{fig:initial}.\todo{Complete the figure}
% The 5D variant is more restrictive. In Morpion 5D there has to be a gap between moves placed on a same line.

  \begin{figure}
    \centering
      \includegraphics[width=0.49\textwidth]{figures/empty.pdf}
      \includegraphics[width=0.49\textwidth]{figures/empty.pdf}
      \caption{\label{fig:initial}
	The initial position of Morpion Solitaire is depicted on the right. On the left there is a position which is up to $4$--th move legal 
both in Morpion 5D and Morpion 5T variants, but the $5$--th move is legal only in Morpion 5T. 
      }
\end{figure}

The problem is notoriously hard for computers. 
For $34$ years the longest known sequence in the Morpion 5T game
  was one of 170 moves discovered by C.-H. Bruneau in $1976$. 
Despite considerable computational effort, until $2010$ the computer generated
  sequences were much shorter. 
In $2010$ Rosin \cite{rosin} obtained the current world records of $178$ moves in Morpion 5T 
  and of $82$ moves in Morpion 5D using a specialized Monte Carlo algorithm. This work  was
   recognized as a best paper of the IJCAI conference in 2011. The webpage~\cite{boyer} maintained by Christian Boyer, contains an extensive and up-to-date information about records in all Morpion Solitaire variants.

\subsection{Morpion Solitaire and linear programming}
The rules of Morpion Solitaire do not limit the size of the grid on which the game is played, hence a priori
  it is not clear if the sequences have to be bounded.
%A popular magazine \emph{Science \& Vie} published in $1970$'s different bounds submitted by its readers for the maximal length of a sequence in Morpion 5T  (the bounds ranged from $540$ to $20736$), but without detailed and/or valid proofs.
%The first rigorously proved bound of $705$ in Morpion 5T was published in $2006$ \cite[Demaine et al.]{demaine}.
%The best known bound of $485$ in Morpion 5T was proved in~\cite{ijcai}.
%In this paper we prove an upper bound of $84$ on the length of Morpion 5D sequence, 
%improving upon a bound of $121$ found earlier by \cite[Kawamura et al.]{japonczycy}.
%\begin{comment}
%Proofs of upper bounds discussed above exploit geometric and combinatorial properties of graphs that are obtained as Morpion Solitaire positions. 
%To set the mood we'll discuss the bound of $705$ moves for Morpion 5T as it was proved in~\cite{demaine}.
%A position of a Morpion Solitaire gameplay (Figure~\ref{fig:small}) has a graph structure.
%Its vertices are placed in the $\mathbb{Z}^2$ grid.
%Let $n$ denote the number of vertices, 
%  corresponding both to the dots placed by moves and to the dots from the starting cross.
%The edges correspond to the segments placed on the grid by moves.
%Every move adds a single vertex and four edges to the graph.
%Therefore a Morpion position graph has the following two properties: 1) its edges have unit length in $\ell_\infty$ metric; 2) $4n - e = 4 \cdot 36 = 144$. \todo{I do not like the idea of putting here $n%%$, $e$, $l_\infty$ and basically I am afraid that someone less patient may stop reading here without getting to the main result - maybe we can make a picture instead of this?}
%In~\cite{brass} P. Brass proved that if a planar graph $G$ has $n$ vertices, 
%  then the maximum number of edges in $G$ that have unit length in $\ell_\infty$ metric is equal to
%  $
%    s(n) = \lfloor 4n - \sqrt{28n - 12} \rfloor.
%  $
%Hence
%$
%  4n - 144 = e \leq s(n) = \lfloor 4n - \sqrt{28n - 12} \rfloor.
%$
%The maximal $n$ that satisfies this inequality is $n = 741$. 
%Considering the initial $36$ dots this gives an upper bound of $705$ on the number of moves 
%  in Morpion Solitaire~\cite{demaine}.
%  
%Observe that the Morpion position graph has additional geometrical property.
%The set of its edges may be covered by a set of segment-disjoint \emph{moves} consisting of four consecutive, parallel, distinct unit length segments. 
%We call such graphs \emph{unmarked unordered Morpion graphs} (see Section~\ref{sec:linear} for formal definitions).
% In~\cite{ijcai} we used linear programming to show a bound of $485$ on the number of vertices in such a graph, under additional constraints about the size of its bounding box  that follow from rules of %%Morpion 5T and from a variant of an isoperimetric inequality. %\todo{The same stuff as one paragraph below}
%\end{comment}
 %\todo{Insert why the game is epxressible via a linear program}
However, as a single player game, Morpion Solitaire can be  encoded in a natural way as a linear optimization problem with the optimization target being the length of the sequence. On its own the encoding is not very helpful, because the problem is too large to be practically solvable. Nevertheless, this inspires a natural approach towards construction of upper bounds. Instead of solving the Morpion Solitaire, we solve
a more general game such that
\begin{itemize}
\item every gameplay of ordinary  Morpion Solitaire is a gameplay of the generalization,
\item the generalization is practically solvable and
\item an upper bound proved for this more general game is still interesting for the Morpion Solitaire. 
\end{itemize}
We do not expect that the more general game will be playable by humans. Here is a summary of results we managed to obtain using this approach:
\begin{itemize}
\item the %first rigorously proved 
bound of $705$ in Morpion 5T was proved in $2006$ in \cite[Demaine et al.]{demaine} using a careful geometric analysis of moves in Morpion 5T and a potential argument. The current best bound of $485$ in Morpion 5T was obtained in~\cite{ijcai} using an appropriate generalization of Morpion 5T. In particular in \cite{ijcai} it was shown that the bound of $586$ can be obtained via the relaxation of the linear program encoding the rules of  Morpion 5T;
\item a geometric analysis of Morpion 5D led in \cite[Kawamura et al.]{japonczycy} to an upper bound of $121$ which improved on an earlier bound of $144$ found in \cite{demaine} using the potential method; using a generalization of Morpion 5D, in this paper we prove an upper bound of $84$ on the length of Morpion 5D sequences;
%, improving upon a bound of $121$ found earlier by  using a geometrical analysis of moves in Morpion 5D. 
% will significantly improve on the above geometric bounds \cite[Kawamura et al.]{japonczycy};  
%A certain generlization of Morpion Solitaire 5T we analyzed in \cite{ijcai}, however this method 
the method used in  \cite{ijcai} does not seem very helpful in Morpion 5D --- later in this paper we discuss differences between these two generalizations. 
\end{itemize}

%\begin{comment}  
%In the present paper we shall use additional combinatorial property of Morpion graphs:
%one can find an assignment such that to each move is assigned one of its dots and the assignment is one-to-one (see Section \ref{sec:linear} for a precise definition).
%We call such graphs \emph{unordered Morpion graphs}. % (Figure~\ref{fig:85}).
%Every Morpion 5D gameplay generates such assignment but there are assignments 
%which are not generated by any gameplay (see example of such situation in Figure\ref{fig:85})
%\end{comment} 

%\todo{The following paragraph should be written in a different way --- emphasize Theorem, maybe Lemmas, the table} 
\subsection{Organization of the paper}
In Section \ref{sec:geometry} we formulate the main results:
\begin{itemize}
\item in Theorem \ref{thm:boxes} we show that every Morpion 5D gameplay must be contained in a relatively small rectangular box around the starting cross in Figure \ref{fig:initial}; this already proves the bound of $85$ in Morpion 5D,
\item in Corollary \ref{cor:84} we improve this bound to 84 through an analysis of $4$ special cases.
\end{itemize}
The proof of these results is divded between next three sections. In Section \ref{sec:linear} we define a useful generalization of Morpion 5D 
and encode it as a linear program. In Section \ref{sec:gemmating} we generate a list of rectangular boards (``boxes'') such that any position of morpion 5D is contained in one of the boxes --- this is achieved using an auxilliary linear program. In Section \ref{sec:upper} we solve 
the generalization of Morpion 5D on all boards found in Section \ref{sec:gemmating} and conclude the proof of Theorem \ref{thm:boxes} along with Corollary \ref{cor:84}. Moreover, we apply the same approach to symmetric variants of Morpion 5T and Morpion 5D and solve completely these
two games.

%We prove that the
%  size of graphs with these combinatorial and geometrical properties does not exceed $85$.
%We will also use additional constraint about a size of the bounding box of the graph that follows from
%  calculations combined with the rules of the game.
%We then use additional argument to show that the graphs of size $85$ do not correspond to Morpion positions.

%In~\cite{ijcai} we used isoperimetric inequality combined with mixed integer programming to obtain an upper bound of $485$ for Morpion 5T. 
%The method employed in~\cite{ijcai} does not give useful upper bound for the 5D variant.
%In the present paper we use a new technique of gemmating to limit the size of a bounding box of a Morpion 5D graph combined with a new reduction of Morpion Solitaire to a mixed integer programming to obtain a bound of $84$ for Morpion 5D. \todo{List precisely results which are important for this paper}

  
% !TEX root = morpion5d.tex

\section{Geometry of Morpion Solitaire positions}
\label{sec:geometry}

%\todo{An introductory sentence}
In this section we collect a number of observations regarding geometric properties of Morpion Solitaire positions and formulate the basic definitions as well as the main result of this paper. 
In order to make linear computations feasible, we will limit the size of boards to rectangular ``boxes''.  In Section \ref{sec:gemmating} we explain why this is enough to solve the general problem of bounding the sequences in Morpion 5D. 
\begin{definition}
A box is a rectangular subset of ${\mathbb Z}^2$ bounded by $4$ edges parallel to the coordinate axes. A bounding box of a Morpion 5D position is the smallest box
containing this position. %rectangular part of the grid that contains this position
%edges parallel to the coordinate axes). 
\todo{Add a precise figure with a bounding box and dimensions, e.g. Figure~\ref{fig:small}}
\end{definition}

Every Morpion 5D graph contains the initial cross, hence the smallest bounding box has dimensions $9 \times 9$.
The graph depicted in Figure~\ref{fig:85} has a bounding box with dimensions $14 \times 13$.
However, the dimensions of the bounding box do not give information about the position of the initial cross inside. Since
this information is important for computations, we introduce the following %, which is important.
%We will employ the following convention.

\begin{notation*}
The distance of a bounding box to the initial cross can be described by $4$ numbers: the distances of the sides of the box to the edges of the cross. 
% is distances of its edges from the edges of the cross.
For graph depicted in Figure~\ref{fig:85} the distance from top edge of the cross to the top side of the bounding box is equal to $3$. For the right side it is $4$, for the bottom side it is $1$ and for the left side it is also $1$. 
The bounding box of the position in Figure~\ref{fig:85} is described as $(3,4,1,1)$.
\end{notation*}
  
\begin{table}[ht]
\centering
%
    \begin{tabular}{|l|l|l|l|}
    \hline
    No &  Bounding box &  Max size  \\
    \hline%
    
1&(3, 4, 1, 1)& 85.0\\
2&(3, 4, 2, 1)& 85.0\\
3&(4, 3, 1, 2)& 85.0\\
4&(4, 3, 1, 3)& 85.0\\
5&(2, 4, 2, 1)& 84.0\\
6&(2, 4, 2, 2)& 84.0\\
7&(2, 5, 1, 2)& 84.0\\
8&(2, 5, 2, 1)& 84.0\\
9&(2, 5, 2, 2)& 84.0\\
10&(3, 3, 1, 2)& 84.0\\
11&(3, 3, 2, 2)& 84.0\\
12&(3, 4, 1, 2)& 84.0\\
13&(3, 4, 1, 3)& 84.0\\
14&(3, 4, 2, 2)& 84.0\\
15&(3, 4, 3, 2)& 84.0\\
16&(4, 3, 0, 2)& 84.0\\
%
    \hline
    \end{tabular}%
\hspace*{5mm}
%
    \begin{tabular}{|l|l|l|l|}
    \hline
    No &  Bounding box &  Max size  \\
    \hline%
    
17&(4, 3, 2, 3)& 84.0\\
18&(2, 3, 2, 1)& 83.0\\
19&(2, 3, 2, 2)& 83.0\\
20&(2, 5, 1, 1)& 83.0\\
21&(3, 2, 1, 2)& 83.0\\
22&(3, 3, 1, 3)& 83.0\\
23&(3, 3, 3, 3)& 83.0\\
24&(3, 4, 2, 3)& 83.0\\
25&(3, 5, 1, 1)& 83.0\\
26&(3, 5, 2, 1)& 83.0\\
27&(4, 3, 0, 3)& 83.0\\
28&(4, 4, 0, 1)& 83.0\\
29&(4, 4, 0, 2)& 83.0\\
30&(4, 4, 1, 1)& 83.0\\
31&(4, 4, 1, 2)& 83.0\\
32&(4, 4, 1, 3)& 83.0\\
33&(4, 5, 1, 2)& 83.0\\
%
    \hline
    \end{tabular}%
 

\caption{Bounding boxes mentioned in Theorem \ref{thm:boxes}.\ref{thm:boxes:list} that admit Morpion 5D graphs of sizes $85$, $84$ and $83$. }
\label{tbl:boundingboxes}
\end{table}

Below we formulate the key theorem in this paper: %  \todo{Theorem, proved in Section 4, why is it important}
\begin{theorem}
\begin{enumerate}
\item Up to a symmetry every Morpion 5D {\em position} is contained in one of the following boxes:
$(4, 3, 3, 3)$, $(4, 4, 3, 2)$, $(5, 3, 2, 3)$, $(4, 3, 4, 2)$, $(5, 3, 3, 2)$, $(5, 4, 2, 1)$, 
$(6, 2, 2, 1)$, $(5, 4, 0, 2)$, $(5, 1, 4, 0)$. \label{thm:boxes:list}
\item Every box contained in one of the nine boxes listed in 1, with the exception of boxes $(6, 2, 2, 0)$ 
  and $(6, 0, 2, 0)$, is a bounding box of a Morpion 5D {\em graph}.
\item For each box $\mathcal{B}$ described in 2, the size of a maximal Morpion 5D graph with the bounding box $\mathcal{B}$ does not exceed $85$.
  Bounding boxes that admit Morpion 5D graphs of sizes greater than $82$ are listed 
    in Table~\ref{tbl:boundingboxes}. 
% All other boxes described in 2 are bounding boxes of graphs with size at most $82$.
\end{enumerate} 
\label{thm:boxes}
\end{theorem}
We will present a proof of this theorem in the next three sections with a summary of the argument in Subsection \ref{subsec:84}. 
In Section \ref{sec:linear} we precisely define the notion of a Morpion 5D graph. Intuitively, Morpion 5D graphs are obtained from Morpion 5D positions by forgetting about the order in which the moves were played.


%We will consider a class of graphs that are obtained from Morpion 5D positions by forgetting about 
%  the order in which the moves were played. %\todo{Definition of an unordered Morpion 5D graph}


%We will prove using linear programming that that the maximum size of  an unordered Morpion 5D graph with a bounding box equal to one of the bounding boxes listed in Table~\ref{tbl:boundingboxes}
% is $85$. \todo{This looks like a comment which should immediatelly after formulation of the theorem: it says, that we are not dealing
% with all morpion 5d graphs, but only with graphs generated by Morpion 5D positions - it is enough for the corollary that 85 is an upper bound -
% later there will be one more corollary improving on this bound to 84}

\begin{corollary}
\label{cor:84}
The longest sequence of moves in Morpion 5D does not exceed $84$.
\end{corollary}
\begin{proof} 
In Table \ref{tbl:boundingboxes} there are four bounding boxes with maximal graphs of size $85$. 
For these we formulate mixed integer problems with additional constraints that force the graph to
  be a Morpion 5D position.
These problems are much harder to solve, but with correct choice of solver optimization parameters we are able to show
  that there are no solutions of these problems of size $85$. Equipped with additional definitions and lemmas formulated later in this paper, in Subsection \ref{subsec:84} we add some technical 
information to this proof. 
\end{proof}


% !TEX root = main.tex

\section{Linear Relaxation}
\label{linear}

% Lattice graph, move.

A \emph{lattice point} on a plane is a point with integer coordinates. A \emph{lattice graph} is a graph with vertices in lattice points and edges consisting of pairs $(p,q)$, where $p$ and $q$ are two different neighboring points, that is $p\neq q$ and $p=(n,m)$ and $q=(n\pm i,m\pm j)$ for some $i,j=0,1$. %Notice, that edges are congruent to segments from the point $(0,0)$ to one of the points $(0, 1), (0, 1), (1, 1), (1, 0)$.
We call such edges the \emph{lattice edges}.

% Morpion graph.

A \emph{move} in a lattice graph $G = (V, E)$ is a set of four consecutive parallel lattice edges. 
%The four edges must be horizontal, vertical or diagonal. 
We let $\mathcal{M}(G)$ to be the set of all moves in a graph $G$.
We start with the following observation, which simply rephrases the rules of Morpion 5T++ formulated in the Introduction. 

\begin{lemma}%\todo{Optional: change this into a definition and just remark, that incidentally these are Morpion 5++ positions. This would save us a bit of work related to explanation of Morption 5++ tules.}
\label{lem:char5pluplus}
A graph $G = (V, E)$ is a Morpion 5T++ position graph if and only if it satisfies the following conditions
  \begin{enumerate}
  \item[\namedlabel{m1}{(M1)}] $G$ is a lattice graph,
  \item[\namedlabel{m2}{(M2)}] $4 \cdot \# V - \# E = 144$,
  \item[\namedlabel{m3}{(M3)}] The set $E$ of edges of $G$ can be decomposed into a collection of disjoint moves.
  \end{enumerate}
\end{lemma}
%\begin{proof}
%  basically induction on the number of moves (but we have a stash of unused dots)
%\end{proof}

% Linear constraints on 5++ position.

Let $B = (V_B, E_B)$ be a fixed lattice graph that we shall call \emph{the board}. 
In applications, it will be a sufficiently large octagonal lattice graph with a full set of edges.
Below we define linear constraints that describe all subgraphs of $B$ that satisfy conditions \ref{m1}--- \ref{m3} of Lemma~\ref{lem:char5pluplus}.

We define the following set of structural {\em binary} variables, that is variables assuming values $0,1$:
\[
  \tag{LP1}
  \{ \dt_v \colon v \in V_B \} \cup \{ \move_m \colon m \in \mathcal{M}(B) \}.
  \label{lp1}
\]

\noindent
For each $e \in E_B$ and $v \in e$ we declare the following constraints:
\begin{equation}
  \tag{LP2}  \sum_{ m \in \mathcal{M}(B), e \in m } \move_m \leq \dt_v.
  \label{lp2}
\end{equation}

\begin{equation}
  \tag{LP3} \sum_{v \in V_B} \dt_v = 36 + \sum_{m \in \mathcal{M}(B)} \move_m. 
  \label{lp3}
\end{equation}

The following two lemmas describe correspondence between binary-valued solutions of a mixed integer programming problem (\ref{lp1}) - (\ref{lp3}) and subgraphs of $B$ that are Morpion 5T++ positions.

\begin{lemma} Let $G=(V_G,E_G)$ be a subgraph of $B$ and a Morpion 5T++ position obtained by a sequence $\mathcal{M}$ of moves. If
\[
  \dt_v = \left\{ 
    \begin{array}{ll}
      0 & \text{ if } v \not\in V_G \\
      1 & \text{ if } v \in V_G
    \end{array}
  \right.
    \text{ and }
  \move_m = \left\{ 
    \begin{array}{ll}
      0 & \text{ if } v \not\in \mathcal{M} \\
      1 & \text{ if } v \in \mathcal{M}
    \end{array}
  \right.,
\]
then conditions (\ref{lp1}), (\ref{lp2}) and (\ref{lp3}) hold. 
\end{lemma}

\begin{proof}
If $\dt_v=0$, then there is no move passing through $v$, hence the left hand side of (\ref{lp2}) is equal to $0$. If $\dt_v=1$, then  condition (\ref{lp2}) means that every segment $e$ played in the game can appear in exactly one move.  Condition (\ref{lp3}) means that the number of dots placed is higher by 36 than the number of moves made.
\end{proof}

\begin{lemma} Assume that a set of variables defined by condition (\ref{lp1}) satisfies conditions (\ref{lp2}) and (\ref{lp3}).
Let $G = (V_G, E_G)$ be a graph with a set of vertices
\[
  V_G = \{ v \in V_B \colon \dt_v = 1 \}
\]
and a set of edges
\[
  E_G = \{ e \in E_B \colon \exists_{m \in \mathcal{M}(B)}\ e \in m, \move_m = 1 \}.
\]
Then $G$ is a Morpion 5T++ position and a subgraph of $G$. 
\end{lemma}
\begin{proof}
%Assume that we have a feasible binary-valued solution of the linear programming problem, that is
% we have $\dt_v, \move_m \in \{ 0, 1 \}$ such that \ref{lp1}--\ref{lp3}  holds.
We will show that $G$ satisfies conditions \ref{m1} --- \ref{m3} of Lemma~\ref{lem:char5pluplus}.

By the definition of $E_G$, if $e \in E_G$ then there exists $m \in \mathcal{M}(B)$ such that $\move_m = 1$.
By (\ref{lp2}), if $\move_m = 1$, then $\dt_v = 1$ for each $v \in V_B$ such that $v \in e \in m$. It means that graph $G$ contains vertices of its edges, therefore it is a well defined subgraph of $B$, hence it is a lattice graph and it satisfies \ref{m1}. 

From (\ref{lp2}) follows, that the moves $\move_m$ must be disjoint in the sense, that they cannot contain the same edge twice. This implies condition \ref{m3} of Lemma \ref{lem:char5pluplus}. From disjointness and condition (\ref{lp3}) follows condition \ref{m2} of Lemma \ref{lem:char5pluplus}.
\end{proof}

% LPP

We consider a linear relaxation of the MIP problem (\ref{lp1})---(\ref{lp3}). We let structural variables to be real-valued, subject to bounds
\begin{equation}
  \tag{LP4} 0 \leq \dt_v, \move_m \leq 1.
  \label{lp4}
\end{equation}
\noindent
In the relaxation we maximize the objective function 
\begin{equation}
  \tag{LP0} \sum_{m \in \mathcal{M}(B)}  \move_m  % \rightarrow max.
  \label{lp0}
\end{equation}%\todo{Shouln't it be formulated before the Lemma? Otherwise what is optimized there?}
Clearly, an optimal solution to the linear programming problem (LP0) - (LP4) gives an upper bound for the length of a Morpion 5T++ game on a board $B$.

\subsection{The problem of the infinite grid}
\label{inf_grid}

Observe that any lattice graph that consists of $9$ vertex-disjoint moves has $45$ vertices and $36$ edges and satisfies conditions \ref{m1} --- \ref{m3} of Lemma~\ref{lem:char5pluplus}, hence it is a Morpion 5T++ position graph and consequently Morpion 5T++ positions can have arbitrarily large diameter in the plane ${\mathbb R}^2$.

The following table summarizes solutions of the linear relaxation of Morpion 5T++ on square $n \times n$ boards (where $n$ is the number of edges on the side).% with $n = 10, \ldots, 100$.
%\todo{Tell about the reference machine.}
\begin{figure}[H]

\bgroup
\def\arraystretch{1.5}
\setlength\tabcolsep{1mm}
\begin{tabular}{|c|c|c|c|c|c|c|c|c|c|}
\hline
10 & 20 & 30 & 40 & 50 & 60 & 70 & 80 & 90 & 100 \\
\hline
64.00 & 278.50 & 619.53 & 876.55 & 1130.01 & 1387.54 & 1641.74 & 1898.13 & 2152.86 & 2408.54 \\
\hline
\end{tabular}
\egroup
\caption{The top row contains the length $n$ of the edge of a given square and the bottom row contains solutions to the relaxed problem (\ref{lp0})---(\ref{lp4}) on the $n\times n$ board. }
\end{figure}

We do not know whether the objective function (\ref{lp0}) is bounded or not on the infinite grid. However, the bound of $705$ moves derived in \cite{demaine} %from Brass formula~\cite{brass}\todo{Brass formula - maybe not such great idea to mention it here, unless we elaborate on this formula} 
holds for Morpion 5T++. 
This shows that we get no useful upper bound for positions satisfying \ref{m1} --- \ref{m3} using our linear relaxation method. To get a bound, we have to use another properties of Morpion 5T positions to bound the size of the board. This will be done in the next Section.%\todo[inline]{(Henryk) I added one paragraph at the end of introduction trying to explain the above and without a reference to brass. Here I would focus on explaining what is in this table.}



  
\section{Gemmating process}
\label{sec:gemmating}

In order to control number of involved cases which must be solved by the linear solevr 
we:\begin{itemize}
\item consider bounding boxes up to symmetries and 
\item make sure that the number of considered bounding boxes is close to the minimum.
\end{itemize}

\subsection{Symmetries}
The $8$-element group of isometries of the plane generated by 
\begin{itemize}
\item reflections with respect to the axes and 
\item rotations around the center of the $\Cross$ through the straight angle,
\end{itemize}
 %There is an $8$-element group of isometries of the $\mathbb{Z}^2$ grid that 
leave the $\Cross$ in place. 
\begin{lemma}
Up to these $8$ symmetries every Morpion 5D position is contained in a bounding box $(a,b,c,d)$ such that
\begin{itemize}
\item $a$ is the maximum of $a,b,c,d$,
\item $d\leq b$,
\item if the maximum appears twice in $a,b,c,d$ and $a=b$ then we require that $c\leq d$. 
\end{itemize}
\end{lemma}

\begin{proof}
Let us recall that $a,b,c,d$ are responsible for the top, left, bottom and right dimensions of the bounding box. 
We have to prove that every box $(a,b,c,d)$ through symmetries can be turned into one of the above boxes. 
In Appendix \ref{app:symmetry} we include a pseudocode finding the approriate box and here we explain how it works. 

First we make sure that $a$ is the maximum 
of $a,b,c,d$ via rotations. In order to satisfy the requirement $d\leq b$ we apply a symmetry with respect to the vertical axis. 
Assume now that $a=b$. In order to satisfy the requirements  $c\leq d$ we apply a  reflection with respect to the line $y=x$.\todo{Andrzej, please have a look at this "proof"}
%  $a \leq b,c,d$ and $b \leq d$. 
\end{proof}


\subsection{Resizing of boxes}
Starting from the smallest box containing the $\Cross$, we generate a list of of all boxes relevant for 
Morpion 5D gameplays. That is, iteratively we check using a linear solver whether the box can be extended. 
The iterative step is summarized in the following code.
\begin{lstlisting}[language = Python,
  basicstyle=\ttfamily\scriptsize,keywordstyle=\color{red},backgroundcolor=\color{white}]
while unsolved:
    box = unsolved.pop()
    # we use a linear solver to establish if a given box can extended
    result = solve(box)         
    solved.append(box)

    if result.bound > bound:
        bound = result.bound
        
    if result.type == "FEASIBLE":
        [ a, b, c, d ] = box

        # we add four potential new boxes to the list of boxes which should be analyzed
        gemmate = [ [ a+1, b, c, d], [a, b+1, c, d], [a, b, c+1, d], [a, b, c, d+1],
                    [ a+1,b+1, c,d], [a, b+1,c+1,d], [a,b, c+1,d+1], [a+1,b,c, d+1] ]
        
        # ... and eliminate these casses which were 
        #          - already solved or 
        #          - are already present on the stack.
        # this process is done up to symmetry 
        # the symmetry code is the Appendix             
        for g in gemmate:
            if symmetryClass(g) not in solved + unsolved:
                unsolved.append(symmetryClass(g))
\end{lstlisting}

\begin{lemma}
Every graph of a Morpion 5D position fits into one of the bounding boxes found by the resizing procedure.   
\end{lemma}

\begin{proof}
Let us remark that this is a special property of Morpion 5D positions. In general, we do not expect Morpion 5D graphs to fit into one of the boxes, however the argument presented 
in \cite{demaine} using the potential method applies Morpion 5D graphs (even to relaxations in the sense of Lemma \todo{add Lemma's number}), hence we know a priort that Morpion 5D graphs
does not exceed 144 vertices. 

However, in this Lemma we specifically use information about the structure of a move in Morpion 5D. Namely, in the linear program mentioned in the procedure we specifically require that there 
exists a legal move of Morpion 5D crossing the boundary of the box. If it is the case, then we add $8$ new bounding boxes to the stack of unprocessed boxes. Since every
Morpion 5D position results from extending a given position by a legal move, this guarantees that we can fit such position in one of the generated boxes. 
\todo{Andrzej, please have a look at this "proof"}
\end{proof} 
%  infeasible models, 
%  algorithm pseudo-code (python ?),



% !TEX root = morpion5d.tex

\section{Upper bounds}
\label{sec:upper}

%  discussion of results
  
\subsection{An upper bound of \therecord for Morpion 5D}
\label{subsec:84}

% We summarize arguments which amounts to the proof of main Theorem \ref{thm:boxes}
% \begin{proof}
% We use Lemma about resizing and then Lemma about coding as linear programs. 
% \end{proof}
In order to conclude the proof of Theorem \ref{thm:boxes} we apply the process described in Section \ref{sec:gemmating} to produce all bounding boxes relevant 
to Morpion 5D game. We solve the linear problem \L{1}-\L{5} specified in Definition \ref{def:mip} in all these boxes. Thanks to 
Lemmas \ref{lem:solutions} and \ref{lem:graphs}, the solutions of linear programs are in one-to-one correspondence with existence of
Morpion 5D graphs. This concludes the proof of Theorem \ref{thm:boxes}.


Theorem \ref{thm:boxes} implies the upper bound of $85$ which follows from solving the optimization
problem \L{1}-\L{5} on all boxes generated in Section \ref{sec:gemmating}. 
In order to improve the result to $84$ we have to consider three bounding boxes which admit marked Morpion 5D graphs of size $85$. These
are specifically boxes $(4,3,1,1)$, $(4,3,1,2)$, $(4,3,1,3)$.
On these three boards %we solve a more difficult problem with the cutoff condition set to $84.9$ --- 
not only we require that a given solution is a 
Morpion 5D graph (conditions \L{1}-\L{5} of Definition \ref{def:mip}), but additionally we also require that the graph conforms to condition \L{7} of Definition \ref{def:mip}. 
By Lemmas \ref{lem:solutions} and \ref{lem:graphs} this means that we try to solve the full Morpion 5D game on the bounding boxes  $(4,3,1,1)$, $(4,3,1,2)$, $(4,3,1,3)$.
In general it does not seem to be a feasible task, but we succeeded to complete the computations with the cutoff condition of $84.9$.
%\todo{Explain what is acyclity - covered in a Lemma}

\subsection{A solution of Morpion 5D with center symmetry}
We consider Morpion 5D positions which are invariant with respect to $8$ symmetries 
specified in Subsection \ref{subsec:symmetries}. We say that a Morpion 5D position or a graph is {\em symmetric} if it is invariant with respect to these symmetries.  
In analogy to Theorem \ref{thm:boxes} we can formulate the following result. 
\begin{theorem}
\begin{enumerate}
\item Up to symmetry every symmetric Morpion 5D {\em position} is contained in box
$(4, 2, 4, 2)$ or box $(3, 3, 3, 3)$.
\item Every box contained in one of the boxes $(4, 2, 4, 2)$, $(3, 3, 3, 3)$ is a bounding box of a symmetric Morpion 5D {\em graph}.
\item For each box described in $2$, the size of a maximal Morpion 5D graph is $76$ for boxes $(2, 1, 2, 1), (2, 2, 2, 2)$; $74$ for boxes $(4,2,4,2)$, $(4,1,4,1)$, $(3,2,3,2)$, $(3,1,3,1)$; $72$ for boxes $(1,1,1,1)$, $(1,0,1,0)$ and $68$ or less for other boxes.
\item For each box described in $3$, the size of a maximal Morpion 5D position is $68$. \label{thm:sym_boxes:four}
\end{enumerate} 
\label{thm:sym_boxes}
\end{theorem}

The proof is analogous to the proof of Theorem \ref{thm:boxes}. In the proof we additionally use the condition \L{6} of Definition \ref{def:mip}. To prove Theorem \ref{thm:sym_boxes}.\ref{thm:sym_boxes:four}, we use the condition \L{7}. In order to speed up computations we used an additional optimization: instead of solving the game on boxes, we solve it on slightly smaller octagons. This generates more cases, because we resize octagons in $8$ and not $4$ directions. However, the size of octagons is growing slower, hence the problems are easier to solve.  

\begin{corollary}
\label{cor:68}
The longest sequence of moves leading to a symmetric Morpion 5D position is equal to $68$.
\end{corollary}
\begin{proof} 
The upper bound follows from Theorem \ref{thm:sym_boxes}.
A record consisting of $68$ moves was found by M.~Quist and is presented in \cite{boyer}. % \todo{It may be worthwhile to reproduce it and comment can it be found by a computer}
\end{proof}
We are not aware of any previous upper bounds for symmetric Morpion 5D or Morpion 5T.



\section{Conclusion}

\subsection{Computations}
The total time to solve all involved linear problems appearing in Table \ref{tbl:boundingboxes} amounted to approximately 3000 hours on a single core of a 
Linux machine equipped with {\tt Intel\textsuperscript{\textregistered}} {\tt Xeon\textsuperscript{\textregistered}} {\tt CPU 5620@2.40GHz} with 24GB of RAM. 
We used \cite{gurobi} linear optimization software. The code generating the linear programs is available in the repository \cite{thewebpage}. 

\subsection{Further research}
The results presented in this paper are based on a specific generalization of Morpion 5D game. Using this generalization the 
task of reducing the bound from $84$ down to $82$ seems to be feasible, but requires significant computational effort. It would be interesting
to see another generalization which is less costly in terms of required computations, that is avoids the condition \L{7} in favor of another, easier to compute 
requirement. 

The same method can be used to show an upper bound in the symmetric Morpion 5T. The method used in \cite{ijcai} leads to the bound of $485$ in arbitrary Morpion 5T game. 
A record of 136 moves in symmetric Morpion 5T was found by M.~Quist and is presented in \cite{boyer}. We conjecture that an upper bound of approximately 200 moves can be found
using methods presented in this paper.  


%\todo{nrpa on small boards? remark about octagonal boards}


\printbibliography 

\appendix

\section{Symmetry}
\label{app:symmetry}

\todo{As it is written now it does not make any good - we cannot expect that anyone will try to guess what this code is doing}
\begin{lstlisting}[language = Python,
  basicstyle=\ttfamily\scriptsize,keywordstyle=\color{red},backgroundcolor=\color{white}]
def symmetryClass(b):
    if b[0] == b[1] and b[1] == b[2] and b[2] == b[3]:
        return b
    
    while b[0] < b[1] or b[0] < b[2] or b[0] < b[3]:
        b = b[1:] + b[:1]
    
    if (b[1] < b[3]):
        b[1], b[3] = b[3], b[1]

    if (b[1] < b[0] and b[2] < b[0] and b[3] < b[0]):
        return b
        
    if b[0] == b[3] or b[0] == b[1]:
        while not b[1] == b[0]:
            b = b[1:] + b[:1]
        if (b[2] < b[3]):
            b[2], b[3] = b[3], b[2]
    return b
\end{lstlisting}

    
\end{document}

